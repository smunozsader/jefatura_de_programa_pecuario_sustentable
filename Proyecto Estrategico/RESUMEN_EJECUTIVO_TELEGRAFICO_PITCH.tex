\documentclass[12pt,letterpaper]{article}
\usepackage[utf8]{inputenc}
\usepackage[spanish,mexico]{babel}
\usepackage[left=2.5cm,right=2.5cm,top=2.5cm,bottom=2.5cm,headheight=20pt]{geometry}
\usepackage{graphicx}
\usepackage{fancyhdr}
\usepackage{setspace}
\usepackage{lastpage}
\usepackage{parskip}
\usepackage{booktabs}
\usepackage{array}
\usepackage{multirow}
\usepackage{float}
\usepackage{xcolor}
\usepackage{colortbl}
\usepackage{amsmath}

% Define SADER colors
\definecolor{sadergreen}{RGB}{0,102,51}
\definecolor{saderverde}{RGB}{0,102,51}
\definecolor{saderred}{RGB}{180,0,0}
\definecolor{sadergris}{RGB}{80,80,80}
\definecolor{sadergold}{RGB}{204,153,0}
\definecolor{saderblue}{RGB}{0,51,102}

% Header and footer
\pagestyle{fancy}
\fancyhf{}
\fancyhead[C]{\textcolor{sadergris}{\small PITCH EJECUTIVO - RENACIMIENTO GANADERO MAYA 2026-2030}}
\fancyfoot[C]{\textcolor{sadergris}{\small Página \thepage\ de \pageref{LastPage}}}
\renewcommand{\headrulewidth}{0.4pt}
\renewcommand{\footrulewidth}{0pt}

\begin{document}

% ========================================
% PORTADA PITCH
% ========================================
\begin{titlepage}
\centering
\vspace*{0.5cm}

{\LARGE\bfseries\color{sadergreen} PITCH EJECUTIVO}\\[0.3cm]
{\Large\bfseries MACROPROYECTO ESTRATÉGICO}\\[0.2cm]
{\large\bfseries Renacimiento Ganadero Maya}\\[0.2cm]
{\normalsize 2026-2030}\\[0.8cm]

\includegraphics[width=0.22\textwidth]{logo_sader.png}\\[0.6cm]

{\normalsize\bfseries Secretaría de Agricultura y Desarrollo Rural}\\[0.2cm]
{\small Oficina de Representación en la Entidad Federativa Yucatán}\\[0.8cm]

{\large\textbf{Inversión Total Optimizada:}}\\[0.2cm]
{\Large\bfseries\color{sadergreen} \$1,052.0 MDP}\\[0.3cm]
{\small Esquema Híbrido: \$921.2M Subsidio + \$166.7M Crédito}\\[0.8cm]

{\normalsize\textbf{6 Componentes Integrados | 1,820 Beneficiarios}}\\[0.3cm]
{\small 
• SSPi: 6,000 ha transformadas\\
• Repoblamiento: 12,000 vaquillas F1\\
• Centro Genético: 120,000 dosis/año\\
• Lechería: 75 UPP tecnificadas\\
• Meliponicultura: 500 productores\\
• Plataforma Digital: Trazabilidad T-MEC
}\\[0.8cm]

{\small Diciembre 2025}

\end{titlepage}

% ========================================
% CONTENIDO TELEGRÁFICO
% ========================================

\section{El Desafío}

\textbf{Situación Actual Crítica:}
\begin{itemize}
    \item \textbf{Inventario SIAP 2023:} 605,536 cabezas bovinas
    \item \textbf{Sector lechero:} -35.7\% última década
    \item \textbf{Productividad:} 60\% por debajo del potencial
    \item \textbf{Vulnerabilidad climática:} 60\% superficie afectada
\end{itemize}

\section{La Oportunidad}

\textbf{Convergencia Estratégica:}
\begin{itemize}
    \item \textbf{T-MEC:} Acceso preferencial mercados EE.UU./Canadá
    \item \textbf{PEF 2026:} \$18,500 MDP etiquetados ganadería sustentable
    \item \textbf{Plan Maya:} Directriz 4.1.1 modernización sectorial
    \item \textbf{Ventana única:} Convergencia políticas + recursos + mercados
\end{itemize}

\section{Indicadores de Impacto Clave}

\begin{table}[H]
\centering
\footnotesize
\begin{tabular}{|p{7cm}|c|}
\hline
\rowcolor{sadergreen!20}
\textbf{Indicador} & \textbf{Meta 2030} \\
\hline
\textbf{Beneficiarios Directos} & 1,820 (1,320 UPP + 500 meliponicultores) \\
\hline
\textbf{Incremento Productividad} & +388\% (becerros al destete) \\
\hline
\textbf{Captura CO\textsubscript{2}} & 765,000 ton equivalente \\
\hline
\textbf{Exportaciones Anuales} & \$150+ millones USD \\
\hline
\textbf{Recuperación Inventario} & +400,000 cabezas bovinas \\
\hline
\textbf{Inclusión de Género} & 35\% mujeres ($\geq$350 productoras) \\
\hline
\textbf{Participación Juvenil} & 23\% jóvenes rurales \\
\hline
\rowcolor{sadergold!20}
\textbf{ROI Proyectado} & \textbf{TIR 18.5\% | VAN \$1,245 MDP} \\
\hline
\end{tabular}
\end{table}

\section{Los 6 Componentes Estratégicos}

\subsection{1. Sistemas Silvopastoriles Intensivos (SSPi)}
\textbf{Inversión:} \$333.4 MDP + \$166.7M crédito\\
\textbf{Objetivo:} 6,000 ha | 120 UPP | \$55,573/ha\\
\textbf{Resultados:} +1,167\% productividad | 2.8 UA/ha | Ratio pago 4.0:1

\subsection{2. Repoblamiento Ganadero Bovino}
\textbf{Inversión:} \$150.1 MDP\\
\textbf{Objetivo:} 12,000 vaquillas F1 | 1,075 UPP\\
\textbf{Resultados:} Cruces Bos indicus × Bos taurus | Trazabilidad SINIIGA

\subsection{3. Centro de Mejoramiento Genético (Tizimín)}
\textbf{Inversión:} \$150.0 MDP\\
\textbf{Objetivo:} 120,000 dosis/año | 880 UPP atendidas\\
\textbf{Resultados:} ISO/IEC 17025:2017 | Acreditación OIE | Evaluaciones genómicas

\subsection{4. Desarrollo Lechero Tropical}
\textbf{Inversión:} \$89.5 MDP\\
\textbf{Objetivo:} 75 UPP tecnificadas | +40\% producción\\
\textbf{Resultados:} 750 vaquillas Gyrolando | SSPi lecheros | Cadena frío

\subsection{5. Meliponicultura Sustentable Maya}
\textbf{Inversión:} \$42.5 MDP\\
\textbf{Objetivo:} 500 productores (350 mujeres) | 6 ton/año\\
\textbf{Resultados:} Xunan Kab (\textit{Melipona beecheii}) | Mercados premium | Identidad cultural

\subsection{6. Plataforma Digital Sanitaria}
\textbf{Inversión:} \$8.5 MDP\\
\textbf{Objetivo:} 1,820 beneficiarios | Sistema CESO\\
\textbf{Resultados:} Validación APHIS-USDA | Trazabilidad T-MEC | Anti-abigeato

\section{Presupuesto Consolidado}

\begin{table}[H]
\centering
\footnotesize
\begin{tabular}{|p{6cm}|r|c|}
\hline
\rowcolor{sadergreen!20}
\textbf{Componente} & \textbf{MDP} & \textbf{\%} \\
\hline
\textbf{1. SSPi (6,000 ha)} & \$333.4 & 30.6\% \\
\textbf{2. Repoblamiento (12K F1)} & \$150.1 & 13.8\% \\
\textbf{3. Centro Genético} & \$150.0 & 13.8\% \\
\textbf{4. Lechería Tropical} & \$89.5 & 8.2\% \\
\textbf{5. Meliponicultura} & \$42.5 & 3.9\% \\
\textbf{6. Plataforma Digital} & \$8.5 & 0.8\% \\
\hline
\rowcolor{sadergold!20}
\textbf{SUBTOTAL PRODUCTIVO} & \textbf{\$774.0} & \textbf{71.1\%} \\
\hline
\textbf{Infraestructura Ganadera} & \$60.0 & 5.5\% \\
\textbf{Crédito SSPi (50\%)} & \$166.7 & 15.3\% \\
\textbf{Equipo Técnico (5 años)} & \$16.9 & 1.6\% \\
\hline
\rowcolor{sadergreen!30}
\textbf{GRAN TOTAL} & \textbf{\$1,052.0} & \textbf{100.0\%} \\
\hline
\end{tabular}
\end{table}

\textbf{Esquema Financiero:}
\begin{itemize}
    \item \textbf{Federal (60\%):} \$552.7 MDP via FOFAY
    \item \textbf{Estatal (30\%):} \$276.4 MDP 
    \item \textbf{Productores (10\%):} \$92.1 MDP
    \item \textbf{Crédito Productivo:} \$166.7 MDP (ratio 4.0:1)
\end{itemize}

\section{Impactos Proyectados}

\subsection{Económico}
\begin{itemize}
    \item \textbf{PIB agropecuario:} +\$2,500 MDP anuales (2030)
    \item \textbf{Empleos directos:} 1,820 permanentes
    \item \textbf{Empleos indirectos:} 5,460 en cadena extendida
    \item \textbf{Retorno inversión:} TIR 18.5\% | VAN \$1,245 MDP
\end{itemize}

\subsection{Ambiental}
\begin{itemize}
    \item \textbf{Captura CO\textsubscript{2}:} 765,000 ton certificables
    \item \textbf{Reducción GEI:} -45\% por unidad producto
    \item \textbf{Biodiversidad:} +400\% especies/ha en SSPi
    \item \textbf{Infiltración hídrica:} +333\% vs tradicional
\end{itemize}

\subsection{Social}
\begin{itemize}
    \item \textbf{Género:} 35\% mujeres (637 productoras)
    \item \textbf{Juventud:} 23\% jóvenes rurales (419 beneficiarios)
    \item \textbf{Cultural:} Revaloración prácticas mayas ancestrales
    \item \textbf{Seguridad alimentaria:} Autoabasto proteína animal
\end{itemize}

\section{Factores Críticos de Éxito}

\begin{enumerate}
    \item \textbf{Compromiso tripartito:} Convenios PEC Federal-Estatal-Productores
    \item \textbf{Focalización Pareto:} 80\% recursos en 11 municipios (80.3\% concentración)
    \item \textbf{Equipo especializado:} 5 profesionales OREF + asistencia continua
    \item \textbf{Integración sistémica:} 6 componentes como ecosistema único
    \item \textbf{Monitoreo robusto:} Indicadores SMART + evaluaciones externas
\end{enumerate}

\section{Cronograma Ejecutivo}

\textbf{Fase 1 (2026):} Centro Genético + Plataforma Digital + SSPi piloto\\
\textbf{Fase 2 (2027-2028):} Expansión SSPi + Repoblamiento + Lechería\\
\textbf{Fase 3 (2029-2030):} Consolidación + Meliponicultura + Evaluación

\section{Próximos Pasos}

\begin{enumerate}
    \item \textbf{Inmediato:} Presentación SADER Central para recursos PEC 2026
    \item \textbf{Q1 2026:} Firma convenio estatal + constitución equipo OREF
    \item \textbf{Q2 2026:} Inicio operaciones componentes prioritarios
    \item \textbf{2026-2030:} Implementación escalonada + monitoreo continuo
\end{enumerate}

\vspace{1cm}

\begin{center}
\textbf{\textcolor{sadergreen}{\Large LA OPORTUNIDAD ES AHORA}}\\[0.5cm]

\textbf{Convergencia única:} Políticas + Presupuesto + Mercados + Tecnología\\[0.3cm]

\textbf{\textcolor{sadergreen}{Yucatán líder en ganadería sustentable de México}}\\
\textbf{\textcolor{sadergreen}{Renacimiento Ganadero Maya 2026-2030}}
\end{center}

\end{document}
