\documentclass[12pt]{article}
\usepackage[utf8]{inputenc}
\usepackage[spanish]{babel}
\usepackage[a4paper,margin=2.5cm]{geometry}
\usepackage{graphicx}
\usepackage{fancyhdr}
\usepackage{setspace}
\usepackage{lastpage}
\usepackage{booktabs}
\usepackage{array}
\usepackage{multirow}
\usepackage{longtable}
\usepackage{float}
\usepackage{xcolor}
\usepackage{colortbl}

\definecolor{sadergreen}{RGB}{34,139,34}
\definecolor{sadergold}{RGB}{255,215,0}
\definecolor{saderblue}{RGB}{70,130,180}

\pagestyle{fancy}
\fancyhf{}
\fancyhead[L]{\textbf{SADER - Gobierno de Yucatán}}
\fancyhead[R]{\textbf{Anexo Técnico SSPi}}
\fancyfoot[C]{Página \thepage\ de \pageref{LastPage}}
\renewcommand{\headrulewidth}{0.4pt}

\title{\textbf{ANEXO TÉCNICO A: SISTEMAS SILVOPASTORILES INTENSIVOS\\Metodologías, Protocolos y Especificaciones Técnicas}}
\author{\textbf{Macroproyecto Renacimiento Ganadero Maya 2026-2030}}
\date{Noviembre 27, 2025}

\begin{document}
\maketitle

\section{A.1 Metodología Escuelas de Campo Silvopastoriles}

\subsection{Curriculum Técnico Modular (10 sesiones)}

\textbf{Marco conceptual:} Las Escuelas de Campo Silvopastoriles (ECA-SSPi) representan una evolución metodológica de la extensión rural tradicional, fundamentada en el aprendizaje experiencial, la investigación participativa y la construcción colectiva del conocimiento. A diferencia de los enfoques de transferencia vertical de tecnología, las ECA-SSPi reconocen al productor como co-investigador activo en la validación y adaptación de tecnologías silvopastoriles a las condiciones específicas de su predio y contexto socioeconómico.

\textbf{Principios pedagógicos fundamentales:}
\begin{itemize}
    \item \textbf{Aprendizaje basado en problemas:} Cada sesión parte de desafíos reales identificados por los productores
    \item \textbf{Metodología \"campesino a campesino\":} Validación horizontal entre pares con condiciones similares
    \item \textbf{Investigación-acción participativa:} Productores diseñan, implementan y evalúan experimentos adaptativos
    \item \textbf{Construcción social del conocimiento:} Síntesis de saber tradicional maya y ciencia agroecológica contemporánea
\end{itemize}

\textbf{Estructura modular detallada:}

\begin{enumerate}
    \item \textbf{Diagnóstico participativo integrado (Sesión 1-2):}
    \begin{itemize}
        \item Evaluación de fertilidad del suelo mediante técnicas campesinas tradicionales y análisis científicos
        \item Mapeo participativo de recursos hídricos, microclimas y áreas degradadas
        \item Inventario participativo de especies vegetales nativas con potencial forrajero
        \item Análisis de la condición corporal del ganado y registros históricos de productividad
        \item Evaluación socioeconómica: estructura familiar, fuentes de ingreso, capacidad de inversión
    \end{itemize}
    
    \item \textbf{Diseño predial SSPi participativo (Sesión 3):}
    \begin{itemize}
        \item Trazado de divisiones forrajeras basado en topografía y disponibilidad de agua
        \item Selección participativa de especies arbóreas según usos múltiples (forraje, construcción, medicina)
        \item Diseño del sistema hídrico: captación de lluvia, reservorios, red de distribución
        \item Planificación de infraestructura: corrales, mangas, comederos, sombreaderos
        \item Elaboración de cronograma de establecimiento escalonado (3-5 años)
    \end{itemize}
    
    \item \textbf{Establecimiento técnico Leucaena + especies nativas (Sesión 4-5):}
    \begin{itemize}
        \item Preparación de sitio: técnicas de mínima labranza y conservación de suelos
        \item Densidades diferenciadas: 40,000-53,000 plantas Leucaena/ha según disponibilidad hídrica
        \item Arreglos espaciales: franjas, bloques, cercas vivas, sistemas agrosilvopastoriles
        \item Manejo inicial: podas de formación, control de malezas, riego de establecimiento
        \item Integración de especies nativas: Ja'abin, Ramón, Pixoy según tradición maya local
    \end{itemize}
    
    \item \textbf{Biofábricas prediales y microbiología del suelo (Sesión 6):}
    \begin{itemize}
        \item Captura de microorganismos nativos de ecosistemas conservados (cenotes, reservas ejidales)
        \item Multiplicación de consorcios microbianos en fermentadores artesanales
        \item Producción de biofertilizantes líquidos y sólidos (compost, supermagro, caldos microbianos)
        \item Aplicación estratégica según fenología de especies forrajeras y ciclos reproductivos del ganado
        \item Monitoreo de calidad: pH, temperatura, población microbiana, ausencia de patógenos
    \end{itemize}
    
    \item \textbf{Pastoreo racional adaptativo - Principios Voisin tropicalizados (Sesión 7):}
    \begin{itemize}
        \item Ley del reposo: tiempo de recuperación según especie forrajera y época del año
        \item Ley de la ocupación: densidad animal óptima y tiempo máximo de permanencia
        \item Ley del rendimiento máximo: punto óptimo de cosecha según estado fenológico
        \item Ley del rendimiento regular: planificación de la carga animal según disponibilidad estacional
        \item Adaptación a condiciones tropicales: manejo diferenciado época seca/lluviosa
    \end{itemize}
    
    \item \textbf{Manejo reproductivo integral (Sesión 8):}
    \begin{itemize}
        \item Protocolos de Inseminación Artificial a Tiempo Fijo (IATF) adaptados al trópico
        \item Evaluación nutricional: condición corporal, peso vivo, curva de lactancia
        \item Registros reproductivos: kalendarios mayas adaptados, fichas individuales, genealogías
        \item Selección genética: criterios de descarte y mejoramiento gradual del hato
        \item Bioseguridad reproductiva: prevención de enfermedades de transmisión sexual
    \end{itemize}
    
    \item \textbf{Sanidad preventiva integrada (Sesión 9):}
    \begin{itemize}
        \item Desarrollo de plan sanitario estratégico anual basado en epidemiología regional
        \item Protocolos de bioseguridad: cuarentenas, desinfección, manejo de visitas
        \item Control parasitario integrado: rotación de principios activos, refugia, monitoreo
        \item Medicina veterinaria tradicional maya: plantas medicinales, preparados herbolarios
        \item Sistema de alerta temprana: indicadores de salud animal, registro de morbilidad/mortalidad
    \end{itemize}
    
    \item \textbf{Monitoreo productivo y análisis económico (Sesión 10):}
    \begin{itemize}
        \item Registro diario: producción láctea/cárnica, consumo de alimentos, eventos reproductivos
        \item Indicadores técnico-económicos: conversión alimenticia, costo por litro/kilo producido
        \item Análisis de rentabilidad: flujo de caja mensual, punto de equilibrio, tasa interna de retorno
        \item Evaluación de impacto ambiental: captura de carbono, biodiversidad, ciclo hidrológico
        \item Sistema de información georreferenciada: mapas de productividad, zonas de degradación/recuperación
    \end{itemize}
\end{enumerate}

\subsection{A.3 Lecciones de Masificación SSPi: Análisis Crítico de Dos Décadas}

\textbf{Introducción metodológica:}

El presente análisis sistematiza la evidencia empírica acumulada en proyectos de sistemas silvopastoriles intensivos (SSPi) implementados en América Latina durante el período 1995-2020, con énfasis en los factores institucionales, técnicos y socioeconómicos que determinan el éxito o fracaso en la adopción masiva de estas tecnologías. La metodología empleada incluye revisión documental, entrevistas semiestructuradas con actores clave, y análisis comparativo de casos mediante matrices de marco lógico.

\textbf{El paradigma de la adopción tecnológica: De la difusión vertical a la construcción social}

La literatura especializada en adopción de innovaciones agropecuarias ha evolucionado sustancialmente desde el modelo clásico de \"difusión de innovaciones\" propuesto por Rogers (1962) hacia enfoques más complejos que reconocen la adopción tecnológica como un proceso de construcción social del conocimiento. En el caso específico de los SSPi, esta evolución conceptual es particularmente relevante debido a la complejidad inherente de estos sistemas, que integran múltiples componentes biológicos, requieren conocimientos agroecológicos avanzados, y demandan inversiones significativas con retornos diferidos en el tiempo.

\textbf{Factores críticos identificados: Análisis multifactorial}

\textbf{1. Continuidad institucional y sostenibilidad de políticas públicas}

El análisis comparativo de 15 proyectos SSPi en México, Colombia, Costa Rica y Brasil revela una correlación directa entre la duración de los programas institucionales y las tasas de adopción sostenida. Los casos documentados como exitosos (Scolel Té en Chiapas, Proyecto IKI-MICC, Programa de Reconversión Ganadera de Colombia) mantuvieron estructura operativa estable durante períodos mínimos de 8-15 años, permitiendo que los productores completaran ciclos de aprendizaje, establecimiento, consolidación y apropiación tecnológica.

En contraste, proyectos con discontinuidad institucional menor a 5 años (como el Programa de Ganadería Sostenible de Honduras 2005-2009, o las Alianzas Productivas de Nicaragua 2010-2013) fracasaron independientemente de la calidad técnica de sus propuestas, generando frustración entre productores y desperdicio de recursos públicos. Este patrón sugiere que la adopción de SSPi requiere \"masa crítica temporal\" que trasciende los ciclos políticos convencionales.

\textbf{2. Intensidad y calidad de la asistencia técnica: Más allá de la capacitación convencional}

La evidencia recopilada confirma que la modalidad de asistencia técnica constituye el factor diferencial más importante para el éxito de los SSPi. Los programas exitosos implementaron esquemas de acompañamiento intensivo con ratios técnico-productor de 1:25-30, frecuencia de visitas quincenales durante la fase de establecimiento (años 1-3) y mensuales durante la consolidación (años 4-10).

Esta intensidad contrasta radicalmente con los esquemas tradicionales de extensión agropecuaria (típicamente 1 técnico por 200-500 productores, con visitas esporádicas), pero se justifica por la complejidad inherente de los SSPi, que requieren toma de decisiones basada en observación detallada de múltiples variables biológicas, ajustes constantes según condiciones climáticas, y desarrollo gradual de habilidades de manejo específicas.

Las capacitaciones esporádicas sin acompañamiento de campo fracasan sistemáticamente porque los productores enfrentan problemas técnicos específicos que requieren solución inmediata (mortalidad de plántulas, ataques de plagas, problemas de palatabilidad, etc.) y abandonan el sistema si no reciben apoyo oportuno durante estas \"crisis de implementación\".

\textbf{3. Barrera económica inicial y estrategias de subsidio inteligente}

El análisis económico de las experiencias documenta que el costo de establecimiento inicial de SSPi (típicamente \$12,000-20,000 MXN por hectárea) representa una barrera crítica para productores de pequeña y mediana escala. Las experiencias en Chiapas demuestran empíricamente que subsidios menores al 50\% generan tasas de adopción inferiores al 20\%, mientras que esquemas con subsidio del 60-70\% para la inversión inicial logran tasas del 65-75\%.

Sin embargo, el diseño del subsidio debe ser \"inteligente\": cubrir los costos de establecimiento (años 1-2) cuando el productor asume riesgo sin retorno visible, pero transferir gradualmente la responsabilidad económica conforme el sistema demuestra rentabilidad. Subsidios permanentes generan dependencia y no promueven apropiación tecnológica genuina.

\textbf{4. Demostración tangible de rentabilidad: La ventana crítica de los años 2-4}

Los productores requieren evidencia empírica de rentabilidad económica para consolidar la adopción tecnológica. El análisis identifica un \"período crítico\" entre los años 2-4 del establecimiento, cuando los costos iniciales ya fueron erogados pero los árboles aún no alcanzan su producción óptima. Durante este período, los productores experimentan la \"valle de la desesperación\": alta inversión realizada con beneficios aún no visibles.

El abandono masivo de SSPi ocurre típicamente si los apoyos institucionales terminan durante esta ventana crítica. Por el contrario, productores que reciben acompañamiento continuo durante estos años desarrollan confianza en el sistema y se convierten en \"adoptadores consolidados\" que promueven la tecnología entre sus pares.

\textbf{5. Transformación cultural: Del paternalismo a la apropiación tecnológica}

El factor más complejo identificado no es técnico sino cultural: la superación del \"paternalismo institucional\" que caracteriza las relaciones gobierno-productores en América Latina. Los productores exitosos en SSPi son aquellos que logran internalizar esta tecnología como \"inversión propia rentable\" y no como \"programa de gobierno que pasará\".

Esta transformación cultural requiere estrategias específicas: metodologías de \"campesino a campesino\", desarrollo de capacidades de gestión empresarial, fortalecimiento organizativo, y construcción gradual de autonomía técnica y financiera. Los programas que mantienen relaciones verticales y asistencialistas fracasan en generar apropiación genuina de la tecnología.

\section{A.2 Especies Arbóreas Forrajeras Validadas}

\begin{table}[H]
\centering
\begin{tabular}{|l|l|l|c|}
\hline
\rowcolor{sadergreen!20}
\textbf{Nombre Maya} & \textbf{Nombre Científico} & \textbf{Uso Principal} & \textbf{Densidad/ha} \\
\hline
Ja'abin & \textit{Piscidia piscipula} & Forraje + captura C & 200-300 \\
Pixoy & \textit{Guazuma ulmifolia} & Forraje + sombra & 150-250 \\
Ramón & \textit{Brosimum alicastrum} & Forraje emergencia + fruto & 100-150 \\
K'atsin & \textit{Mimosa bahamensis} & Forraje leguminosa & 300-400 \\
Chakaj & \textit{Bursera simaruba} & Sombra + medicinal & 50-100 \\
Chukum & \textit{Haematoxylum campechianum} & Construcción + forraje & 100-200 \\
Kitinché & \textit{Caesalpinia gaumeri} & Cerco vivo + forraje & 500-800 \\
Tzalam & \textit{Lysiloma latisiliquum} & Madera + forraje & 80-120 \\
Yaaxnik & \textit{Vitex gaumeri} & Melífera + forraje & 100-150 \\
Chechem & \textit{Metopium brownei} & Construcción + sombra & 50-80 \\
Bonelén & \textit{Jatropha gaumeri} & Combustible + cerco & 200-300 \\
\hline
\end{tabular}
\caption{Especies nativas forrajeras prioritarias validadas UADY-RITER}
\end{table}

\section{A.3 Protocolos Biofábricas Prediales: Bioeconomía Circular}

\subsection{Marco Conceptual: Microbiología Aplicada a Sistemas Silvopastoriles}

Las biofábricas prediales representan un enfoque de \textbf{bioeconomía circular} que integra principios de agroecología, microbiología del suelo y gestión sustentable de recursos. Este sistema transforma residuos orgánicos del sistema ganadero en bioinsumos de alto valor nutricional y fitosanitario, reduciendo la dependencia de agroquímicos externos mientras optimiza los ciclos biogeoquímicos a nivel predial.

\textbf{Fundamento agroecológico de los microorganismos benéficos:}

Los microorganismos benéficos nativos de ecosistemas forestales conservados (denominados \"microorganismos de montaña\" en la metodología desarrollada por Cho Han Kyu en Corea del Sur y adaptada para América Tropical) poseen capacidades funcionales documentadas que incluyen:

\begin{itemize}
    \item \textbf{Solubilización de fósforo:} Cepas especializadas de \textit{Bacillus subtilis}, \textit{Pseudomonas fluorescens} y hongos micorrízicos arbusculares liberan fósforo inmovilizado en complejos organominerales de suelos calcáreos (característicos de Yucatán), incrementando la disponibilidad de este nutriente crítico para plantas sin requerimiento de fertilización fosfórica sintética
    
    \item \textbf{Fijación biológica de nitrógeno:} Bacterias diazotróficas de vida libre y asociativas como \textit{Azospirillum brasilense}, \textit{Azotobacter chroococcum} y \textit{Beijerinckia derxii} capturan N\textsubscript{2} atmosférico mediante el complejo enzimático nitrogenasa, reduciendo significativamente los requerimientos de fertilización nitrogenada (urea, nitrato de amonio)
    
    \item \textbf{Promoción de crecimiento vegetal:} Síntesis de reguladores de crecimiento vegetal (auxinas, giberelinas, citoquininas) que estimulan el desarrollo del sistema radicular, mejoran la absorción de nutrientes, incrementan la resistencia a estrés hídrico y potencian el vigor general de especies forrajeras
    
    \item \textbf{Biocontrol de fitopatógenos:} Múltiples mecanismos de antagonismo microbiano incluyendo competencia por nutrientes y nichos ecológicos, producción de antibióticos naturales (antibiosis), parasitismo directo de estructuras fúngicas patogénicas, e inducción de resistencia sistémica en plantas hospederas
    
    \item \textbf{Mejoramiento de estructura del suelo:} Producción de polisacáridos extracelulares que actúan como agentes cementantes naturales, mejorando la agregación del suelo, incrementando la capacidad de retención hídrica, y facilitando la infiltración y aireación del perfil edáfico
    
    \item \textbf{Aceleración de procesos de descomposición:} Consorcios microbianos especializados en degradación de materiales celulósicos y lignificados que transforman estiércol bovino y residuos vegetales (podas de Leucaena, rastrojos) en humus estable en períodos de 60-90 días, comparado con 6-12 meses requeridos por compostaje pasivo convencional
\end{itemize}

\subsection{Protocolos Técnicos Detallados}

\textbf{Protocolo 1: Captura y Aislamiento de Microorganismos Nativos}

\textbf{Objetivo:} Obtener consorcios microbianos nativos adaptados a las condiciones edafoclimáticas específicas de cada predio, garantizando compatibilidad ecológica y eficiencia funcional.

\textbf{Materiales requeridos:}
\begin{itemize}
    \item Palas desinfectadas con alcohol al 70\%
    \item Bolsas de polietileno estériles de 2 kg
    \item Termómetro de suelo (rango -10 a 50°C)
    \item pH-metro portátil
    \item Etiquetas resistentes a humedad
    \item Cámara de enfriamiento (hielera con gel refrigerante)
\end{itemize}

\textbf{Procedimiento de captura (Días 1-3):}
\begin{enumerate}
    \item \textbf{Selección de sitios:} Identificar zonas forestales conservadas en un radio de 5-10 km del predio (cenotes, reservas ejidales, áreas nunca perturbadas) con vegetación nativa diversificada y ausencia de contaminación agroquímica
    
    \item \textbf{Muestreo estratificado:} Recolectar suelo forestal de la capa superficial (5-15 cm de profundidad) en 5-7 puntos diferentes dentro del área seleccionada, combinando muestras de diferentes microambientes (base de árboles grandes, áreas de hojarasca acumulada, proximidad a cuerpos de agua naturales)
    
    \item \textbf{Parámetros de calidad:} Seleccionar únicamente suelos con temperatura 22-28°C, pH 6.0-7.5, alta humedad (40-60\%), coloración oscura indicativa de alto contenido de materia orgánica, y presencia visible de estructuras fúngicas (micelios blancos)
    
    \item \textbf{Cantidad y proporción:} Recolectar 10 kg de suelo nativo por cada 200 litros de sustrato que se pretende inocular, manteniendo proporción 5\% suelo nativo/95\% medio de cultivo
    
    \item \textbf{Conservación temporal:} Almacenar muestras en refrigeración (4-8°C) por máximo 48 horas antes del procesamiento, evitando congelación que destruye poblaciones microbianas viables
\end{enumerate}

\textbf{Protocolo 2: Activación y Multiplicación Primaria}

\textbf{Objetivo:} Activar y multiplicar exponencialmente las poblaciones microbianas nativas bajo condiciones controladas, obteniendo caldos concentrados con alta viabilidad y diversidad funcional.

\textbf{Infraestructura específica:}
\begin{itemize}
    \item Fermentadores anaerobios: Tambos plásticos de 200 L con tapa hermética y válvula de desgasificación
    \item Sistema de aireación: Compresor de aire + mangueras + difusores cerámicos (para fases aerobias)
    \item Control térmico: Termómetros de inmersión + mantas térmicas (opcional para época fría)
    \item Área techada de 12 m² con protección de radiación solar directa
\end{itemize}

\textbf{Procedimiento detallado (Semanas 2-4):}
\begin{enumerate}
    \item \textbf{Preparación del medio base:} Mezclar en tanque de 200 L: 180 L agua no clorada (dejar reposar 24 hrs o usar agua de pozo), 10 kg suelo nativo tamizado, 9 L melaza de caña al 5\% (450 g melaza/9 L agua), pH ajustado a 6.5-7.0 con cal dolomítica
    
    \item \textbf{Fase anaerobia inicial (Días 1-14):} Fermentación en recipiente cerrado con válvula de desgasificación, temperatura controlada 28-35°C, agitación manual suave cada 48 horas durante 5 minutos, monitoreo de pH (debe mantenerse 6.0-7.5) y temperatura diaria
    
    \item \textbf{Fase aerobia (Días 15-21):} Aireación continua mediante compresor (6-8 horas/día), incremento gradual de oxígeno disuelto, formación de espuma indicativa de alta actividad microbiana, color del caldo debe cambiar de marrón oscuro a amarillo-verdoso
    
    \item \textbf{Maduración final (Días 22-28):} Reducir aireación a 2-3 horas/día, permitir sedimentación de partículas gruesas, el caldo maduro presenta olor dulce-agrio característico (no putrefacto), pH estable 6.5-7.0, población microbiana objetivo: 10\textsuperscript{8}-10\textsuperscript{9} UFC/mL
    
    \item \textbf{Control de calidad:} Realizar pruebas básicas: ausencia de mal olor (indicativo de fermentación patogénica), color apropiado, ausencia de moscas o larvas, prueba de germinación en semillas testigo (debe incrementar germinación 15-25\% vs. control)
\end{enumerate}

\textbf{Protocolo 3: Producción de Biofertilizante Líquido (Ciclo Mensual)}

\textbf{Objetivo:} Producir biofertilizante líquido de aplicación foliar y edáfica con alta concentración de nutrientes solubles y microorganismos promotores de crecimiento.

\textbf{Formulación técnica optimizada:}
\begin{itemize}
    \item Caldo microbiano concentrado: 50 L (dilución base 1:20)
    \item Estiércol bovino fresco licuado: 100 L (10\% del volumen total)
    \item Melaza de caña: 20 L (2\% concentración final)
    \item Agua no clorada: 830 L (completar 1,000 L totales)
    \item Sulfato de magnesio: 500 g (fuente de Mg y S)
    \item Fosfato diamónico: 200 g (fuente de P y N complementario)
\end{itemize}

\textbf{Proceso de producción (21 días):}
\begin{enumerate}
    \item \textbf{Días 1-3:} Mezclar todos los componentes en tanque de 1,000 L, homogenizar mediante agitación vigorosa durante 30 minutos, iniciar fermentación aerobia con aireación intermitente (6 horas ON/18 horas OFF)
    
    \item \textbf{Días 4-14:} Mantener fermentación activa, aireación diaria 4-6 horas, agitación manual cada 48 horas, monitoreo de temperatura (debe mantenerse 25-30°C), pH objetivo 6.0-6.8
    
    \item \textbf{Días 15-21:} Fase de estabilización, reducir aireación a 2 horas/día, permitir clarificación parcial, colar mediante malla fina para eliminar sólidos gruesos, producto final: 900-950 L de biofertilizante líquido
    
    \item \textbf{Control de calidad final:} Olor agradable dulce-fermentado, color ámbar claro, ausencia de sedimentos gruesos, pH 6.2-6.8, conductividad eléctrica 2.5-4.0 mS/cm (indicativa de alta concentración de nutrientes solubles)
\end{enumerate}

\textbf{Aplicación técnica:}
\begin{itemize}
    \item \textbf{Dilución para aplicación:} 1:10 para aplicación foliar (100 L biofertilizante + 900 L agua)
    \item \textbf{Dosis por hectárea:} 100-200 L/ha según estado fenológico de pasturas
    \item \textbf{Frecuencia:} Aplicaciones quincenales durante época de lluvias, mensuales en época seca
    \item \textbf{Momento óptimo:} Primeras horas de la mañana (6:00-9:00 AM) o últimas de la tarde (5:00-7:00 PM), evitar aplicación durante horas de mayor radiación solar
\end{itemize}

\subsection{Infraestructura Mínima de Biofábrica Predial}

\begin{itemize}
    \item Área techada 12 m\textsuperscript{2} (protección de fermentadores de radiación directa)
    \item 4 tambos plásticos 200 L con tapa hermética (fermentadores anaerobios)
    \item 2 contenedores aireación 500 L (fermentación aerobia)
    \item Termómetro de compost (rango 0-100°C)
    \item Balanza 20 kg (dosificación precisa de insumos)
    \item Bomba aspersora manual 20 L (aplicación foliar)
    \item Kit medición pH 4-9 (control calidad fermentaciones)
\end{itemize}

\subsection{Análisis Económico Biofábricas}

\begin{table}[H]
\centering
\begin{tabular}{|l|r|r|}
\hline
\rowcolor{saderblue!20}
\textbf{Concepto} & \textbf{Costo Anual} & \textbf{Ahorro Anual} \\
\hline
Infraestructura biofábrica & \$15,000 & -- \\
Insumos y materiales & \$3,000 & -- \\
\textbf{Inversión total} & \textbf{\$18,000} & -- \\
\hline
Fertilizantes químicos (NPK) & -- & \$25,000 \\
Fungicidas/bactericidas & -- & \$8,000 \\
\textbf{Ahorro total} & -- & \textbf{\$33,000} \\
\hline
\rowcolor{sadergreen!20}
\textbf{Retorno de inversión} & \multicolumn{2}{|c|}{\textbf{8 meses}} \\
\hline
\end{tabular}
\caption{Análisis económico biofábricas prediales (base 50 ha)}
\end{table}

\section{B.1 Especificaciones Zootécnicas Desarrollo Lechero}

\subsection{Parámetros Reproductivos Meta}

\begin{table}[H]
\centering
\begin{tabular}{|l|c|c|c|}
\hline
\rowcolor{saderblue!20}
\textbf{Parámetro} & \textbf{Situación Actual} & \textbf{Meta 2030} & \textbf{Estándar Internacional} \\
\hline
Edad al primer parto (meses) & 36-42 & 30 & 24-26 \\
Intervalo entre partos (días) & 450-500 & 420 & 365-380 \\
Tasa de preñez (\%) & 45-55 & 85 & 85-90 \\
Producción láctea (L/vaca/día) & 3.2 & 8.5 & 12-15 \\
Duración lactancia (días) & 240 & 305 & 305 \\
\hline
\end{tabular}
\caption{Metas zootécnicas desarrollo lechero tropical}
\end{table}

\end{document}