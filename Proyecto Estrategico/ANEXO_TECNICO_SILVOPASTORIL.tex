\documentclass[12pt]{article}
\usepackage[utf8]{inputenc}
\usepackage[spanish]{babel}
\usepackage[a4paper,margin=2.5cm]{geometry}
\usepackage{graphicx}
\usepackage{fancyhdr}
\usepackage{setspace}
\usepackage{lastpage}
\usepackage{booktabs}
\usepackage{array}
\usepackage{multirow}
\usepackage{longtable}
\usepackage{float}
\usepackage{xcolor}
\usepackage{colortbl}

\definecolor{sadergreen}{RGB}{34,139,34}
\definecolor{sadergold}{RGB}{255,215,0}
\definecolor{saderblue}{RGB}{70,130,180}

\pagestyle{fancy}
\fancyhf{}
\fancyhead[L]{\textbf{SADER - Gobierno de Yucatán}}
\fancyhead[R]{\textbf{Anexo Técnico SSPi}}
\fancyfoot[C]{Página \thepage\ de \pageref{LastPage}}
\renewcommand{\headrulewidth}{0.4pt}

\title{\textbf{ANEXO TÉCNICO A: SISTEMAS SILVOPASTORILES INTENSIVOS\\Metodologías, Protocolos y Especificaciones Técnicas}}
\author{\textbf{Macroproyecto Renacimiento Ganadero Maya 2026-2030}}
\date{Noviembre 27, 2025}

\begin{document}
\maketitle

\section{A.1 Metodología Escuelas de Campo Silvopastoriles}

\subsection{Curriculum Técnico Modular (10 sesiones)}

\begin{enumerate}
    \item \textbf{Diagnóstico participativo:} evaluación fertilidad suelo, condición praderas, carga animal actual
    \item \textbf{Diseño predial SSPi:} trazado de divisiones, especies arbóreas, sistema hídrico, infraestructura
    \item \textbf{Establecimiento Leucaena + especies nativas:} densidades, arreglos espaciales, manejo inicial
    \item \textbf{Biofábricas prediales:} captura y multiplicación de microorganismos benéficos de montaña, producción de bioinsumos (reducción 75-90\% costo vs. agroquímicos)
    \item \textbf{Pastoreo racional adaptativo:} principios Voisin, toma de decisiones por observación
    \item \textbf{Manejo reproductivo:} protocolos IATF, evaluación condición corporal, registros
    \item \textbf{Sanidad preventiva:} plan sanitario estratégico, bioseguridad, control parasitario integrado
    \item \textbf{Monitoreo productivo:} registro diario, indicadores técnico-económicos, análisis de rentabilidad
    \item \textbf{Comercialización asociativa:} esquemas de valor agregado, certificaciones, mercados diferenciados
    \item \textbf{Gestión empresarial:} costos de producción, punto de equilibrio, flujo de caja
\end{enumerate}

\section{A.2 Especies Arbóreas Forrajeras Validadas}

\begin{table}[H]
\centering
\begin{tabular}{|l|l|l|c|}
\hline
\rowcolor{sadergreen!20}
\textbf{Nombre Maya} & \textbf{Nombre Científico} & \textbf{Uso Principal} & \textbf{Densidad/ha} \\
\hline
Ja'abin & \textit{Piscidia piscipula} & Forraje + captura C & 200-300 \\
Pixoy & \textit{Guazuma ulmifolia} & Forraje + sombra & 150-250 \\
Ramón & \textit{Brosimum alicastrum} & Forraje emergencia + fruto & 100-150 \\
K'atsin & \textit{Mimosa bahamensis} & Forraje leguminosa & 300-400 \\
Chakaj & \textit{Bursera simaruba} & Sombra + medicinal & 50-100 \\
Chukum & \textit{Haematoxylum campechianum} & Construcción + forraje & 100-200 \\
Kitinché & \textit{Caesalpinia gaumeri} & Cerco vivo + forraje & 500-800 \\
Tzalam & \textit{Lysiloma latisiliquum} & Madera + forraje & 80-120 \\
Yaaxnik & \textit{Vitex gaumeri} & Melífera + forraje & 100-150 \\
Chechem & \textit{Metopium brownei} & Construcción + sombra & 50-80 \\
Bonelén & \textit{Jatropha gaumeri} & Combustible + cerco & 200-300 \\
\hline
\end{tabular}
\caption{Especies nativas forrajeras prioritarias validadas UADY-RITER}
\end{table}

\section{A.3 Protocolos Biofábricas Prediales}

\subsection{Protocolo de Captura y Multiplicación}

\begin{enumerate}
    \item \textbf{Captura de microorganismos nativos (Mes 1):} Recolección de suelo forestal de zonas conservadas (cenotes, reservas ejidales) con alta carga microbiana. Proporción: 10 kg suelo/200 litros substrato inicial
    
    \item \textbf{Activación y multiplicación primaria (Semana 2-4):} Mezcla de suelo nativo + melaza 5\% + agua no clorada en fermentadores anaerobios (tambos 200 L). Temperatura controlada 28-35°C. Aireación cada 48 hrs. Producto: caldo microbiano concentrado (10\textsuperscript{8}-10\textsuperscript{9} UFC/mL)
    
    \item \textbf{Producción de biofertilizante líquido (mensual):} Dilución caldo concentrado 1:20 + estiércol fresco licuado 10\% + melaza 2\%. Fermentación aerobia 21 días. Rendimiento: 1,000 L/lote. Aplicación foliar/suelo: 100-200 L/ha diluido 1:10
    
    \item \textbf{Producción de supermagro enriquecido (bimestral):} Caldo microbiano + estiércol fresco + ceniza + melaza + sales minerales (sulfatos Zn, Cu, B, Mo). Fermentación 45 días. Aplicación: 50 L/ha en etapas críticas
    
    \item \textbf{Compostaje acelerado (continuo):} Pilas estiércol + residuos de poda SSPi + caldo microbiano como acelerador. Volteos semanales. Temperatura 55-65°C. Maduración 60-75 días. Aplicación: 5-10 ton/ha cada 2 años
\end{enumerate}

\subsection{Infraestructura Mínima de Biofábrica Predial}

\begin{itemize}
    \item Área techada 12 m\textsuperscript{2} (protección de fermentadores de radiación directa)
    \item 4 tambos plásticos 200 L con tapa hermética (fermentadores anaerobios)
    \item 2 contenedores aireación 500 L (fermentación aerobia)
    \item Termómetro de compost (rango 0-100°C)
    \item Balanza 20 kg (dosificación precisa de insumos)
    \item Bomba aspersora manual 20 L (aplicación foliar)
    \item Kit medición pH 4-9 (control calidad fermentaciones)
\end{itemize}

\subsection{Análisis Económico Biofábricas}

\begin{table}[H]
\centering
\begin{tabular}{|l|r|r|}
\hline
\rowcolor{saderblue!20}
\textbf{Concepto} & \textbf{Costo Anual} & \textbf{Ahorro Anual} \\
\hline
Infraestructura biofábrica & \$15,000 & -- \\
Insumos y materiales & \$3,000 & -- \\
\textbf{Inversión total} & \textbf{\$18,000} & -- \\
\hline
Fertilizantes químicos (NPK) & -- & \$25,000 \\
Fungicidas/bactericidas & -- & \$8,000 \\
\textbf{Ahorro total} & -- & \textbf{\$33,000} \\
\hline
\rowcolor{sadergreen!20}
\textbf{Retorno de inversión} & \multicolumn{2}{|c|}{\textbf{8 meses}} \\
\hline
\end{tabular}
\caption{Análisis económico biofábricas prediales (base 50 ha)}
\end{table}

\section{B.1 Especificaciones Zootécnicas Desarrollo Lechero}

\subsection{Parámetros Reproductivos Meta}

\begin{table}[H]
\centering
\begin{tabular}{|l|c|c|c|}
\hline
\rowcolor{saderblue!20}
\textbf{Parámetro} & \textbf{Situación Actual} & \textbf{Meta 2030} & \textbf{Estándar Internacional} \\
\hline
Edad al primer parto (meses) & 36-42 & 30 & 24-26 \\
Intervalo entre partos (días) & 450-500 & 420 & 365-380 \\
Tasa de preñez (\%) & 45-55 & 85 & 85-90 \\
Producción láctea (L/vaca/día) & 3.2 & 8.5 & 12-15 \\
Duración lactancia (días) & 240 & 305 & 305 \\
\hline
\end{tabular}
\caption{Metas zootécnicas desarrollo lechero tropical}
\end{table}

\end{document}