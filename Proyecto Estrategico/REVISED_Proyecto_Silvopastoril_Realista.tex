\documentclass[12pt,letterpaper]{article}
\usepackage[utf8]{inputenc}
\usepackage[spanish]{babel}
\usepackage{geometry}
\usepackage{graphicx}
\usepackage{fancyhdr}
\usepackage{setspace}
\usepackage{lastpage}
\usepackage{parskip}
\usepackage{booktabs}
\usepackage{array}
\usepackage{multirow}
\usepackage{longtable}
\usepackage{float}
\usepackage{xcolor}
\usepackage{colortbl}

% Configuración avanzada de tipografía
\usepackage[T1]{fontenc}  % Codificación de fuente mejorada
\usepackage{lmodern}      % Fuente Latin Modern (mejorada)
\usepackage{microtype}    % Mejoras micro-tipográficas
\usepackage{csquotes}     % Manejo inteligente de comillas

% Configuración de espaciado mejorado
\linespread{1.1}         % Interlineado ligeramente aumentado para mejor legibilidad

% Configuración de página y márgenes exactos SADER
\geometry{top=2.5cm,bottom=2.5cm,left=3cm,right=3cm,headheight=20pt}
\pagestyle{fancy}
\fancyhf{}
\rfoot{\thepage\ de \pageref{LastPage}}
\renewcommand{\headrulewidth}{0pt}
\renewcommand{\footrulewidth}{0pt}

% Logo oficial en todas las páginas
\fancyhead[L]{\includegraphics[width=2.8cm]{logo_sader.png}}

% Definición de colores SADER
\definecolor{sadergreen}{RGB}{34,139,34}
\definecolor{sadergold}{RGB}{255,215,0}

\begin{document}

% ========================================
% PORTADA OFICIAL
% ========================================
\begin{titlepage}
    \centering
    \vspace*{1cm}
    \includegraphics[width=0.28\textwidth]{logo_sader.png}\\[2.5cm]
    
    \vspace{1.5cm}
    {\large\bfseries Proyecto Estratégico:\par}
    \vspace{2.5cm}
    
    {\Huge\bfseries Repoblamiento Ganadero Bovino\par}
    {\Huge\bfseries con Sistemas Silvopastoriles\par}
    {\Huge\bfseries en Yucatán\par}
    \vspace{1.5cm}
    {\LARGE Noviembre 2025\par}
    {\LARGE Fundamentado en Estadísticas Oficiales SIAP\par}
    
    \vfill
    
    {\large Mérida, Yucatán, 21 de noviembre de 2025\par}
    \vspace{0.5cm}
    {\large MVZ SERGIO MUÑOZ DE ALBA MEDRANO\par}
    {\large Prestador de Servicios Independiente\par}
    {\large Oficina Estatal de Representación en la Entidad Federativa Yucatán (OREF Yucatán)\par}
    {\large Secretaría de Agricultura y Desarrollo Rural (SADER)\par}
\end{titlepage}

% ========================================
% PÁGINA DE CONTENIDO
% ========================================
\clearpage
\tableofcontents
\clearpage
\setcounter{page}{3}

% ========================================
% INTRODUCCIÓN
% ========================================
\section{Introducción}

El proyecto busca incrementar el hato ganadero bovino de manera sostenible mediante la implementación de sistemas silvopastoriles (SSP) en las regiones sur y oriente de Yucatán. Estos sistemas integran pastos mejorados (\textit{Cynodon nlemfuensis} y \textit{Brachiaria brizantha}), leguminosas (\textit{Leucaena leucocephala}), y árboles nativos (\textit{Inga edulis} y \textit{Brosimum alicastrum}), promoviendo sostenibilidad, resiliencia climática y reducción de emisiones de metano.

Se fundamenta en la colaboración técnica con la \textbf{Universidad Autónoma de Yucatán (UADY)} y \textbf{The Nature Conservancy (TNC)}, que aportan investigación aplicada sobre SSP y su impacto ambiental en condiciones tropicales.

% ========================================
% JUSTIFICACIÓN
% ========================================
\section{Justificación Basada en Datos Oficiales SIAP}

\subsection{Situación Actual del Hato Ganadero Yucateco (2023)}

Según el \textbf{Sistema de Información Agroalimentaria y Pesquera (SIAP)}, Yucatán cuenta con:

\begin{itemize}
    \item \textbf{605,536 cabezas bovinas totales} (2023)
    \begin{itemize}
        \item 602,180 cabezas para carne (99.4\%)
        \item 3,356 cabezas para leche (0.6\%)
    \end{itemize}
\end{itemize}

\subsection{Evolución del Inventario Ganadero (2014-2023)}

\begin{table}[H]
\centering
\begin{tabular}{|l|c|c|c|}
\hline
\rowcolor{sadergreen!20}
\textbf{Concepto} & \textbf{2014} & \textbf{2023} & \textbf{Variación} \\
\hline
Ganado de Carne & 553,509 & 602,180 & +8.8\% (+48,671) \\
\hline
Ganado de Leche & 5,220 & 3,356 & \textbf{-35.7\% (-1,864)} \\
\hline
\rowcolor{sadergold!30}
\textbf{Total} & \textbf{558,729} & \textbf{605,536} & \textbf{+8.4\% (+46,807)} \\
\hline
\end{tabular}
\caption{Evolución del Inventario Ganadero Yucateco según SIAP}
\end{table}

\textbf{Análisis}: El sector cárnico muestra crecimiento moderado pero constante (+0.9\% anual), mientras el lechero presenta una \textbf{preocupante reducción de 35.7\%} en 9 años, evidenciando la necesidad de programas de repoblamiento especializado.

\subsection{Problemática Identificada}

\begin{enumerate}
    \item \textbf{Pastizales degradados}: Limitada productividad por sobrepastoreo
    \item \textbf{Baja eficiencia reproductiva}: Tasas de preñez 65-75\% vs 85\%+ óptimas
    \item \textbf{Vulnerabilidad climática}: Sequías recurrentes afectan disponibilidad forrajera
    \item \textbf{Presión de cambio de uso de suelo}: Competencia con desarrollo urbano/turístico
\end{enumerate}

Los SSP, según estudios de TNC-UADY, pueden aumentar la captura de carbono (15-25 ton CO\textsubscript{2}eq/ha) y reducir emisiones de metano entérico (20-30\%), alineándose con la \textbf{Estrategia Nacional de Mitigación} y la Directriz 4.1.1 del \textit{Plan Estatal de Desarrollo Renacimiento Maya 2024-2030}.

% ========================================
% OBJETIVOS
% ========================================
\section{Objetivos}

\subsection{Objetivo General}

Incrementar la productividad ganadera mediante la reconversión a sistemas silvopastoriles y el repoblamiento estratégico del hato bovino, contribuyendo a la \textbf{seguridad alimentaria}, la \textbf{sostenibilidad ambiental} y el \textbf{mejoramiento de ingresos} de pequeños y medianos productores yucatecos.

\subsection{Objetivos Específicos}

\begin{enumerate}
    \item \textbf{Reconvertir 6,000 hectáreas} a sistemas silvopastoriles durante el período 2026-2030 (120 UPP × 50 ha)
    \item \textbf{Incrementar el hato ganadero en 12,000 cabezas} mediante repoblamiento con vaquillas F1 de calidad genética
    \item \textbf{Mejorar la eficiencia reproductiva} del hato existente (75\% → 85\% tasa de preñez)
    \item \textbf{Reducir emisiones de GEI} en 20-30\% por unidad de producto mediante SSPi
    \item \textbf{Capacitar 120 productores} (1 por UPP) en tecnologías silvopastoriles y manejo reproductivo
\end{enumerate}

% ========================================
% POBLACIÓN OBJETIVO
% ========================================
\section{Población Objetivo}

\textbf{Pequeños y medianos productores ganaderos} con las siguientes características:
\begin{itemize}
    \item Hato de \textbf{10-50 cabezas bovinas}
    \item Superficie ganadera de \textbf{5-15 hectáreas}
    \item Ubicados en \textbf{regiones sur y oriente} de Yucatán
    \item Compromiso con la \textbf{adopción de prácticas sostenibles}
    \item Participación en organizaciones productivas locales
\end{itemize}

% ========================================
% DIMENSIONES DEL PROYECTO
% ========================================
\section{Dimensiones del Proyecto}

\subsection{Meta de Reconversión de Tierras}

\textbf{6,000 hectáreas totales (120 UPP)} distribuidas como:
\begin{itemize}
    \item \textbf{Año 1 (2026)}: 1,200 ha infraestructura + establecimiento Leucaena (30 UPP)
    \item \textbf{Año 2 (2027)}: 1,200 ha (30 UPP) - Maduración Leucaena cohorte 2026
    \item \textbf{Año 3 (2028)}: 1,200 ha (30 UPP)
    \item \textbf{Año 4 (2029)}: 1,200 ha (30 UPP)
    \item \textbf{Año 5 (2030)}: 1,200 ha (30 UPP)
\end{itemize}

\textbf{Nota crítica:} La Leucaena leucocephala requiere 6-9 meses de maduración antes de introducir ganado.

\subsection{Modelo de Repoblamiento con Proyección Realista}

\textbf{PARÁMETROS ZOOTÉCNICOS APLICADOS:}
\begin{itemize}
    \item Tasa de supervivencia vaquillas: \textbf{90\%}
    \item Edad primer servicio: \textbf{18 meses}
    \item Edad primer parto: \textbf{30 meses}
    \item Tasa de preñez objetivo: \textbf{80\%} (mejora gradual desde 75\% actual)
    \item Intervalo entre partos: \textbf{14 meses}
    \item Proporción hembras: \textbf{50\%}
    \item Tasa mortalidad: \textbf{3\%} anual
\end{itemize}

\textbf{CRONOGRAMA DE INTRODUCCIÓN (7 ENTREGAS ESCALONADAS):}
\begin{itemize}
    \item \textbf{2026}: Sin entregas (construcción infraestructura + establecimiento 1,200 ha Leucaena)
    \item \textbf{T3-T4 2027}: 1,000 vaquillas (500 + 500) tras maduración Leucaena
    \item \textbf{2028}: 3,000 vaquillas (1,000 T1 + 2,000 T3)
    \item \textbf{2029}: 6,000 vaquillas (3,000 T1 + 3,000 T3)
    \item \textbf{2030}: 2,000 vaquillas (T1)
    \item \textbf{Total}: 12,000 vaquillas F1 en 7 entregas escalonadas
\end{itemize}

\textbf{Sincronización crítica:} Cada entrega se sincroniza con disponibilidad de hectáreas maduras y capacidad de carga instalada.

\subsection{Proyección de Crecimiento del Hato}

\begin{table}[H]
\centering
\footnotesize
\begin{tabular}{|c|c|c|c|c|c|}
\hline
\rowcolor{sadergreen!20}
\textbf{Año} & \textbf{Vaquillas} & \textbf{Hectáreas} & \textbf{Hato} & \textbf{Nacimientos} & \textbf{Total} \\
 & \textbf{Introducidas} & \textbf{SSPi} & \textbf{Acumulado} & & \textbf{Incremento} \\
\hline
2026 & 0 & 1,200 & 0 & 0 & 0 \\
\hline
2027 & 1,000 & 2,400 & 900* & 0 & 900 \\
\hline
2028 & 3,000 & 3,600 & 3,600 & 0 & 3,600 \\
\hline
2029 & 6,000 & 4,800 & 9,000 & 360** & 9,360 \\
\hline
2030 & 2,000 & 6,000 & 10,800 & 3,744*** & 14,544 \\
\hline
\rowcolor{sadergold!30}
\multicolumn{5}{|l|}{\textbf{TOTAL PROYECTADO AL 2030}} & \textbf{14,544} \\
\hline
\end{tabular}
\caption{Proyección de Crecimiento del Hato con Infraestructura Sincronizada}
\end{table}

*Considerando 90\% supervivencia (1,000 vaquillas × 0.9) \\
**Primeros partos de cohorte 2027 (900 vaquillas × 80\% preñez × 50\% hembras) \\
***Partos de cohortes 2027-2029

% ========================================
% COMPONENTES TÉCNICOS
% ========================================
\section{Componentes Técnicos del Proyecto}

\subsection{Paquete Tecnológico Silvopastoril}

\begin{table}[H]
\centering
\footnotesize
\begin{tabular}{|l|c|c|c|}
\hline
\rowcolor{sadergreen!20}
\textbf{Componente} & \textbf{Unidad} & \textbf{Costo Unit.} & \textbf{Costo/ha} \\
\hline
\multicolumn{4}{|l|}{\textbf{Establecimiento de Pastos}} \\
\hline
Semilla \textit{Cynodon nlemfuensis} & 3 kg & \$250/kg & \$750 \\
\hline
Semilla \textit{Brachiaria brizantha} & 2 kg & \$280/kg & \$560 \\
\hline
Preparación y siembra & 4 jornales & \$180/jornal & \$720 \\
\hline
\multicolumn{4}{|l|}{\textbf{Componente Arbóreo}} \\
\hline
Plantas \textit{Leucaena leucocephala} & 150 plantas & \$8/planta & \$1,200 \\
\hline
Plantas nativas (Brosimum, Inga) & 50 plantas & \$15/planta & \$750 \\
\hline
Plantación & 6 jornales & \$180/jornal & \$1,080 \\
\hline
\multicolumn{4}{|l|}{\textbf{Infraestructura Básica}} \\
\hline
Cercos eléctricos & 1,500 m & \$45/m & \$6,750 \\
\hline
Bebederos móviles & 2 unidades & \$1,800/unidad & \$3,600 \\
\hline
Sistema de agua & 150 m tubería & \$35/m & \$5,250 \\
\hline
\multicolumn{4}{|l|}{\textbf{Insumos Biológicos}} \\
\hline
Biofertilizantes & 1 ton & \$1,200/ton & \$1,200 \\
\hline
Inoculantes & 5 dosis & \$60/dosis & \$300 \\
\hline
Capacitación Técnica & 1 productor & \$2,500 & \$2,500 \\
\hline
\rowcolor{sadergold!30}
\multicolumn{3}{|l|}{\textbf{TOTAL POR HECTÁREA}} & \textbf{\$55,573} \\
\hline
\end{tabular}
\caption{Desglose del Paquete Tecnológico Silvopastoril}
\end{table}

\subsection{Programa de Repoblamiento}

\textbf{Costo por vaquilla}: \$18,000 MXN
\begin{itemize}
    \item Vaquilla de 12-15 meses: \$15,000
    \item Transporte y manejo: \$1,500
    \item Certificación sanitaria: \$800
    \item Seguro ganadero (6 meses): \$700
\end{itemize}

\subsection{Asistencia Técnica Especializada}

\textbf{Metodología de Escuelas de Campo (ECA):}
\begin{itemize}
    \item 12 sesiones teórico-prácticas por año
    \item Temas: manejo silvopastoril, reproducción, sanidad
    \item Seguimiento técnico mensual
    \item Evaluación de adopción tecnológica
\end{itemize}

% ========================================
% PRESUPUESTO
% ========================================
\section{Presupuesto Quinquenal}

\subsection{Año 1 (2026) - Fase de Infraestructura}

\begin{table}[H]
\centering
\begin{tabular}{|l|c|c|}
\hline
\rowcolor{sadergreen!20}
\textbf{Concepto} & \textbf{Cantidad} & \textbf{Costo Total (MXN)} \\
\hline
Reconversión SSPi (30 UPP) & 1,200 ha & \$22,200,000 \\
\hline
Infraestructura (corrales, cercas) & 30 UPP & \$12,000,000 \\
\hline
Vaquillas & 0 & \$0 \\
\hline
Coordinación técnica & - & \$3,500,000 \\
\hline
\rowcolor{sadergold!30}
\textbf{TOTAL AÑO 1} & & \textbf{\$37,700,000} \\
\hline
\end{tabular}
\caption{Presupuesto Fase de Infraestructura (sin ganado)}
\end{table}

\textbf{Nota:} 2026 se dedica exclusivamente a construcción de infraestructura y establecimiento de Leucaena. No se introducen vaquillas hasta T3-T4 2027 tras período de maduración de 6-9 meses.

\subsection{Años 2-5 (2027-2030) - Escalamiento}

\begin{table}[H]
\centering
\footnotesize
\begin{tabular}{|l|c|c|c|c|c|}
\hline
\rowcolor{sadergreen!20}
\textbf{Concepto} & \textbf{2027} & \textbf{2028} & \textbf{2029} & \textbf{2030} & \textbf{Total} \\
\hline
Reconversión SSPi & \$22.2M & \$22.2M & \$22.2M & \$22.2M & \$88.8M \\
\hline
Infraestructura & \$12.0M & \$12.0M & \$12.0M & \$12.0M & \$48.0M \\
\hline
Vaquillas & \$18.0M & \$54.0M & \$108.0M & \$36.0M & \$216.0M \\
 & (1,000) & (3,000) & (6,000) & (2,000) & (12,000) \\
\hline
Asistencia técnica & \$4.0M & \$4.5M & \$5.0M & \$5.5M & \$19.0M \\
\hline
\rowcolor{sadergold!30}
\textbf{SUBTOTAL} & \textbf{\$56.2M} & \textbf{\$92.7M} & \textbf{\$147.2M} & \textbf{\$75.7M} & \textbf{\$371.8M} \\
\hline
\end{tabular}
\caption{Presupuesto Escalamiento (2027-2030)}
\end{table}

\textbf{Entregas escalonadas:} 2027 (1,000), 2028 (3,000), 2029 (6,000), 2030 (2,000) = 12,000 vaquillas F1

\subsection{Presupuesto Total Quinquenal (2026-2030)}

\begin{table}[H]
\centering
\begin{tabular}{|l|c|c|}
\hline
\rowcolor{sadergreen!20}
\textbf{Concepto} & \textbf{Monto (MXN)} & \textbf{Porcentaje} \\
\hline
Sistemas silvopastoriles (6,000 ha) & \$111,000,000 & 27.1\% \\
\hline
Infraestructura (120 UPP) & \$60,000,000 & 14.6\% \\
\hline
Repoblamiento ganadero (12,000) & \$216,000,000 & 52.7\% \\
\hline
Asistencia técnica & \$22,500,000 & 5.5\% \\
\hline
\rowcolor{sadergold!30}
\textbf{TOTAL PROYECTO} & \textbf{\$409,500,000} & \textbf{100\%} \\
\hline
\end{tabular}
\caption{Distribución del Presupuesto Total (Revisado)}
\end{table}

\textbf{Incremento presupuestal:} El presupuesto aumenta de \$245.25M a \$409.5M para incluir infraestructura adecuada (corrales, bebederos, cercas) y 1,200 ha adicionales (5,000 → 6,000 ha).

\subsection{Esquema de Financiamiento}

\begin{itemize}
    \item \textbf{Federal (60\%)}: \$245,700,000
    \item \textbf{Estatal (30\%)}: \$122,850,000
    \item \textbf{Productores (10\%)}: \$40,950,000
\end{itemize}

\textbf{Justificación:} El esquema 60-30-10 refleja la importancia estratégica del proyecto para seguridad alimentaria nacional y compromiso de pequeños productores.

% ========================================
% IMPACTOS ESPERADOS
% ========================================
\section{Impactos Esperados}

\textbf{Impactos Productivos (2030)}

\begin{itemize}
    \item \textbf{Incremento del hato}: 14,544 cabezas (12,000 introducidas + 2,544 crías) = +2.4\% del inventario estatal
    \item \textbf{Carga animal objetivo}: 3.5-4.0 UA/ha vs 0.4 UA/ha actual (incremento 775-900\%)
    \item \textbf{Mejora productividad cárnica}: 35-45\% por animal (200 kg → 270-290 kg peso vivo)
    \item \textbf{Aumento producción láctea}: 4-6 L/vaca/día adicionales en sistema doble propósito
    \item \textbf{Cobertura}: 120 UPP beneficiadas directamente (30 productores/año × 4 años)
\end{itemize}

\subsection{Impactos Ambientales}

\begin{itemize}
    \item \textbf{Captura de carbono}: 90,000-150,000 ton CO\textsubscript{2}eq acumuladas (6,000 ha × 15-25 ton/ha)
    \item \textbf{Reducción emisiones metano}: 20-30\% por unidad animal mediante inclusión Leucaena
    \item \textbf{Conservación biodiversidad}: Corredores biológicos en 6,000 ha con árboles nativos
    \item \textbf{Mejora calidad suelos}: Incremento materia orgánica 15-25\% por fijación N\textsubscript{2}
\end{itemize}

\subsection{Impactos Socioeconómicos}

\begin{itemize}
    \item \textbf{Empleos generados}: 600 empleos directos (5/UPP × 120 UPP), 1,500 indirectos
    \item \textbf{Incremento ingresos}: 25-35\% en UPP participantes (\$15,000 → \$20,000/mes)
    \item \textbf{Fortalecimiento cadenas}: Mejor integración productor-industria con volumen crítico
    \item \textbf{Transferencia tecnológica}: Modelo replicable en Península de Yucatán (Campeche, Quintana Roo)
\end{itemize}

% ========================================
% ANÁLISIS COSTO-BENEFICIO
% ========================================
\section{Análisis Costo-Beneficio}

\subsection{Inversión: \$409.5M MXN (5 años)}

\subsection{Beneficios Proyectados}

\begin{table}[H]
\centering
\begin{tabular}{|l|c|}
\hline
\rowcolor{sadergreen!20}
\textbf{Concepto de Beneficio} & \textbf{Valor (MXN)} \\
\hline
Incremento valor producción (5 años) & \$280,000,000 \\
\hline
Servicios ambientales (valor carbono) & \$65,000,000 \\
\hline
Empleos generados (masa salarial) & \$350,000,000 \\
\hline
\rowcolor{sadergold!30}
\textbf{RETORNO TOTAL ESTIMADO} & \textbf{\$695,000,000} \\
\hline
\rowcolor{sadergold!30}
\textbf{TIR} & \textbf{24-28\% anual} \\
\hline
\rowcolor{sadergold!30}
\textbf{ROI} & \textbf{1.70} \\
\hline
\end{tabular}
\caption{Análisis de Retorno de Inversión (Revisado)}
\end{table}

\textbf{Nota:} A pesar del incremento presupuestal (+67\%), el ROI se mantiene atractivo (1.70) con TIR superior al costo de oportunidad del capital.

% ========================================
% FACTORES DE RIESGO
% ========================================
\section{Factores de Riesgo y Mitigación}

\subsection{Riesgos Identificados}

\begin{enumerate}
    \item \textbf{Climáticos}: Sequías, huracanes
    \item \textbf{Sanitarios}: Brotes epidémicos
    \item \textbf{Económicos}: Fluctuaciones precios
    \item \textbf{Técnicos}: Baja adopción tecnológica
\end{enumerate}

\subsection{Estrategias de Mitigación}

\begin{enumerate}
    \item \textbf{Seguros paramétricos} para riesgos climáticos
    \item \textbf{Protocolos sanitarios} preventivos
    \item \textbf{Contratos de compra-venta} a precios mínimos
    \item \textbf{Programa intensivo} de transferencia tecnológica
\end{enumerate}

% ========================================
% CRONOGRAMA
% ========================================
\section{Cronograma de Implementación}

\subsection{Fase I - Infraestructura (Enero-Diciembre 2026)}

\begin{itemize}
    \item \textbf{T1-T2}: Selección 30 UPP piloto, diseño participativo
    \item \textbf{T3}: Construcción infraestructura (corrales, bebederos, cercas)
    \item \textbf{T4}: Establecimiento 1,200 ha Leucaena leucocephala
    \item \textbf{Resultado}: 30 UPP con infraestructura completa, SIN ganado
\end{itemize}

\subsection{Fase II - Maduración y Primera Entrega (2027)}

\begin{itemize}
    \item \textbf{T1-T2}: Maduración Leucaena (6-9 meses), establecimiento 1,200 ha adicionales
    \item \textbf{T3}: Primera entrega 500 vaquillas F1
    \item \textbf{T4}: Segunda entrega 500 vaquillas F1 (total 1,000)
    \item \textbf{Resultado}: 2,400 ha SSPi establecidas, 1,000 vaquillas en sistema
\end{itemize}

\subsection{Fase III - Escalamiento (2028-2029)}

\begin{itemize}
    \item \textbf{2028}: Establecimiento 1,200 ha + entregas 1,000 (T1) + 2,000 (T3) vaquillas
    \item \textbf{2029}: Establecimiento 1,200 ha + entregas 3,000 (T1) + 3,000 (T3) vaquillas
    \item \textbf{Resultado}: 4,800 ha SSPi, 10,000 vaquillas acumuladas
\end{itemize}

\subsection{Fase IV - Consolidación (2030)}

\begin{itemize}
    \item \textbf{T1}: Última entrega 2,000 vaquillas (total 12,000)
    \item \textbf{T2-T4}: Completar 1,200 ha finales (total 6,000 ha)
    \item \textbf{Resultado}: 120 UPP consolidadas, 12,000 vaquillas + progenie, evaluación integral
\end{itemize}

\textbf{Principio rector:} Infraestructura → Establecimiento → Maduración → Ganado (nunca invertir el orden)

% ========================================
% MARCO INSTITUCIONAL
% ========================================
\section{Marco Institucional y Coordinación}

\subsection{Alianzas Estratégicas}

\begin{itemize}
    \item \textbf{SADER}: Financiamiento y coordinación general
    \item \textbf{UADY}: Investigación y desarrollo tecnológico
    \item \textbf{TNC}: Metodologías ambientales y monitoreo
    \item \textbf{FIRA}: Esquemas crediticios complementarios
    \item \textbf{Gobierno Estatal}: Contrapartida y facilitación
\end{itemize}

\subsection{Estructura Operativa}

\begin{itemize}
    \item \textbf{Coordinación General}: Delegación SADER Yucatán
    \item \textbf{Componente Técnico}: UADY-TNC
    \item \textbf{Seguimiento}: Sistema de monitoreo georreferenciado
    \item \textbf{Evaluación}: Consultor externo independiente
\end{itemize}

% ========================================
% CONCLUSIONES
% ========================================
\section{Conclusiones}

Este proyecto \textbf{fundamentado en datos oficiales SIAP y principios zootécnicos sólidos} representa una oportunidad estratégica para modernizar la ganadería yucateca mediante un enfoque \textbf{logísticamente viable y ambientalmente responsable}.

Con una inversión de \textbf{\$409.5M MXN}, el proyecto puede generar \textbf{impactos significativos} tanto productivos como ambientales, estableciendo un \textbf{modelo replicable} para otras regiones de México.

La colaboración \textbf{SADER-UADY-TNC} garantiza el \textbf{rigor técnico} necesario, mientras que el enfoque de \textbf{sistemas silvopastoriles} posiciona a Yucatán como \textbf{referente nacional} en ganadería climáticamente inteligente.

\textbf{LECCIÓN CRÍTICA:} El proyecto es técnica, económica y ambientalmente viable SOLAMENTE cuando se respeta la secuencia \textbf{infraestructura → establecimiento → maduración (6-9 meses) → ganado}. Invertir este orden resultaría en fracaso operativo.

\textbf{La viabilidad depende de la disciplina en la ejecución, no solo del diseño.}

\end{document}