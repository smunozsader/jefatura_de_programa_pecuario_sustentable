\documentclass[11pt,a4paper]{article}
\usepackage[utf8]{inputenc}
\usepackage[T1]{fontenc}
\usepackage[spanish]{babel}
\usepackage{lmodern}
\usepackage{xcolor}
\usepackage{booktabs}
\usepackage{multirow}
\usepackage{array}
\usepackage{geometry}
\geometry{margin=2.5cm}
\usepackage{fancyhdr}
\usepackage{lastpage}
\usepackage{pdflscape}
\usepackage{colortbl}

\definecolor{verdeSader}{RGB}{0,102,51}
\definecolor{grisClaro}{gray}{0.95}

\pagestyle{fancy}
\fancyhf{}
\rhead{\textcolor{verdeSader}{\textbf{PEF 2026 – Ganadería Sustentable}}}
\lhead{Resumen Ejecutivo}
\cfoot{Página \thepage\ de \pageref{LastPage}}

\begin{document}

\begin{center}
    {\LARGE\textbf{Presupuesto Etiquetado para Ganadería Sustentable}}\\[0.4cm]
    {\large\textbf{Presupuesto de Egresos de la Federación 2026}}\\[0.3cm]
    {\small Resumen Ejecutivo – Noviembre 2025}\\[1cm]
\end{center}

\begin{tabular}{p{4.8cm} p{9.2cm}}
\arrayrulecolor{verdeSader}
\toprule
\textbf{Ramo presupuestal}          & Ramo 20 – Secretaría de Agricultura y Desarrollo Rural (SADER) \\
\textbf{Presupuesto total SADER 2026} & \textbf{109,456 millones de pesos} (+5.2\% real vs 2025) \\
\textbf{Recursos etiquetados para ganadería sustentable} & \textbf{≈ 18,500 millones de pesos} (18\% del ramo) \\
\textbf{Operación principal}        & Programa Especial Concurrente para el Desarrollo Rural Sustentable (PECDRS) \\
\bottomrule
\end{tabular}

\vspace{1.2cm}

\textbf{\large Programas con presupuesto explícito para ganadería sustentable 2026}

\begin{table}[h!]
\centering
\rowcolors{2}{grisClaro}{white}
\begin{tabular}{>{\raggedright\arraybackslash}p{5.8cm} >{\centering\arraybackslash}p{3.2cm} >{\raggedright\arraybackslash}p{5cm}}
\toprule
\textbf{Programa / Componente} & \textbf{Presupuesto 2026 (mdp)} & \textbf{Concurrencia con Entidades Federativas} \\
\midrule
Programa de Fomento a la Agricultura, Ganadería, Pesca y Acuacultura (S304) & 12,000 (≈4,500 para ganadería) & Sí – Convenios de Concertación y FOFAE \\
\rowcolor{verdeSader!10}
Componente Bienestar para Pequeños y Medianos Ganaderos (Producción para el Bienestar) & 6,500 & Sí – Convenios de Coordinación 2025-2030 \\
Crédito Ganadero a la Palabra (integrado en S304) & 2,000 & Sí – Ventanillas estatales y agentes técnicos \\
\rowcolor{verdeSader!10}
\textbf{Total aproximado etiquetado} & \textbf{≈ 18,500} & \\
\bottomrule
\end{tabular}
\end{table}

\vspace{1cm}

\textbf{\large Principales características de la concurrencia}

\begin{itemize}
    \item Todos los programas son \textbf{sujetos a convenios específicos con las 32 entidades federativas}.
    \item Operan bajo el \textbf{Programa Especial Concurrente para el Desarrollo Rural Sustentable (Anexo 11 del Decreto)}.
    \item Requieren aportación estatal promedio del 25\% (condiciona transferencia federal).
    \item Incluyen recursos para \textbf{SINIIGA / SINIDA} (trazabilidad y combate al abigeato).
    \item Fortalecen el \textbf{Plan Binacional México–EE.UU. contra tuberculosis bovina (T-MEC)}.
\end{itemize}

\vspace{1cm}

\begin{center}
\begin{tabular}{|>{\columncolor{verdeSader!15}}c|c|}
\hline
\rowcolor{verdeSader}
\textcolor{white}{\textbf{Fortalezas}} & \textcolor{white}{\textbf{Oportunidades}} \\
\hline
\multirow{3}{*}{\parbox{6.5cm}{\centering Presupuesto explícito y creciente\\Mecanismos concurrentes consolidados\\Vinculación directa con trazabilidad y T-MEC}} 
& \parbox{6.5cm}{\centering Incrementar 10\% en servicios técnicos veterinarios\\Mayor desglose territorial en 2027\\Fortalecer ventanillas únicas estatales} \\
\hline
\end{tabular}
\end{center}

\vspace{1.5cm}

{\small
\textbf{Fuentes:} Decreto PEF 2026 (DOF 25-nov-2025), Anexos 11, 24 y 25; Reglas de Operación SADER 2026 (DOF 30-ene-2026).\\
Elaborado por equipo multidisciplinario en desarrollo agropecuario sustentable, salud animal y políticas públicas rurales.
}

\end{document}