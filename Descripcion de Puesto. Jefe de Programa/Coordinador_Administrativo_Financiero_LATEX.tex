\documentclass[12pt,letterpaper]{article}
\usepackage[utf8]{inputenc}
\usepackage[spanish]{babel}
\usepackage{geometry}
\usepackage{graphicx}
\usepackage{fancyhdr}
\usepackage{setspace}
\usepackage{lastpage}
\usepackage{parskip}
\usepackage{booktabs}
\usepackage{array}
\usepackage{multirow}
\usepackage{longtable}
\usepackage{float}
\usepackage{xcolor}
\usepackage{colortbl}
\usepackage{amsmath}
\usepackage{ragged2e}

% Colores SADER
\definecolor{saderblue}{RGB}{0,51,102}
\definecolor{sadergreen}{RGB}{34,139,34}
\definecolor{sadergray}{RGB}{128,128,128}
\definecolor{sadergold}{RGB}{255,215,0}

% Márgenes exactos SADER
\geometry{top=2.5cm,bottom=2.5cm,left=3cm,right=3cm,headheight=20pt}

% Encabezado y pie de página
\pagestyle{fancy}
\fancyhf{}
\rfoot{\thepage}
\renewcommand{\headrulewidth}{0pt}
\fancyhead[L]{\includegraphics[width=2.8cm]{logo_sader.png}}

\begin{document}

% ========================================
% PORTADA OFICIAL
% ========================================
\begin{titlepage}
    \centering
    \vspace*{0.3cm}
    \includegraphics[width=0.25\textwidth]{logo_sader.png}\\[0.8cm]
    
    \vspace{0.4cm}
    {\normalsize\bfseries Perfil Profesional y Descripción Técnica:\par}
    \vspace{0.6cm}
    
    {\LARGE\bfseries COORDINADOR\par}
    {\LARGE\bfseries ADMINISTRATIVO-FINANCIERO\par}
    \vspace{0.5cm}
    {\Large Especialista en FOFAY y Accountability:\par}
    \vspace{0.3cm}
    {\normalsize • Fideicomiso FOFAY (\$52.78 millones MXN)\par}
    {\normalsize • Administración Financiera Especializada\par}
    {\normalsize • Accountability y Transparencia\par}
    {\normalsize • Auditoría y Contraloría Interna\par}
    {\normalsize • Sistemas de Información Gerencial\par}
    \vspace{0.5cm}
    {\Large Macroproyecto Renacimiento Ganadero Maya\par}
    {\Large Yucatán 2026-2030\par}
    
    \vfill
    
    {\normalsize Mérida, Yucatán, 28 de noviembre de 2025\par}
    \vspace{0.2cm}
    {\normalsize SADER REPRESENTACIÓN ESTATAL YUCATÁN\par}
    {\normalsize Subdelegación Agropecuaria\par}
    \vspace{0.2cm}
    {\normalsize Código: COORD-ADM-FIN-008\par}
    {\normalsize Secretaría de Agricultura y Desarrollo Rural (SADER)\par}
\end{titlepage}

% ========================================
% ÍNDICE AUTOMÁTICO
% ========================================
\clearpage
\thispagestyle{empty}
\vspace*{3cm}
{\large\bfseries Contenido}\\[2cm]

\tableofcontents

\clearpage
\setcounter{page}{3}

% ========================================
% CONTENIDO
% ========================================

\section{Caracterización Institucional del Puesto}

\justifying

\subsection{Identificación Administrativa}

\begin{table}[H]
\centering
\caption{Datos Generales del Coordinador Administrativo-Financiero}
\label{tab:datos_generales}
\begin{tabular}{p{5cm}p{8cm}}
\toprule
\rowcolor{sadergreen!20}
\textbf{Campo} & \textbf{Especificación Técnica} \\
\midrule
Denominación oficial & Coordinador Administrativo-Financiero FOFAY \\
Código de identificación & COORD-ADM-FIN-008 \\
Dependencia jerárquica & Jefe de Programa de Producción Pecuaria Sustentable \\
Línea de reporte directo & Coordinación administrativa directa con Jefe de Programa \\
Ámbito territorial & Estado de Yucatán (\$814.9 M total, \$52.78 M operativo) \\
Categoría de plaza & Coordinador especializado nivel TC-15 tabulador SADER \\
Salario anual & \$500,000 MXN (financiado vía FOFAY) \\
Especialización primaria & Administración fiduciaria y accountability gubernamental \\
\bottomrule
\end{tabular}
\end{table}

\subsection{Contexto Programático Específico}

Este puesto estratégico coordina la administración fiduciaria de \$814.9 millones MXN del Macroproyecto a través del FOFAY (Fideicomiso Fondo de Fomento Agropecuario Yucatán), garantizando transparencia, eficiencia y accountability en la ejecución financiera del programa federal más ambicioso en ganadería sustentable. La posición lidera sistemas de información gerencial que integran 5 componentes técnicos, coordina auditorías internas y externas, y asegura cumplimiento normativo con estándares federales, estatales e internacionales para un proyecto que involucra financiamiento tripartito (60\% federal - 30\% estatal - 10\% productores).

\section{Responsabilidades Principales}

\subsection{Administración del Fideicomiso FOFAY}

\textbf{1. Coordinación Financiera Integral:}
\begin{itemize}
    \item Administrar fideicomiso FOFAY de \$52.78 millones MXN (5 años)
    \item Coordinar flujos de caja tripartitos (federal/estatal/productores)
    \item Supervisar dispersión de \$10.556 millones anuales operativos
    \item Gestionar cuentas bancarias especializadas por componente
    \item Coordinar con fiduciario institucional (banco desarrollo)
    \item Implementar controles internos de tesorería
    \item Optimizar rendimientos financieros de reservas
\end{itemize}

\textbf{2. Presupuestación y Planeación Financiera:}
\begin{itemize}
    \item Elaborar presupuestos anuales por componente y actividad
    \item Coordinar modificaciones presupuestales con autoridades
    \item Implementar sistema de devengado contable
    \item Gestionar calendarios de ministraciones federales/estatales
    \item Supervisar cumplimiento de metas físico-financieras
    \item Elaborar proyecciones de flujo de efectivo quinquenales
\end{itemize}

\subsection{Sistemas de Información y Control}

\textbf{3. Plataforma de Información Gerencial:}
\begin{itemize}
    \item Desarrollar dashboard ejecutivo tiempo real
    \item Implementar sistema ERP especializado agropecuario
    \item Coordinar integración con sistemas SADER (e.SADER)
    \item Generar reportes automáticos multi-nivel
    \item Mantener trazabilidad completa de recursos
    \item Coordinar con plataforma SIG para georreferenciación
\end{itemize}

\textbf{4. Contabilidad y Registro Especializado:}
\begin{itemize}
    \item Supervisar contabilidad gubernamental (CONAC)
    \item Implementar registro contable por centro de costos
    \item Coordinar depreciación de activos especializados
    \item Gestionar inventarios de equipos y materiales
    \item Supervisar conciliaciones bancarias automatizadas
    \item Elaborar estados financieros mensuales especializados
\end{itemize}

\subsection{Accountability y Transparencia}

\textbf{5. Auditoría y Control Interno:}
\begin{itemize}
    \item Coordinar auditorías internas trimestrales
    \item Facilitar auditorías externas (SFP, ASF, SEAY)
    \item Implementar matriz de riesgos operativos y financieros
    \item Supervisar cumplimiento normativo multi-nivel
    \item Coordinar con órganos internos de control
    \item Gestionar sistema de quejas y denuncias ciudadanas
\end{itemize}

\textbf{6. Transparencia y Rendición de Cuentas:}
\begin{itemize}
    \item Publicar información en Plataforma Nacional de Transparencia
    \item Elaborar informes trimestrales de avances físico-financieros
    \item Coordinar con Comité de Transparencia estatal
    \item Atender solicitudes de información ciudadana
    \item Implementar mecanismos de participación social
    \item Documentar impactos sociales y ambientales
\end{itemize}

\section{Perfil del Puesto}

\subsection{Requisitos Académicos y Experiencia}

\begin{table}[H]
\centering
\caption{Requisitos del Coordinador Administrativo-Financiero}
\label{tab:requisitos}
\begin{tabular}{p{4cm}p{9cm}}
\toprule
\rowcolor{sadergreen!20}
\textbf{Requisito} & \textbf{Detalle} \\
\midrule
Formación académica & Licenciatura en Administración, Contaduría o Economía. Especialización en administración pública o gerencia de proyectos \\
Experiencia mínima & 8 años en administración financiera gubernamental, 3 años en fideicomisos públicos \\
Conocimientos indispensables & CONAC, Ley de Disciplina Financiera, administración de fideicomisos, auditoria gubernamental, transparencia \\
Certificaciones obligatorias & Contador Público certificado, conocimientos en Ley General de Transparencia \\
Idiomas & Inglés básico para coordinación con organismos internacionales \\
Habilidades técnicas & ERP gubernamental, Excel avanzado, sistemas contables, análisis financiero \\
Disponibilidad & 90\% oficina administrativa, disponibilidad para auditorías \\
\bottomrule
\end{tabular}
\end{table}

\subsection{Competencias Técnicas Específicas}

\textbf{Competencias obligatorias:}
\begin{itemize}
    \item Administración de fideicomisos públicos especializados
    \item Contabilidad gubernamental (CONAC) y disciplina financiera
    \item Sistemas ERP y plataformas de información gerencial
    \item Auditoría interna y gestión de riesgos institucionales
    \item Marco normativo de transparencia y accountability
    \item Análisis financiero y proyecciones presupuestales
\end{itemize}

\textbf{Competencias deseables:}
\begin{itemize}
    \item Experiencia en proyectos de desarrollo rural
    \item Coordinación con organismos internacionales
    \item Gestión de financiamiento climático
    \item Certificaciones en project management (PMP)
    \item Conocimientos en evaluación de impacto social
    \item Manejo de bases de datos avanzadas
\end{itemize}

\section{Indicadores de Desempeño}

\begin{table}[H]
\centering
\caption{Métricas del Coordinador Administrativo-Financiero}
\label{tab:kpis}
\begin{tabular}{p{6cm}p{3cm}p{3cm}}
\toprule
\rowcolor{sadergreen!20}
\textbf{Indicador} & \textbf{Meta} & \textbf{Frecuencia} \\
\midrule
Recursos ejercidos vs programados & $\geq$95\% & Mensual \\
Tiempo de dispersión de recursos & $\leq$5 días & Semanal \\
Observaciones de auditoría & 0 & Anual \\
Cumplimiento normativo & 100\% & Permanente \\
Informes transparencia publicados & 4/año & Trimestral \\
Satisfacción beneficiarios & $\geq$90\% & Semestral \\
Estados financieros oportunos & 12/año & Mensual \\
Eficiencia administrativa & $\leq$6.5\%* & Anual \\
\bottomrule
\end{tabular}
\end{table}

*\textit{Gastos administrativos como \% del presupuesto total}

\section{Metodología de Trabajo}

\subsection{Ciclo de Administración Financiera}

\textbf{Fase 1: Planeación y Presupuestación}
\begin{itemize}
    \item Elaboración presupuesto anual participativo
    \item Calendarización de ministraciones por fuente
    \item Definición de metas físico-financieras por trimestre
    \item Establecimiento de controles internos especializados
    \item Coordinación con Secretaría de Finanzas estatal
\end{itemize}

\textbf{Fase 2: Ejecución y Monitoreo}
\begin{itemize}
    \item Autorización y dispersión de recursos diarios
    \item Monitoreo tiempo real de avances financieros
    \item Seguimiento de cumplimiento de metas por componente
    \item Identificación y mitigación de riesgos operativos
    \item Coordinación con equipo técnico multidisciplinario
\end{itemize}

\textbf{Fase 3: Control y Evaluación}
\begin{itemize}
    \item Auditorías internas trimestrales especializadas
    \item Elaboración de informes de rendición de cuentas
    \item Evaluación de eficiencia y eficacia administrativa
    \item Implementación de mejoras en procesos
    \item Preparación para auditorías externas
\end{itemize}

\subsection{Cronograma Anual de Actividades}

\textbf{Enero-Marzo (Planeación):}
\begin{itemize}
    \item Elaboración presupuesto siguiente ejercicio
    \item Auditoría interna Q4 ejercicio anterior
    \item Cierre contable y estados financieros anuales
    \item Planeación estratégica administrativa
\end{itemize}

\textbf{Abril-Junio (Implementación):}
\begin{itemize}
    \item Inicio nuevo ejercicio presupuestal
    \item Implementación mejoras en procesos
    \item Auditoría interna Q1 y evaluación
    \item Capacitación equipo en nuevos procedimientos
\end{itemize}

\textbf{Julio-Septiembre (Seguimiento):}
\begin{itemize}
    \item Evaluación semestral físico-financiera
    \item Auditoría interna Q2 y ajustes
    \item Modificaciones presupuestales necesarias
    \item Actualización sistemas de información
\end{itemize}

\textbf{Octubre-Diciembre (Evaluación):}
\begin{itemize}
    \item Auditoría interna Q3 y seguimiento
    \item Evaluación anual de resultados
    \item Preparación cierre ejercicio fiscal
    \item Elaboración informes anuales especializados
\end{itemize}

\section{Coordinación Institucional}

El Coordinador Administrativo-Financiero coordinará con:

\begin{itemize}
    \item \textbf{Fiduciario FOFAY:} Institución bancaria administradora
    \item \textbf{SADER Federal:} Programa Especial Concurrente (PEC)
    \item \textbf{Secretaría de Finanzas Yucatán:} Recursos estatales
    \item \textbf{Organizaciones Productoras:} Aportaciones y transparencia
    \item \textbf{Auditoría Superior de la Federación:} Auditorías externas
    \item \textbf{Órgano Interno de Control:} Seguimiento normativo
\end{itemize}

\section{Infraestructura Administrativa}

\subsection{Sistemas y Plataformas}

\begin{itemize}
    \item \textbf{ERP Especializado:} Sistema integral de gestión administrativa
    \item \textbf{Plataforma FOFAY:} Acceso directo a cuentas fiduciarias
    \item \textbf{e.SADER:} Integración con sistemas federales
    \item \textbf{SIIF Estatal:} Coordinación con sistemas estatales
    \item \textbf{Portal Transparencia:} Publicación automática de información
\end{itemize}

\section{Marco Normativo}

\subsection{Legislación Aplicable}

\begin{itemize}
    \item \textbf{Federal:} Ley de Disciplina Financiera, Ley General de Transparencia
    \item \textbf{Contable:} Postulados Básicos de Contabilidad Gubernamental (CONAC)
    \item \textbf{Fiduciaria:} Ley de Instituciones de Crédito, Ley Orgánica NAFIN
    \item \textbf{Auditoría:} Normas de Auditoría Gubernamental, Ley de Fiscalización
    \item \textbf{Estatal:} Ley de Disciplina Financiera del Estado de Yucatán
\end{itemize}

\section{Condiciones Laborales}

\subsection{Condiciones Específicas}

\begin{itemize}
    \item \textbf{Base:} Oficinas administrativas FOFAY en Mérida
    \item \textbf{Horario:} Lunes a viernes, horario administrativo estándar
    \item \textbf{Disponibilidad:} Auditorías y cierres contables (fines de semana)
    \item \textbf{Salario anual:} \$500,000 MXN + prestaciones superiores
    \item \textbf{Capacitación:} 60 horas anuales en normatividad actualizada
    \item \textbf{Bonificación:} Incentivos por cumplimiento de metas
\end{itemize}

\section{Impacto Institucional}

\subsection{Contribución Estratégica}

El Coordinador Administrativo-Financiero garantiza la viabilidad institucional del macroproyecto mediante administración fiduciaria eficiente y transparente. Su gestión determina la credibilidad gubernamental del programa y su capacidad de replicación en otros estados.

La implementación de sistemas de información gerencial de vanguardia posiciona al programa como referente nacional en accountability y transparencia de recursos públicos especializados.

\section{Conclusiones Técnicas}

La coordinación administrativa-financiera constituye el backbone institucional que permite la ejecución técnica exitosa de los 5 componentes del macroproyecto. La gestión eficiente de \$814.9 millones MXN requiere expertise especializado en administración fiduciaria y sistemas de accountability que garanticen la legitimidad social y política del programa.

\vspace{2cm}

\noindent\textbf{Jefe de Programa de Producción Pecuaria Sustentable}

\vspace{1cm}

\noindent\textbf{Subdelegado Agropecuario - SADER Yucatán}

\vspace{2cm}

\noindent\textit{Mérida, Yucatán, a 28 de noviembre de 2025}

\end{document}