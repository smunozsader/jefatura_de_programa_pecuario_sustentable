\documentclass[12pt,letterpaper]{article}
\usepackage[utf8]{inputenc}
\usepackage[spanish]{babel}
\usepackage{geometry}
\usepackage{graphicx}
\usepackage{fancyhdr}
\usepackage{setspace}
\usepackage{lastpage}
\usepackage{parskip}
\usepackage{booktabs}
\usepackage{array}
\usepackage{multirow}
\usepackage{longtable}
\usepackage{float}
\usepackage{xcolor}
\usepackage{colortbl}
\usepackage{amsmath}
\usepackage{ragged2e}

% Colores SADER
\definecolor{saderblue}{RGB}{0,51,102}
\definecolor{sadergreen}{RGB}{34,139,34}
\definecolor{sadergray}{RGB}{128,128,128}
\definecolor{sadergold}{RGB}{255,215,0}

% Márgenes exactos SADER
\geometry{top=2.5cm,bottom=2.5cm,left=3cm,right=3cm,headheight=20pt}

% Encabezado y pie de página
\pagestyle{fancy}
\fancyhf{}
\rfoot{\thepage}
\renewcommand{\headrulewidth}{0pt}
\fancyhead[L]{\includegraphics[width=2.8cm]{logo_sader.png}}

\begin{document}

% ========================================
% PORTADA OFICIAL
% ========================================
\begin{titlepage}
    \centering
    \vspace*{0.3cm}
    \includegraphics[width=0.25\textwidth]{logo_sader.png}\\[0.8cm]
    
    \vspace{0.4cm}
    {\normalsize\bfseries Perfil Profesional y Descripción Técnica:\par}
    \vspace{0.6cm}
    
    {\LARGE\bfseries MÉDICO VETERINARIO ZOOTECNISTA\par}
    {\LARGE\bfseries ESPECIALISTA EN SANIDAD ANIMAL\par}
    \vspace{0.5cm}
    {\Large Coordinador de Campañas Sanitarias TBC y GBG:\par}
    \vspace{0.3cm}
    {\normalsize • Tuberculosis Bovina (Protocolo APHIS-USDA)\par}
    {\normalsize • Gusano Barrenador del Ganado (Planta Mosca Estéril)\par}
    {\normalsize • Trazabilidad SINIIGA y Certificación T-MEC\par}
    {\normalsize • Vigilancia Epidemiológica Activa\par}
    {\normalsize • Protocolos Sanitarios Binacionales\par}
    \vspace{0.5cm}
    {\Large Macroproyecto Renacimiento Ganadero Maya\par}
    {\Large Yucatán 2026-2030\par}
    
    \vfill
    
    {\normalsize Mérida, Yucatán, 28 de noviembre de 2025\par}
    \vspace{0.2cm}
    {\normalsize SADER REPRESENTACIÓN ESTATAL YUCATÁN\par}
    {\normalsize Subdelegación Agropecuaria\par}
    \vspace{0.2cm}
    {\normalsize Código: MVZ-SA-YUC-002\par}
    {\normalsize Secretaría de Agricultura y Desarrollo Rural (SADER)\par}
\end{titlepage}

% ========================================
% ÍNDICE AUTOMÁTICO
% ========================================
\clearpage
\thispagestyle{empty}
\vspace*{3cm}
{\large\bfseries Contenido}\\[2cm]

\tableofcontents

\clearpage
\setcounter{page}{3}

% ========================================
% CONTENIDO
% ========================================

\section{Caracterización Institucional del Puesto}

\justifying

\subsection{Identificación Administrativa}

\begin{table}[H]
\centering
\caption{Datos Generales del Puesto Especializado MVZ Sanidad Animal}
\label{tab:datos_generales}
\begin{tabular}{p{5cm}p{8cm}}
\toprule
\rowcolor{sadergreen!20}
\textbf{Campo} & \textbf{Especificación Técnica} \\
\midrule
Denominación oficial & Médico Veterinario Zootecnista - Especialista en Sanidad Animal \\
Código de identificación & MVZ-SA-YUC-002 \\
Dependencia jerárquica & Jefe de Programa de Producción Pecuaria Sustentable \\
Línea de reporte directo & Coordinación técnica con SENASICA y Jefe de Programa SADER Yucatán \\
Ámbito territorial & Estado de Yucatán (106 municipios, 1,075 UPP objetivo) \\
Categoría de plaza & Técnico especializado nivel TC-12 tabulador SADER \\
Salario anual & \$420,000 MXN (financiado vía FOFAY) \\
Especialización primaria & Campañas sanitarias TBC y GBG, protocolos APHIS-USDA \\
\bottomrule
\end{tabular}
\end{table}

\subsection{Contexto Programático Específico}

El puesto se enmarca como componente técnico especializado del Macroproyecto "Renacimiento Ganadero Maya" 2026-2030, con responsabilidad directa sobre los aspectos de sanidad animal que condicionan la certificación T-MEC y el acceso a mercados internacionales. La posición es crítica para el éxito del programa, ya que coordina dos de los componentes más complejos: la erradicación del Gusano Barrenador del Ganado (GBG) mediante la Planta de Mosca Estéril (\$300 MDP) y la certificación de tuberculosis bovina bajo protocolos APHIS-USDA (\$51.5 MDP).

\section{Responsabilidades Principales}

\subsection{Coordinación de Campañas Sanitarias}

\textbf{1. Tuberculosis Bovina (TBC) - Protocolo T-MEC:}
\begin{itemize}
    \item Coordinar la certificación de las 1,075 UPP bajo estándares APHIS-USDA
    \item Supervisar aplicación de pruebas diagnósticas (tuberculina PPD)
    \item Gestionar trazabilidad individual mediante SINIIGA
    \item Coordinar con laboratorios certificados OIE para confirmación diagnóstica
    \item Elaborar reportes epidemiológicos mensuales para SENASICA
    \item Mantener actualizada la base de datos de hatos certificados TBC-libre
\end{itemize}

\textbf{2. Gusano Barrenador del Ganado (GBG) - Planta Mosca Estéril:}
\begin{itemize}
    \item Coordinar la vigilancia epidemiológica activa en 106 municipios
    \item Supervisar recolección y envío de muestras larvarias al laboratorio
    \item Gestionar logística de liberaciones de moscas estériles (250M/semana)
    \item Coordinar con flota aérea especializada para dispersión sistemática
    \item Monitorear indicadores de erradicación (densidad poblacional GBG)
    \item Elaborar protocolos de bioseguridad para la planta de irradiación
\end{itemize}

\subsection{Trazabilidad y Certificación Digital}

\textbf{3. Gestión de Plataformas Digitales:}
\begin{itemize}
    \item Operar la plataforma CESO (Certificación Sanitaria Online) APHIS-USDA
    \item Mantener actualizado el registro SINIIGA de las 1,075 UPP participantes
    \item Generar certificados sanitarios para exportación bajo T-MEC
    \item Coordinar con autoridades binacionales para validación de protocolos
    \item Supervisar cumplimiento de estándares de trazabilidad individual
\end{itemize}

\subsection{Coordinación Técnica y Capacitación}

\textbf{4. Capacitación Especializada:}
\begin{itemize}
    \item Capacitar a productores en protocolos sanitarios preventivos
    \item Coordinar con MVZ locales para implementación de campañas
    \item Desarrollar material didáctico especializado en sanidad animal
    \item Supervisar aplicación correcta de protocolos de bioseguridad
    \item Evaluar y certificar competencias técnicas de personal de campo
\end{itemize}

\section{Perfil del Puesto}

\subsection{Requisitos Académicos y Experiencia}

\begin{table}[H]
\centering
\caption{Requisitos Específicos del MVZ Sanidad Animal}
\label{tab:requisitos}
\begin{tabular}{p{4cm}p{9cm}}
\toprule
\rowcolor{sadergreen!20}
\textbf{Requisito} & \textbf{Detalle} \\
\midrule
Formación académica & Médico Veterinario Zootecnista titulado con cédula profesional. Especialidad o Maestría en Epidemiología o Sanidad Animal deseable \\
Experiencia mínima & 5 años en campañas sanitarias oficiales, 3 años en tuberculosis bovina o enfermedades de control oficial \\
Conocimientos indispensables & Tuberculosis bovina, GBG, protocolos APHIS-USDA, SINIIGA, trazabilidad animal, normatividad SENASICA, T-MEC, diagnóstico laboratorial \\
Certificaciones requeridas & Certificación SENASICA en campañas sanitarias, conocimiento de normatividad OIE \\
Idiomas & Inglés técnico intermedio (protocolos APHIS-USDA) \\
Habilidades técnicas & Epidemiología de campo, muestreo, diagnóstico, manejo de bases de datos, SIG básico \\
Disponibilidad & 70\% trabajo de campo, disponibilidad para guardias sanitarias emergentes \\
\bottomrule
\end{tabular}
\end{table}

\subsection{Competencias Técnicas Específicas}

\textbf{Competencias obligatorias:}
\begin{itemize}
    \item Diagnóstico y control de tuberculosis bovina
    \item Manejo de Gusano Barrenador del Ganado
    \item Interpretación de protocolos sanitarios APHIS-USDA
    \item Operación de sistemas de trazabilidad (SINIIGA)
    \item Epidemiología veterinaria aplicada
    \item Técnicas de muestreo y diagnóstico de campo
\end{itemize}

\textbf{Competencias deseables:}
\begin{itemize}
    \item Experiencia en certificación para exportación
    \item Conocimiento de normatividad internacional (OIE)
    \item Manejo de sistemas de información geográfica
    \item Coordinación de programas binacionales México-EE.UU.
\end{itemize}

\section{Indicadores de Desempeño}

\begin{table}[H]
\centering
\caption{Métricas de Desempeño del MVZ Sanidad Animal}
\label{tab:kpis}
\begin{tabular}{p{6cm}p{3cm}p{3cm}}
\toprule
\rowcolor{sadergreen!20}
\textbf{Indicador} & \textbf{Meta} & \textbf{Frecuencia} \\
\midrule
UPP certificadas TBC-libre & 100\% participantes & Anual \\
Cobertura vigilancia GBG & 106 municipios & Mensual \\
Muestras procesadas GBG & $\geq$95\% en 48h & Semanal \\
Certificados sanitarios T-MEC & 100\% solicitados & Mensual \\
Actualización SINIIGA & 100\% UPP & Semanal \\
Capacitaciones impartidas & 2 eventos/mes & Mensual \\
Tiempo respuesta emergencias & $\leq$24 horas & Por evento \\
\bottomrule
\end{tabular}
\end{table}

\section{Condiciones Laborales}

\subsection{Condiciones Generales}

\begin{itemize}
    \item \textbf{Base:} Mérida, Yucatán (oficinas SENASICA/SADER)
    \item \textbf{Viajes:} 70\% del tiempo en campo (106 municipios)
    \item \textbf{Disponibilidad:} Guardias sanitarias 24/7 para emergencias
    \item \textbf{Modalidad de contratación:} Técnico especializado financiado vía FOFAY
    \item \textbf{Salario anual:} \$420,000 MXN + prestaciones
    \item \textbf{Duración del contrato:} 2026-2030 con evaluación anual
    \item \textbf{Horario:} Lunes a sábado + guardias sanitarias
\end{itemize}

\subsection{Prestaciones y Beneficios}

\begin{itemize}
    \item Prestaciones de ley federal
    \item Seguro de gastos médicos mayores
    \item Vehículo oficial especializado para trabajo de campo
    \item Viáticos para supervisión municipal
    \item Capacitación técnica especializada nacional e internacional
    \item Equipo de protección personal y material diagnóstico
\end{itemize}

\section{Marco Legal y Técnico}

\subsection{Normatividad Aplicable}

\begin{itemize}
    \item \textbf{Ley de Sanidad Animal} y su Reglamento
    \item \textbf{NOM-031-ZOO-1995:} Campaña Nacional contra la Tuberculosis Bovina
    \item \textbf{NOM-032-ZOO-1995:} Campaña Nacional contra el Gusano Barrenador
    \item \textbf{Protocolos APHIS-USDA:} Certificación sanitaria para exportación
    \item \textbf{Acuerdos T-MEC:} Capítulo sanitario y fitosanitario
    \item \textbf{Código Sanitario OIE:} Estándares internacionales
\end{itemize}

\section{Coordinación Institucional}

El MVZ Sanidad Animal coordinará técnicamente con:

\begin{itemize}
    \item \textbf{SENASICA Central:} Reportes epidemiológicos y protocolos nacionales
    \item \textbf{APHIS-USDA:} Validación de certificados y protocolos binacionales
    \item \textbf{Laboratorios oficiales:} Diagnóstico confirmatorio TBC y GBG
    \item \textbf{Uniones Ganaderas:} Operación de campañas a nivel regional
    \item \textbf{MVZ autorizados:} Red de profesionales certificados
    \item \textbf{Planta Mosca Estéril:} Coordinación técnica operativa
\end{itemize}

\section{Conclusiones Técnicas}

El perfil "MVZ Especialista en Sanidad Animal" constituye una posición técnica crítica para el éxito del Macroproyecto, responsable de dos componentes que condicionan directamente el acceso a mercados internacionales y la competitividad del sector ganadero yucateco.

La alta especialización requerida y la coordinación binacional hacen de esta posición una de las más complejas técnicamente dentro del equipo, requiriendo competencias únicas en sanidad animal, protocolos internacionales y gestión de sistemas de trazabilidad.

El éxito en este puesto garantiza la certificación sanitaria necesaria para posicionar a Yucatán como plataforma agroexportadora del sureste mexicano con acceso preferencial a mercados premium bajo T-MEC.

\vspace{2cm}

\noindent\textbf{Jefe de Programa de Producción Pecuaria Sustentable}

\vspace{1cm}

\noindent\textbf{Subdelegado Agropecuario - SADER Yucatán}

\vspace{2cm}

\noindent\textit{Mérida, Yucatán, a 28 de noviembre de 2025}

\end{document}