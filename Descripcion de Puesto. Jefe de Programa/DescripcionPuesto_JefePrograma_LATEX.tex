\documentclass[12pt,letterpaper]{article}
\usepackage[utf8]{inputenc}
\usepackage[spanish]{babel}
\usepackage{geometry}
\usepackage{graphicx}
\usepackage{fancyhdr}
\usepackage{setspace}
\usepackage{lastpage}
\usepackage{parskip}
\usepackage{booktabs}
\usepackage{array}
\usepackage{multirow}
\usepackage{longtable}
\usepackage{float}
\usepackage{xcolor}
\usepackage{colortbl}
\usepackage{amsmath}
\usepackage{ragged2e}

% Colores SADER
\definecolor{saderblue}{RGB}{0,51,102}
\definecolor{sadergreen}{RGB}{34,139,34}
\definecolor{sadergray}{RGB}{128,128,128}
\definecolor{sadergold}{RGB}{255,215,0}

% Márgenes exactos SADER
\geometry{top=2.5cm,bottom=2.5cm,left=3cm,right=3cm,headheight=20pt}

% Encabezado y pie de página
\pagestyle{fancy}
\fancyhf{}
\rfoot{\thepage}
\renewcommand{\headrulewidth}{0pt}
\fancyhead[L]{\includegraphics[width=2.8cm]{logo_sader.png}}

\begin{document}

% ========================================
% PORTADA OFICIAL
% ========================================
\begin{titlepage}
    \centering
    \vspace*{0.3cm}
    \includegraphics[width=0.25\textwidth]{logo_sader.png}\\[0.8cm]
    
    \vspace{0.4cm}
    {\normalsize\bfseries Perfil Profesional y Descripción Técnica:\par}
    \vspace{0.6cm}
    
    {\LARGE\bfseries JEFE DE PROGRAMA DE\par}
    {\LARGE\bfseries DESARROLLO PECUARIO SUSTENTABLE\par}
    \vspace{0.5cm}
    {\Large Apoyo Técnico al Macroproyecto Renacimiento Ganadero Maya:\par}
    \vspace{0.3cm}
    {\normalsize • Sistemas Silvopastoriles Intensivos\par}
    {\normalsize • Escuelas de Campo Adaptativas\par}
    {\normalsize • Biofábricas Prediales con Microorganismos Benéficos\par}
    {\normalsize • Especies Forrajeras Nativas Validadas\par}
    {\normalsize • Bioeconomía Circular Ganadera\par}
    \vspace{0.5cm}
    {\Large Yucatán 2026-2030\par}
    \vspace{0.3cm}
    {\normalsize\textit{Puesto de Nueva Creación por Coyuntura Estratégica}\par}
    
    \vfill
    
    {\normalsize Mérida, Yucatán, 27 de noviembre de 2025\par}
    \vspace{0.2cm}
    {\normalsize SADER REPRESENTACIÓN ESTATAL YUCATÁN\par}
    {\normalsize Subdelegación Agropecuaria\par}
    \vspace{0.2cm}
    {\normalsize Código: JPPPS-YUC-001\par}
    {\normalsize Secretaría de Agricultura y Desarrollo Rural (SADER)\par}
\end{titlepage}

% ========================================
% ÍNDICE AUTOMÁTICO
% ========================================
\clearpage
\thispagestyle{empty}
\vspace*{3cm}
{\large\bfseries Contenido}\\[2cm]

\tableofcontents

\clearpage
\setcounter{page}{3}

% ========================================
% CONTENIDO
% ========================================

\section{Justificación de Creación del Puesto}

\justifying

\subsection{Contexto de Oportunidad Estratégica}

La creación del puesto "Jefe de Programa de Desarrollo Pecuario Sustentable" responde a una coyuntura única caracterizada por:

\begin{itemize}
    \item \textbf{Oportunidad federal excepcional:} Disponibilidad de recursos extraordinarios para el sector ganadero a través del Macroproyecto "Renacimiento Ganadero Maya"
    \item \textbf{Necesidad operativa emergente:} El Subdelegado Agropecuario actual requiere apoyo técnico especializado para la coordinación de los seis componentes integrados del macroproyecto
    \item \textbf{Complejidad técnica del programa:} La implementación simultánea de sistemas silvopastoriles, repoblamiento ganadero, campañas sanitarias y certificación digital requiere un perfil técnico dedicado exclusivamente a estas actividades
    \item \textbf{Demanda de coordinación territorial:} La necesidad de supervisar actividades en las 4 regiones ganaderas prioritarias del estado con un equipo técnico especializado de 6 profesionales
\end{itemize}

\subsection{Propuesta de Valor del Puesto}

Este puesto permitirá al Subdelegado Agropecuario:
\begin{itemize}
    \item Mantener su función directiva y de representación institucional sin verse sobrecargado por actividades técnicas operativas
    \item Contar con un enlace técnico especializado que coordine la implementación de campo
    \item Disponer de reportes técnicos detallados para la toma de decisiones estratégicas
    \item Optimizar la ejecución del macroproyecto mediante delegación efectiva de funciones técnicas
\end{itemize}

\section{Caracterización Institucional del Puesto}

\subsection{Identificación Administrativa}

\begin{table}[H]
\centering
\caption{Datos Generales del Puesto Especializado}
\label{tab:datos_generales}
\begin{tabular}{p{5cm}p{8cm}}
\toprule
\rowcolor{sadergreen!20}
\textbf{Campo} & \textbf{Especificación Técnica} \\
\midrule
Denominación oficial & Jefe de Programa de Desarrollo Pecuario Sustentable \\
Código de identificación & JPPPS-YUC-001 \\
Dependencia jerárquica & Subdelegación Agropecuaria -- Representación Estatal SADER Yucatán \\
Línea de reporte directo & Subdelegado Agropecuario de la Representación Estatal SADER Yucatán \\
Ámbito territorial & Estado de Yucatán (106 municipios, 39,524 km²) \\
Categoría de plaza & Nivel N11 tabulador SADER o por honorarios especializados equivalente \\
Equipo técnico asignado & 6 profesionales especializados: 1 Jefe de Programa (\$35.4K mensual), 2 técnicos zootecnistas (\$24.9K c/u), 2 técnicos agrónomos (\$24.9K c/u), 1 MVZ sanidad animal (\$24.9K), distribuidos estratégicamente en 4 regiones ganaderas del estado \\
\bottomrule
\end{tabular}
\end{table}

\subsection{Contexto Programático}

El puesto se enmarca dentro del Macroproyecto Estratégico Concurrente "Renacimiento Ganadero Maya" 2026-2030, constituyendo una posición técnica especializada de nueva creación para apoyar al Subdelegado Agropecuario actual en la ejecución del Macroproyecto "Renacimiento Ganadero Maya". El puesto surge como respuesta a la coyuntura estratégica y oportunidad federal para el desarrollo ganadero sustentable, requiriendo coordinación técnica especializada de un equipo de 6 profesionales con gastos de operación de \$2.02 millones anuales distribuidos en las 4 regiones ganaderas prioritarias del estado.

% ========================================
% MARCO CONCEPTUAL
% ========================================
\section{Marco Conceptual y Objetivo Estratégico}

\subsection{Fundamentación Teórica del Puesto}

El diseño del puesto "Jefe de Programa de Desarrollo Pecuario Sustentable" se fundamenta en la ejecución técnica especializada de programas de desarrollo rural, bajo la supervisión directa del Subdelegado Agropecuario, enfocándose en la implementación de campo y asistencia técnica directa.

La posición requiere la aplicación de competencias técnicas en:

\begin{enumerate}
    \item Implementación de sistemas productivos pecuarios sustentables
    \item Ejecución de protocolos sanitarios de campo
    \item Asistencia técnica directa a productores
    \item Supervisión de equipo técnico especializado
    \item Seguimiento y reporte de actividades técnicas
\end{enumerate}

\subsection{Objetivo Estratégico del Puesto}

\textbf{Objetivo general:} Coadyuvar con el Subdelegado Agropecuario en la ejecución técnica y operativa del componente federal del Macroproyecto "Renacimiento Ganadero Maya" 2026-2030, apoyando en la implementación de los componentes estratégicos que resuelven simultáneamente:

\begin{itemize}
    \item El despoblamiento acelerado del hato bovino (-35\% 2017-2025)
    \item Las barreras sanitarias T-MEC por tuberculosis bovina
    \item El rebrote epidémico de Gusano Barrenador del Ganado
    \item La baja productividad en ganadería lechera tropical
    \item La subutilización del Centro Genético de Tizimín
    \item Las limitaciones para acceso a mercados de exportación premium por falta de certificación digital y trazabilidad APHIS-USDA
\end{itemize}

Todo bajo un paradigma de sistemas silvopastoriles intensivos y en plena alineación con la Directriz 4.1.1 del Plan Estatal de Desarrollo Renacimiento Maya 2024-2030.

\subsection{Indicadores de Desempeño del Puesto}

\begin{table}[H]
\centering
\caption{Métricas Clave de Desempeño (KPIs) del Puesto}
\label{tab:kpis}
\begin{tabular}{p{6cm}p{3cm}p{3cm}}
\toprule
\rowcolor{sadergreen!20}
\textbf{Indicador} & \textbf{Meta 2030} & \textbf{Medición} \\
\midrule
Porcentaje de ejecución presupuestal federal & $\geq$95\% & Anual \\
UPP beneficiadas directamente & 1,075 & Quinquenal \\
Hectáreas convertidas a SSPi & 6,000 ha & Quinquenal \\
Vaquillas F1 distribuidas & 12,000 cabezas & Quinquenal \\
Unidades de producción certificadas TB & 100\% & Anual \\
Productores atendidos/mes por técnico & 11-14 UPP & Mensual \\
Eventos de capacitación & 2 talleres/mes/técnico & Mensual \\
Nivel de satisfacción del equipo técnico & $\geq$85\% & Semestral \\
\bottomrule
\end{tabular}
\end{table}

% ========================================
% RESPONSABILIDADES PRINCIPALES
% ========================================
\section{Responsabilidades Principales}

El titular del puesto será responsable directo de:

\subsection{Coordinación Técnica Integral}

\textbf{1. Apoyar al Subdelegado Agropecuario} en la ejecución técnica y operativa del Macroproyecto "Renacimiento Ganadero Maya", enfocándose en la implementación de campo de los seis componentes integrados:

\begin{itemize}
    \item \textbf{Componente 1:} Transformación a Sistemas Silvopastoriles Intensivos (6,000 hectáreas con \textit{Leucaena leucocephala})
    \item \textbf{Componente 2:} Repoblamiento Ganadero Estratégico (12,000 vaquillas F1 certificadas)
    \item \textbf{Componente 3:} Refundación del Centro de Mejoramiento Genético de Tizimín (certificación OIE/ISO-17025)
    \item \textbf{Componente 4:} Desarrollo Lechero Tropical (+40\% producción mediante 75 módulos tecnificados)
    \item \textbf{Componente 5:} Planta de Mosca Estéril (250M moscas/semana para erradicación GBG)
    \item \textbf{Componente 6:} Certificación Digital TBC + Plataforma CESO-APHIS (habilitación exportaciones T-MEC)
\end{itemize}

\subsection{Gestión de Equipos y Recursos}

\textbf{2. Dirigir y supervisar} al equipo técnico de 6 personas asignado al Programa, garantizando la asistencia técnica, capacitación (mínimo 80 jornadas/año) y seguimiento de indicadores en campo:

\begin{itemize}
    \item \textbf{2 Técnicos Zootecnistas (Nivel O21 - \$24.9K c/u):} Especialistas en sistemas silvopastoriles intensivos, manejo de \textit{Leucaena leucocephala} y pastoreo racional
    \item \textbf{2 Técnicos Agrónomos (Nivel O21 - \$24.9K c/u):} Especialistas en especies forrajeras nativas, suelos tropicales y transferencia tecnológica
    \item \textbf{1 MVZ Sanidad Animal (Nivel O21 - \$24.9K):} Coordinador de campañas TBC y GBG, protocolos SENASICA, trazabilidad SINIIGA
\end{itemize}

\textbf{3. Elaborar los anteproyectos} de presupuesto anual y administrar los recursos asignados, incluyendo:
   \begin{itemize}
       \item Formulación de presupuestos por componente técnico
       \item Gestión de gastos de operación del equipo técnico (\$2.02 millones anuales)
       \item Supervisión de la correcta aplicación de los recursos federales
       \item Coadyuvar con el Subdelegado Agropecuario en la gestión presupuestal
       \item Rendición de cuentas ante las instancias correspondientes
   \end{itemize}

\subsection{Coordinación Interinstitucional}

\textbf{4. Ejecutar acciones técnicas de campo} con técnicos de SENASICA, INIFAP, UADY-FMVZ, asociaciones ganaderas locales y productores, bajo la supervisión y coordinación del Subdelegado Agropecuario.

\textbf{5. Rendir informes técnicos} semanales al Subdelegado Agropecuario sobre avances de campo, elaborar reportes mensuales de actividades del equipo técnico y participar en las reuniones técnicas que sean convocadas por el Subdelegado.

% ========================================
% PERFIL DEL PUESTO
% ========================================
\section{Perfil del Puesto}

\subsection{Requisitos Académicos y Experiencia}

\begin{table}[H]
\centering
\caption{Requisitos del Puesto}
\label{tab:requisitos}
\begin{tabular}{p{4cm}p{9cm}}
\toprule
\rowcolor{sadergreen!20}
\textbf{Requisito} & \textbf{Detalle} \\
\midrule
Formación académica & Médico Veterinario Zootecnista o Ingeniero Agrónomo con especialidad pecuaria. Maestría deseable \\
Experiencia mínima & 8 años en dirección/coordinación de programas pecuarios, 5 años en sistemas silvopastoriles y/o sanidad animal \\
Conocimientos indispensables & Sistemas silvopastoriles intensivos, sanidad animal (TB bovina y GBG), trazabilidad SINIIGA, T-MEC, OIE, mejora genética tropical, manejo de proyectos $>$\$200 MDP \\
Idiomas & Inglés técnico avanzado (interacción APHIS-USDA) \\
Habilidades & Liderazgo técnico, planeación estratégica, gestión de equipos multidisciplinarios, uso de SIG, elaboración de proyectos, negociación con productores \\
Disponibilidad & Para viajar 60\% del tiempo al interior del estado \\
\bottomrule
\end{tabular}
\end{table}

\subsection{Competencias Técnicas Específicas}

\textbf{Competencias obligatorias:}
\begin{itemize}
    \item Diseño y evaluación de sistemas silvopastoriles intensivos
    \item Protocolos sanitarios binacionales (México-EE.UU.)
    \item Trazabilidad y certificación ganadera
    \item Gestión de presupuestos federales complejos
    \item Coordinación interinstitucional multi-nivel
\end{itemize}

\textbf{Competencias deseables:}
\begin{itemize}
    \item Certificación en metodologías de captura de carbono
    \item Experiencia en programas de cooperación internacional
    \item Conocimiento de normatividad OIE e ISO-17025
    \item Manejo de sistemas de información geográfica
\end{itemize}

% ========================================
% CONDICIONES LABORALES
% ========================================
\section{Condiciones Laborales}

\subsection{Condiciones Generales}

\begin{itemize}
    \item \textbf{Base:} Mérida, Yucatán (oficinas FOFAY/SADER)
    \item \textbf{Viajes:} Frecuentes al interior del estado (vehículo oficial y viáticos con cargo a gastos de operación)
    \item \textbf{Disponibilidad:} Para guardias sanitarias emergentes (GBG/TB) y supervisión de equipo técnico
    \item \textbf{Modalidad de contratación:} Plaza temporal especializada financiada vía FOFAY (gastos de operación convenidos)
    \item \textbf{Sueldo bruto mensual:} \$35,448 MXN + prestaciones (\$425,376 MXN anuales - Nivel N11 tabulador SADER)
    \item \textbf{Duración del contrato:} 2026-2030 con posibilidad de extensión según evaluación de resultados
    \item \textbf{Horario:} Lunes a viernes + guardias eventuales + supervisión de campo
\end{itemize}

\subsection{Prestaciones y Beneficios}

\begin{itemize}
    \item Prestaciones de ley federal
    \item Seguro de gastos médicos mayores institucional
    \item Seguro de vida
    \item Vehículo oficial para comisiones
    \item Viáticos para trabajo de campo
    \item Capacitación técnica especializada
\end{itemize}

% ========================================
% MARCO LEGAL Y FINANCIERO
% ========================================
\section{Marco Legal y Financiero del Puesto}

\subsection{Fundamento Jurídico del Financiamiento}

El presente puesto se crea bajo el marco jurídico del convenio de colaboración Federación-Estado para la ejecución del Macroproyecto "Renacimiento Ganadero Maya", sustentado en:

\begin{itemize}
    \item \textbf{Precedente "Alianza para el Campo":} Metodología probada de coinversión tripartita con gastos de operación convenidos
    \item \textbf{Fideicomiso FOFAY:} Instrumento financiero establecido para la canalización y administración de recursos agropecuarios estatales
    \item \textbf{Convenio de colaboración SADER-Gobierno de Yucatán:} Marco legal para la ejecución compartida de programas federales
    \item \textbf{Normatividad de gastos de operación:} 8-12\% del presupuesto total autorizado para estructura técnica y operativa
\end{itemize}

\subsection{Estructura de Rendición de Cuentas}

El titular del puesto será responsable ante:

\begin{enumerate}
    \item \textbf{Comité Técnico del FOFAY:} Informes trimestrales de avance físico-financiero
    \item \textbf{Subdelegación Agropecuaria SADER:} Reportes mensuales de ejecución técnica
    \item \textbf{Órganos de control federal y estatal:} Auditorías y verificaciones de la aplicación de recursos
    \item \textbf{Comité Técnico Estatal del Macroproyecto:} Evaluación de resultados e impactos
\end{enumerate}

% ========================================
% ESTRUCTURA DEL EQUIPO TÉCNICO
% ========================================
\section{Estructura del Equipo Técnico}

\subsection{Marco Organizacional del Equipo}

El equipo técnico está conformado por 6 profesionales especializados (incluyendo al Jefe de Programa), organizados para brindar asistencia técnica directa en campo bajo la supervisión del Subdelegado Agropecuario y la coordinación operativa del Jefe de Programa.

\subsection{Perfiles Técnicos del Equipo}

Los siguientes perfiles técnicos operan bajo la supervisión del Jefe de Programa:

\begin{enumerate}
    \item \textbf{Jefe de Programa (N11 - \$425,376 MXN anuales)} - Supervisión técnica del equipo y enlace con Subdelegado
    \item \textbf{Técnico Zootecnista A (O21 - \$298,740 MXN anuales)} - Sistemas silvopastoriles intensivos y pastoreo
    \item \textbf{Técnico Zootecnista B (O21 - \$298,740 MXN anuales)} - Biofábricas prediales y especies forrajeras
    \item \textbf{Técnico Agrónomo A (O21 - \$298,740 MXN anuales)} - Transferencia tecnológica y capacitación
    \item \textbf{Técnico Agrónomo B (O21 - \$298,740 MXN anuales)} - Suelos tropicales y sistemas de riego
    \item \textbf{MVZ Sanidad Animal (O21 - \$298,740 MXN anuales)} - Campañas sanitarias TBC/GBG
\end{enumerate}

\subsection{Competencias Transversales del Equipo}

Todos los miembros del equipo técnico deben cumplir con competencias mínimas transversales:
\begin{itemize}
    \item Conocimiento del contexto ganadero yucateco
    \item Manejo de sistemas de información (SINIIGA, plataformas digitales)
    \item Habilidades de comunicación con productores rurales
    \item Disponibilidad para trabajo de campo (60\% del tiempo)
    \item Capacidad de trabajar bajo presión y metas cuantificables
    \item Experiencia mínima de 5 años en desarrollo rural o sectores afines
\end{itemize}

% ========================================
% CONCLUSIONES
% ========================================
\section{Plan de Trabajo del Jefe de Programa}

\subsection{Cronograma de Implementación 2026-2030}

\textbf{Año 1 (2026): Establecimiento y Arranque}
\begin{itemize}
    \item \textbf{Ene-Mar:} Conformación y capacitación del equipo técnico (6 profesionales)
    \item \textbf{Abr-Jun:} Establecimiento de sistemas de información y protocolos de campo
    \item \textbf{Jul-Sep:} Inicio de conversión de primeras 1,200 hectáreas a SSPi
    \item \textbf{Oct-Dic:} Implementación de 20 biofábricas piloto y capacitación de 250 productores
\end{itemize}

\textbf{Años 2-4 (2027-2029): Escalamiento}
\begin{itemize}
    \item \textbf{2027:} 2,400 ha SSPi + 3,600 vaquillas F1 + 300 UPP atendidas
    \item \textbf{2028:} 4,200 ha SSPi + 7,200 vaquillas F1 + 650 UPP atendidas  
    \item \textbf{2029:} 5,700 ha SSPi + 10,800 vaquillas F1 + 950 UPP atendidas
\end{itemize}

\textbf{Año 5 (2030): Consolidación}
\begin{itemize}
    \item \textbf{Ene-Jun:} Finalización de conversión (6,000 ha SSPi total)
    \item \textbf{Jul-Sep:} Certificación final TBC del 100\% del hato
    \item \textbf{Oct-Dic:} Evaluación integral y sistematización de experiencias
\end{itemize}

\subsection{Rutina de Trabajo Mensual}

\textbf{Semana 1: Planificación y Coordinación}
\begin{itemize}
    \item Reunión semanal con Subdelegado Agropecuario (lunes)
    \item Planificación de actividades técnicas con el equipo (martes)
    \item Revisión de avances y ajustes de rutas (miércoles-viernes)
\end{itemize}

\textbf{Semana 2-3: Ejecución de Campo}
\begin{itemize}
    \item Supervisión directa de actividades técnicas en las 4 regiones
    \item Asistencia técnica directa a productores prioritarios
    \item Seguimiento de implementación de SSPi y biofábricas
\end{itemize}

\textbf{Semana 4: Evaluación y Reporte}
\begin{itemize}
    \item Consolidación de reportes técnicos del equipo
    \item Elaboración de informe mensual al Subdelegado
    \item Evaluación de desempeño del equipo técnico
\end{itemize}

\subsection{Indicadores de Desempeño Mensual}

\begin{table}[H]
\centering
\caption{Metas Mensuales del Jefe de Programa}
\label{tab:metas_mensuales}
\begin{tabular}{p{6cm}p{2.5cm}p{2.5cm}}
\toprule
\rowcolor{sadergreen!20}
\textbf{Indicador Operativo} & \textbf{Meta/Mes} & \textbf{Acumulado Anual} \\
\midrule
Hectáreas convertidas a SSPi & 100 ha & 1,200 ha \\
Productores capacitados & 20-25 & 300 \\
Vaquillas F1 distribuidas & 200 & 2,400 \\
Biofábricas establecidas & 3-4 & 40 \\
Visitas técnicas de campo & 80-100 & 1,200 \\
Reportes técnicos entregados & 1 & 12 \\
Reuniones con Subdelegado & 4-5 & 60 \\
\bottomrule
\end{tabular}
\end{table}

\section{Conclusiones Técnicas}

El perfil profesional "Jefe de Programa de Desarrollo Pecuario Sustentable" constituye una posición técnica especializada de nueva creación, diseñada para apoyar operativamente al Subdelegado Agropecuario en la ejecución del Macroproyecto "Renacimiento Ganadero Maya" 2026-2030.

La complejidad multidisciplinaria del cargo requiere competencias únicas en gestión de sistemas adaptativos complejos, coordinación interinstitucional y liderazgo técnico de un equipo multidisciplinario de 6 profesionales especializados (incluido el Jefe de Programa).

El éxito del macroproyecto depende críticamente tanto de la selección de un Jefe de Programa que reúna las competencias técnicas y gerenciales necesarias, como del reclutamiento y desarrollo de un equipo técnico altamente especializado capaz de ejecutar efectivamente los seis componentes estratégicos integrados con metas físicas cuantificables.

La estructura organizacional propuesta garantiza la especialización técnica requerida mientras mantiene la coordinación integral necesaria para un proyecto de \$926.5 millones MXN distribuido en 106 municipios.

% ========================================
% VALIDACIÓN
% ========================================
\section{Validación Institucional}

El presente perfil profesional ha sido revisado y validado por las autoridades competentes de la Representación Estatal SADER Yucatán, en cumplimiento de los lineamientos de recursos humanos especializados para programas estratégicos federales.

\vspace{2cm}

\noindent\textbf{Titular de la Representación Estatal SADER Yucatán}

\vspace{1cm}

\noindent\textbf{Subdelegado Agropecuario - Representación Estatal SADER Yucatán}

\vspace{1cm}

\noindent\textbf{Director de Recursos Humanos - SADER Yucatán}

\vspace{2cm}

\noindent\textit{Mérida, Yucatán, a 19 de noviembre de 2025}

\end{document}