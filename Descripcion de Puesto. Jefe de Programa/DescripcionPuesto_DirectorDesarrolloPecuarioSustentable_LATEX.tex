\documentclass[12pt,letterpaper]{article}
\usepackage[utf8]{inputenc}
\usepackage[spanish]{babel}
\usepackage{geometry}
\usepackage{graphicx}
\usepackage{fancyhdr}
\usepackage{setspace}
\usepackage{lastpage}
\usepackage{parskip}
\usepackage{booktabs}
\usepackage{array}
\usepackage{multirow}
\usepackage{longtable}
\usepackage{float}
\usepackage{xcolor}
\usepackage{colortbl}
\usepackage{amsmath}
\usepackage{ragged2e}

% Colores SADER
\definecolor{saderblue}{RGB}{0,51,102}
\definecolor{sadergreen}{RGB}{34,139,34}
\definecolor{sadergray}{RGB}{128,128,128}
\definecolor{sadergold}{RGB}{255,215,0}

% Márgenes exactos SADER
\geometry{top=2.5cm,bottom=2.5cm,left=3cm,right=3cm,headheight=20pt}

% Encabezado y pie de página
\pagestyle{fancy}
\fancyhf{}
\fancyhead[C]{\textcolor{sadergray}{\footnotesize DESCRIPCIÓN DE PUESTO - DIRECTOR DE DESARROLLO PECUARIO SUSTENTABLE}}
\fancyfoot[C]{\textcolor{sadergray}{\small Página \thepage\ de \pageref{LastPage}}}
\renewcommand{\headrulewidth}{0.4pt}
\renewcommand{\footrulewidth}{0pt}

\begin{document}

% ========================================
% PORTADA OFICIAL
% ========================================
\begin{titlepage}
    \centering
    \vspace*{1.5cm}
    
    \vspace{0.4cm}
    {\normalsize\bfseries Perfil Profesional y Descripción Técnica:\par}
    \vspace{0.6cm}
    
    {\LARGE\bfseries DIRECTOR DE\par}
    {\LARGE\bfseries DESARROLLO PECUARIO SUSTENTABLE\par}
    \vspace{0.5cm}
    {\Large Apoyo Técnico al Macroproyecto Renacimiento Ganadero Maya:\par}
    \vspace{0.3cm}
    {\normalsize • Sistemas Silvopastoriles Intensivos\par}
    {\normalsize • Repoblamiento Ganadero Estratégico\par}
    {\normalsize • Centro de Mejoramiento Genético\par}
    {\normalsize • Desarrollo Lechero Tropical\par}
    {\normalsize • Meliponicultura Sustentable Maya\par}
    {\normalsize • Plataforma Digital de Seguimiento Sanitario\par}
    \vspace{0.5cm}
    {\Large Yucatán 2026-2030\par}
    \vspace{0.3cm}
    {\normalsize\textit{Puesto de Nueva Creación por Coyuntura Estratégica}\par}
    
    \vfill
    
    {\normalsize Mérida, Yucatán, 11 de diciembre de 2025\par}
    \vspace{0.2cm}
    {\normalsize GOBIERNO DEL ESTADO DE YUCATÁN\par}
    {\normalsize Secretaría de Desarrollo Rural (SEDER)\par}
    \vspace{0.2cm}
    {\normalsize Código: DDP-SEDER-YUC-001\par}
    {\normalsize Consultor Independiente/Comisionado por SEDER Yucatán\par}
\end{titlepage}

% ========================================
% ÍNDICE AUTOMÁTICO
% ========================================
\clearpage
\thispagestyle{empty}
\vspace*{3cm}
{\large\bfseries Contenido}\\[2cm]

\tableofcontents

\clearpage
\setcounter{page}{3}

% ========================================
% CONTENIDO
% ========================================

\section{Justificación de Creación del Puesto}

\justifying

\subsection{Contexto de Oportunidad Estratégica}

La creación de la \textit{Dirección de Desarrollo Pecuario Sustentable} responde a una coyuntura estratégica excepcional caracterizada por:

\begin{itemize}
    \item \textbf{Nueva Ley Ganadera Yucatán 2025:} El Decreto 129/2025 (publicado 28 nov 2025) establece 69 coincidencias directas con el proyecto, especialmente en trazabilidad (53 menciones) y sanidad animal (47 menciones)
    \item \textbf{Oportunidad federal excepcional:} Disponibilidad de recursos extraordinarios para el sector ganadero a través del Macroproyecto \textit{Renacimiento Ganadero Maya}
    \item \textbf{Necesidad directiva emergente:} La nueva ley menciona 16 veces "dirección" como nivel organizacional, validando la escala directiva requerida
    \item \textbf{Complejidad técnica del programa:} La implementación simultánea de seis componentes estratégicos requiere un perfil técnico dedicado exclusivamente a estas actividades
    \item \textbf{Demanda de coordinación territorial optimizada:} La necesidad de supervisar actividades en las 4 regiones ganaderas prioritarias del estado con un equipo técnico especializado
\end{itemize}

\subsection{Propuesta de Valor del Puesto}

Este puesto permitirá al Secretario de Desarrollo Rural:
\begin{itemize}
    \item Mantener su función directiva y de representación institucional sin verse sobrecargado por actividades técnicas operativas
    \item Contar con un enlace técnico especializado que coordine la implementación de campo
    \item Disponer de reportes técnicos detallados para la toma de decisiones estratégicas
    \item Optimizar la ejecución del macroproyecto mediante delegación efectiva de funciones técnicas
\end{itemize}

\section{Caracterización Institucional del Puesto}

\subsection{Identificación Administrativa}

\begin{table}[H]
\centering
\caption{Datos Generales del Puesto Especializado}
\label{tab:datos_generales}
\begin{tabular}{p{5cm}p{8cm}}
\toprule
\rowcolor{sadergreen!20}
\textbf{Campo} & \textbf{Especificación Técnica} \\
\midrule
Denominación oficial & Director de Desarrollo Pecuario Sustentable \\
Código de identificación & DDP-SEDER-YUC-001 \\
Dependencia jerárquica & Secretaría de Desarrollo Rural (SEDER) -- Gobierno del Estado de Yucatán \\
Línea de reporte directo & Secretario de Desarrollo Rural del Estado de Yucatán \\
Ámbito territorial & Estado de Yucatán (106 municipios, 39,524 km²) \\
Categoría de plaza & Nivel directivo N14 SEDER con estructura de dirección \\
Estructura directiva asignada & Director (\$52.0K mensual) + 3 Subdirectores especializados (\$38.5K c/u) + 5 coordinadores técnicos regionales (\$28.5K c/u) \\
\bottomrule
\end{tabular}
\end{table}

\subsection{Contexto Programático}

La Dirección se enmarca dentro del Macroproyecto Estratégico Concurrente \textit{Renacimiento Ganadero Maya} 2026-2030, constituyendo una unidad directiva especializada habilitada por la Nueva Ley Ganadera Yucatán 2025 (Decreto 129/2025) para liderar la ejecución del macroproyecto de \$818.0 millones MXN con esquema de financiamiento híbrido.

% ========================================
% MARCO CONCEPTUAL
% ========================================
\section{Marco Conceptual y Objetivo Estratégico}

\subsection{Fundamentación Teórica del Puesto}

El diseño de la \textit{Dirección de Desarrollo Pecuario Sustentable} se fundamenta en el liderazgo estratégico y la coordinación directiva de programas ganaderos estatales de gran escala, bajo mandato directo del Secretario de Desarrollo Rural.

La posición requiere la aplicación de competencias técnicas en:

\begin{enumerate}
    \item Implementación de sistemas productivos pecuarios sustentables
    \item Ejecución de protocolos sanitarios de campo
    \item Asistencia técnica directa a productores
    \item Supervisión de equipo técnico especializado
    \item Seguimiento y reporte de actividades técnicas
\end{enumerate}

\subsection{Objetivo Estratégico del Puesto}

\textbf{Objetivo general:} Dirigir y liderar la ejecución estratégica del Macroproyecto \textit{Renacimiento Ganadero Maya} 2026-2030 bajo mandato del Secretario de Desarrollo Rural, supervisando la implementación estatal de los seis componentes estratégicos que resuelven simultáneamente:

\begin{itemize}
    \item El despoblamiento acelerado del hato bovino (-35\% 2017-2025)
    \item Las barreras sanitarias T-MEC por tuberculosis bovina
    \item La baja productividad en ganadería lechera tropical
    \item La subutilización del Centro Genético de Tizimín
    \item Las limitaciones para acceso a mercados de exportación premium por falta de certificación digital y trazabilidad APHIS-USDA
\end{itemize}

Todo bajo un paradigma de sistemas silvopastoriles intensivos y en plena alineación con la Directriz 4.1.1 del Plan Estatal de Desarrollo Renacimiento Maya 2024-2030.

\subsection{Indicadores de Desempeño del Puesto}

\begin{table}[H]
\centering
\caption{Métricas Clave de Desempeño (KPIs) del Puesto}
\label{tab:kpis}
\begin{tabular}{p{6cm}p{3cm}p{3cm}}
\toprule
\rowcolor{sadergreen!20}
\textbf{Indicador} & \textbf{Meta 2030} & \textbf{Medición} \\
\midrule
Porcentaje de ejecución presupuestal federal & $\geq$95\% & Anual \\
UPP beneficiadas directamente & 1,320 & Quinquenal \\
Hectáreas convertidas a SSPi & 6,000 ha & Quinquenal \\
Vaquillas F1 distribuidas & 12,000 cabezas & Quinquenal \\
Unidades de producción certificadas TB & 100\% & Anual \\
Productores meliponícolas capacitados & 500 & Quinquenal \\
Nivel de satisfacción del equipo técnico & $\geq$85\% & Semestral \\
\bottomrule
\end{tabular}
\end{table}

% ========================================
% RESPONSABILIDADES PRINCIPALES
% ========================================
\section{Responsabilidades Principales}

El titular del puesto será responsable directo de:

\subsection{Coordinación Técnica Integral}

\textbf{1. Dirigir bajo mandato del Secretario de Desarrollo Rural} la ejecución estratégica del Macroproyecto \textit{Renacimiento Ganadero Maya}, liderando la implementación estatal de los seis componentes integrados:

\begin{itemize}
    \item \textbf{Componente 1:} Transformación a Sistemas Silvopastoriles Intensivos (6,000 hectáreas con \textit{Leucaena leucocephala})
    \item \textbf{Componente 2:} Repoblamiento Ganadero Estratégico (12,000 vaquillas F1 certificadas)
    \item \textbf{Componente 3:} Refundación del Centro de Mejoramiento Genético de Tizimín (certificación OIE/ISO-17025)
    \item \textbf{Componente 4:} Desarrollo Lechero Tropical (+40\% producción mediante 75 módulos tecnificados)
    \item \textbf{Componente 5:} Meliponicultura Sustentable Maya (500 beneficiarios: 350 mujeres, 115 jóvenes)
    \item \textbf{Componente 6:} Plataforma Digital de Seguimiento Sanitario (sistema CESO-APHIS + certificación T-MEC)
\end{itemize}

\subsection{Gestión de Equipos y Recursos}

\textbf{2. Supervisar estructura técnica optimizada} de 3 Coordinadores Técnicos especializados + flujos de coordinación institucional, garantizando eficiencia operativa y eliminación de duplicaciones:

\begin{itemize}
    \item \textbf{Coordinador Técnico SSPi Senior (Nivel O21 - \$29.9K):} Supervisa SSPi, desarrollo lechero, meliponicultura, especies forrajeras
    \item \textbf{Coordinador Técnico Genética (Nivel O21 - \$29.9K):} Supervisa Centro Tizimín, repoblamiento F1, mejoramiento genético
    \item \textbf{Coordinador Técnico Monitoreo Ambiental (Nivel O21 - \$29.9K):} Supervisa SIG, captura carbono, plataforma digital CESO
    \item \textbf{Flujos Coordinados con Dirección Sanidad SEDER:} TBC, certificación, protocolos APHIS-SENASICA (sin duplicación institucional)
\end{itemize}

\textbf{3. Elaborar los anteproyectos} de presupuesto anual y administrar los recursos asignados, incluyendo:
   \begin{itemize}
       \item Formulación de presupuestos por componente técnico
       \item Gestión de gastos de operación del equipo técnico
       \item Supervisión de la correcta aplicación de los recursos federales
       \item Coadyuvar con el Secretario de Desarrollo Rural en la gestión presupuestal estatal
       \item Rendición de cuentas ante las instancias correspondientes
   \end{itemize}

\subsection{Coordinación Interinstitucional}

\textbf{4. Ejecutar acciones técnicas de campo} con técnicos de SENASICA, INIFAP, UADY-FMVZ, asociaciones ganaderas locales y productores, bajo la supervisión y coordinación del Secretario de Desarrollo Rural.

\textbf{5. Coordinar flujos sanitarios} con la Dirección de Sanidad SEDER existente mediante:
   \begin{itemize}
       \item Reuniones semanales de coordinación sanitaria
       \item Listados mensuales de UPP para certificación TBC
       \item Cronograma trimestral coordinado de campañas
       \item Reportes unificados al Secretario SEDER
   \end{itemize}

\textbf{6. Rendir informes técnicos} semanales al Secretario de Desarrollo Rural sobre avances de campo, elaborar reportes mensuales de actividades del equipo técnico y participar en las reuniones técnicas que sean convocadas por la Secretaría.

% ========================================
% PERFIL DEL PUESTO
% ========================================
\section{Perfil del Puesto}

\subsection{Requisitos Académicos y Experiencia}

\begin{table}[H]
\centering
\caption{Requisitos del Puesto}
\label{tab:requisitos}
\begin{tabular}{p{4cm}p{9cm}}
\toprule
\rowcolor{sadergreen!20}
\textbf{Requisito} & \textbf{Detalle} \\
\midrule
Formación académica & Médico Veterinario Zootecnista o Ingeniero Agrónomo con especialidad pecuaria. Maestría OBLIGATORIA. Doctorado o especialidad directiva deseable \\
Experiencia mínima & 12 años en dirección de programas ganaderos >\$500 MDP, 8 años en liderazgo de equipos multidisciplinarios >10 personas \\
Conocimientos indispensables & Sistemas silvopastoriles intensivos, sanidad animal (TB bovina), trazabilidad SINIIGA, T-MEC, OIE, mejora genética tropical, manejo de proyectos $>$\$200 MDP \\
Idiomas & Inglés técnico avanzado (interacción APHIS-USDA) \\
Habilidades & Liderazgo técnico, planeación estratégica, gestión de equipos multidisciplinarios, uso de SIG, elaboración de proyectos, negociación con productores \\
Disponibilidad & Para viajar 60\% del tiempo al interior del estado \\
\bottomrule
\end{tabular}
\end{table}

\subsection{Competencias Técnicas Específicas}

\textbf{Competencias obligatorias:}
\begin{itemize}
    \item Diseño y evaluación de sistemas silvopastoriles intensivos
    \item Coordinación de flujos sanitarios institucionales
    \item Gestión de proyectos ganaderos integrales
    \item Gestión de presupuestos federales complejos
    \item Coordinación interinstitucional multi-nivel y eliminación de duplicaciones
\end{itemize}

\textbf{Competencias deseables:}
\begin{itemize}
    \item Certificación en metodologías de captura de carbono
    \item Experiencia en programas de cooperación internacional
    \item Conocimiento de normatividad OIE e ISO-17025
    \item Manejo de sistemas de información geográfica
    \item Experiencia en meliponicultura sustentable
\end{itemize}

% ========================================
% CONDICIONES LABORALES
% ========================================
\section{Condiciones Laborales}

\subsection{Condiciones Generales}

\begin{itemize}
    \item \textbf{Base:} Mérida, Yucatán (oficinas SEDER)
    \item \textbf{Viajes:} Frecuentes al interior del estado (vehículo oficial y viáticos con cargo a gastos de operación)
    \item \textbf{Disponibilidad:} Para guardias sanitarias emergentes (TB) y supervisión de equipo técnico
    \item \textbf{Modalidad de contratación:} Plaza de nueva creación directa SEDER con financiamiento estatal
    \item \textbf{Sueldo bruto mensual:} \$52,000 MXN + prestaciones (\$624,000 MXN anuales - Nivel N14 directivo SEDER)
    \item \textbf{Estructura optimizada:} 4 personas (Director + 3 Coordinadores) con ahorro de \$512K anuales por eliminación de duplicaciones
    \item \textbf{Duración del contrato:} 2026-2030 con posibilidad de extensión según evaluación de resultados
    \item \textbf{Horario:} Lunes a viernes + guardias eventuales + supervisión de campo
\end{itemize}

\subsection{Prestaciones y Beneficios}

\begin{itemize}
    \item Prestaciones de ley federal
    \item Seguro de gastos médicos mayores institucional
    \item Seguro de vida
    \item Vehículo oficial para comisiones
    \item Viáticos para trabajo de campo
    \item Capacitación técnica especializada
\end{itemize}

% ========================================
% MARCO LEGAL Y FINANCIERO
% ========================================
\section{Marco Legal y Financiero del Puesto}

\subsection{Fundamento Jurídico del Financiamiento}

El presente puesto se crea bajo el marco jurídico del convenio de colaboración Federación-Estado para la ejecución del Macroproyecto \textit{Renacimiento Ganadero Maya}, sustentado en:

\begin{itemize}
    \item \textbf{Marco legal estatal:} Ley Orgánica de la Administración Pública del Estado de Yucatán
    \item \textbf{Presupuesto directo SEDER:} Recursos estatales asignados para estructura técnica especializada
    \item \textbf{Convenio de colaboración Federal-Estatal:} Marco legal para la ejecución compartida coordinada por SEDER Yucatán
    \item \textbf{Normatividad de gastos de operación:} Lineamientos estatales para estructura técnica y operativa de programas estratégicos
\end{itemize}

\subsection{Estructura de Rendición de Cuentas}

El titular del puesto será responsable ante:

\begin{enumerate}
    \item \textbf{Secretaría de Desarrollo Rural (SEDER):} Reportes mensuales de ejecución técnica y avance físico-financiero
    \item \textbf{Secretaría de Administración y Finanzas del Estado:} Informes trimestrales presupuestales
    \item \textbf{Órganos de control federal y estatal:} Auditorías y verificaciones de la aplicación de recursos
    \item \textbf{Comité Técnico Estatal del Macroproyecto:} Evaluación de resultados e impactos
\end{enumerate}

% ========================================
% PLAN DE TRABAJO
% ========================================
\section{Plan de Trabajo del Director}

\subsection{Cronograma de Implementación 2026-2030}

\textbf{Año 1 (2026): Establecimiento y Arranque}
\begin{itemize}
    \item \textbf{Ene-Mar:} Conformación y capacitación del equipo directivo y técnico
    \item \textbf{Abr-Jun:} Establecimiento de sistemas de información y protocolos de campo
    \item \textbf{Jul-Sep:} Inicio de conversión de primeras 1,200 hectáreas a SSPi
    \item \textbf{Oct-Dic:} Implementación de primeros módulos lecheros y meliponícolas
\end{itemize}

\textbf{Años 2-4 (2027-2029): Escalamiento}
\begin{itemize}
    \item \textbf{2027:} 2,400 ha SSPi + 3,600 vaquillas F1 + 300 UPP atendidas
    \item \textbf{2028:} 4,200 ha SSPi + 7,200 vaquillas F1 + 650 UPP atendidas  
    \item \textbf{2029:} 5,700 ha SSPi + 10,800 vaquillas F1 + 950 UPP atendidas
\end{itemize}

\textbf{Año 5 (2030): Consolidación}
\begin{itemize}
    \item \textbf{Ene-Jun:} Finalización de conversión (6,000 ha SSPi total)
    \item \textbf{Jul-Sep:} Certificación final TBC del 100\% del hato
    \item \textbf{Oct-Dic:} Evaluación integral y sistematización de experiencias
\end{itemize}

\subsection{Indicadores de Desempeño Mensual}

\begin{table}[H]
\centering
\caption{Metas Mensuales del Director}
\label{tab:metas_mensuales}
\begin{tabular}{p{6cm}p{2.5cm}p{2.5cm}}
\toprule
\rowcolor{sadergreen!20}
\textbf{Indicador Operativo} & \textbf{Meta/Mes} & \textbf{Acumulado Anual} \\
\midrule
Hectáreas convertidas a SSPi & 100 ha & 1,200 ha \\
Productores capacitados & 25-30 & 350 \\
Vaquillas F1 distribuidas & 200 & 2,400 \\
Módulos lecheros implementados & 1-2 & 15 \\
Productores meliponícolas incorporados & 8-10 & 100 \\
Reportes técnicos entregados & 1 & 12 \\
Reuniones con Secretario SEDER & 4-5 & 60 \\
\bottomrule
\end{tabular}
\end{table}

\section{Conclusiones Técnicas}

El perfil profesional \textit{Director de Desarrollo Pecuario Sustentable} constituye una posición directiva especializada de nueva creación, diseñada para liderar la ejecución estatal del Macroproyecto \textit{Renacimiento Ganadero Maya} 2026-2030 bajo el mandato del Secretario de Desarrollo Rural.

La complejidad multidisciplinaria del cargo requiere competencias únicas en gestión de sistemas adaptativos complejos, coordinación interinstitucional y liderazgo técnico de los seis componentes estratégicos integrados, maximizando la eficiencia operativa y el impacto técnico.

El éxito del macroproyecto depende críticamente de la selección de un Director que reúna las competencias técnicas y gerenciales necesarias para ejecutar efectivamente un programa de \$818.0 millones MXN distribuido en 106 municipios con metas físicas cuantificables y verificables.

% ========================================
% VALIDACIÓN
% ========================================
\section{Validación Institucional}

El presente perfil profesional ha sido elaborado por consultor independiente comisionado por SEDER Yucatán, en cumplimiento de los lineamientos de recursos humanos especializados para programas estratégicos estatales.

\vspace{2cm}

\noindent\textbf{Secretario de Desarrollo Rural del Estado de Yucatán}

\vspace{1cm}

\noindent\textbf{Subsecretario de Desarrollo Pecuario - SEDER Yucatán}

\vspace{1cm}

\noindent\textbf{Director de Recursos Humanos - SEDER Yucatán}

\vspace{2cm}

\noindent\textit{Mérida, Yucatán, a 11 de diciembre de 2025}

\end{document}