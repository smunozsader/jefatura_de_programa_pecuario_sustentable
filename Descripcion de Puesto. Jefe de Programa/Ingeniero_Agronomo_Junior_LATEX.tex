\documentclass[12pt,letterpaper]{article}
\usepackage[utf8]{inputenc}
\usepackage[spanish]{babel}
\usepackage{geometry}
\usepackage{graphicx}
\usepackage{fancyhdr}
\usepackage{setspace}
\usepackage{lastpage}
\usepackage{parskip}
\usepackage{booktabs}
\usepackage{array}
\usepackage{multirow}
\usepackage{longtable}
\usepackage{float}
\usepackage{xcolor}
\usepackage{colortbl}
\usepackage{amsmath}
\usepackage{ragged2e}

% Colores SADER
\definecolor{saderblue}{RGB}{0,51,102}
\definecolor{sadergreen}{RGB}{34,139,34}
\definecolor{sadergray}{RGB}{128,128,128}
\definecolor{sadergold}{RGB}{255,215,0}

% Márgenes exactos SADER
\geometry{top=2.5cm,bottom=2.5cm,left=3cm,right=3cm,headheight=20pt}

% Encabezado y pie de página
\pagestyle{fancy}
\fancyhf{}
\rfoot{\thepage}
\renewcommand{\headrulewidth}{0pt}
\fancyhead[L]{\includegraphics[width=2.8cm]{logo_sader.png}}

\begin{document}

% ========================================
% PORTADA OFICIAL
% ========================================
\begin{titlepage}
    \centering
    \vspace*{0.3cm}
    \includegraphics[width=0.25\textwidth]{logo_sader.png}\\[0.8cm]
    
    \vspace{0.4cm}
    {\normalsize\bfseries Perfil Profesional y Descripción Técnica:\par}
    \vspace{0.6cm}
    
    {\LARGE\bfseries INGENIERO AGRÓNOMO JUNIOR\par}
    {\LARGE\bfseries SUELOS TROPICALES Y RIEGO\par}
    \vspace{0.5cm}
    {\Large Especialista en Edafología y Sistemas de Riego:\par}
    \vspace{0.3cm}
    {\normalsize • Caracterización de Suelos Tropicales\par}
    {\normalsize • Sistemas de Riego Tecnificado\par}
    {\normalsize • Fertilidad y Nutrición Vegetal\par}
    {\normalsize • Conservación de Suelos y Agua\par}
    {\normalsize • Análisis de Laboratorio Especializados\par}
    \vspace{0.5cm}
    {\Large Macroproyecto Renacimiento Ganadero Maya\par}
    {\Large Yucatán 2026-2030\par}
    
    \vfill
    
    {\normalsize Mérida, Yucatán, 28 de noviembre de 2025\par}
    \vspace{0.2cm}
    {\normalsize SADER REPRESENTACIÓN ESTATAL YUCATÁN\par}
    {\normalsize Subdelegación Agropecuaria\par}
    \vspace{0.2cm}
    {\normalsize Código: ING-AGR-JR-006\par}
    {\normalsize Secretaría de Agricultura y Desarrollo Rural (SADER)\par}
\end{titlepage}

% ========================================
% ÍNDICE AUTOMÁTICO
% ========================================
\clearpage
\thispagestyle{empty}
\vspace*{3cm}
{\large\bfseries Contenido}\\[2cm]

\tableofcontents

\clearpage
\setcounter{page}{3}

% ========================================
% CONTENIDO
% ========================================

\section{Caracterización Institucional del Puesto}

\justifying

\subsection{Identificación Administrativa}

\begin{table}[H]
\centering
\caption{Datos Generales del Ingeniero Agrónomo Junior}
\label{tab:datos_generales}
\begin{tabular}{p{5cm}p{8cm}}
\toprule
\rowcolor{sadergreen!20}
\textbf{Campo} & \textbf{Especificación Técnica} \\
\midrule
Denominación oficial & Ingeniero Agrónomo Junior - Suelos Tropicales y Riego \\
Código de identificación & ING-AGR-JR-006 \\
Dependencia jerárquica & Ingeniero Agrónomo Senior \\
Línea de reporte directo & Coordinación técnica con Ingeniero Agrónomo Senior \\
Ámbito territorial & Estado de Yucatán (6,000 ha SSPi, 120 UPP) \\
Categoría de plaza & Técnico especializado nivel TC-10 tabulador SADER \\
Salario anual & \$360,000 MXN (financiado vía FOFAY) \\
Especialización primaria & Edafología tropical y sistemas de riego tecnificado \\
\bottomrule
\end{tabular}
\end{table}

\subsection{Contexto Programático Específico}

Este puesto especializado garantiza la sostenibilidad edáfica del Componente 1 (SSPi) mediante la caracterización detallada de suelos tropicales calcáreos de Yucatán y el diseño de sistemas de riego tecnificado adaptados a las condiciones kársticas regionales. La posición coordina estudios edafológicos en las 6,000 hectáreas de reconversión, supervisa la instalación de infraestructura de riego en 120 UPP y desarrolla protocolos de conservación de suelos que optimicen la captura de carbono y la productividad forrajera en sistemas silvopastoriles intensivos.

\section{Responsabilidades Principales}

\subsection{Caracterización de Suelos Tropicales}

\textbf{1. Estudios Edafológicos Especializados:}
\begin{itemize}
    \item Realizar caracterización físico-química de suelos en 120 UPP
    \item Determinar capacidad de campo y punto de marchitez permanente
    \item Evaluar estructura, textura y permeabilidad por horizonte
    \item Analizar contenido de materia orgánica y carbono orgánico
    \item Caracterizar pH, conductividad eléctrica y capacidad de intercambio
    \item Determinar disponibilidad de macro y micronutrientes
    \item Elaborar mapas de fertilidad por predio y región
\end{itemize}

\textbf{2. Análisis de Laboratorio:}
\begin{itemize}
    \item Coordinar análisis fisicoquímicos en laboratorio certificado
    \item Supervisar determinaciones de nitrógeno, fósforo y potasio
    \item Evaluar micronutrientes (Fe, Mn, Zn, B, Mo, Cu)
    \item Analizar actividad biológica y diversidad microbiana
    \item Determinar densidad aparente y porosidad total
    \item Medir infiltración básica y conductividad hidráulica
\end{itemize}

\subsection{Sistemas de Riego Tecnificado}

\textbf{3. Diseño de Infraestructura de Riego:}
\begin{itemize}
    \item Diseñar sistemas de riego por aspersión en 120 UPP
    \item Calcular necesidades hídricas por especie forrajera
    \item Dimensionar redes de distribución y almacenamiento
    \item Seleccionar equipos de bombeo y control automatizado
    \item Coordinar perforación de pozos profundos (cenotes)
    \item Supervisar instalación de infraestructura especializada
\end{itemize}

\textbf{4. Eficiencia y Programación del Riego:}
\begin{itemize}
    \item Desarrollar calendarios de riego por zona y época
    \item Calcular láminas y frecuencias óptimas por cultivo
    \item Implementar riego deficitario controlado en época seca
    \item Monitorear eficiencia de aplicación y distribución
    \item Capacitar productores en manejo de sistemas
    \item Evaluar ahorro de agua vs sistemas tradicionales
\end{itemize}

\subsection{Fertilidad y Nutrición Vegetal}

\textbf{5. Planes de Fertilización Específicos:}
\begin{itemize}
    \item Elaborar recomendaciones de fertilización por predio
    \item Calcular dosis de macro y micronutrientes por cultivo
    \item Coordinar aplicación de biofertilizantes especializados
    \item Monitorear respuesta nutricional de \textit{Leucaena leucocephala}
    \item Evaluar eficiencia de fertilización orgánica vs química
    \item Ajustar programas según análisis foliares
\end{itemize}

\textbf{6. Mejoramiento de Fertilidad Natural:}
\begin{itemize}
    \item Promover incorporación de materia orgánica
    \item Coordinar compostaje de residuos ganaderos
    \item Implementar rotaciones que incrementen fertilidad
    \item Evaluar fijación biológica de nitrógeno por leguminosas
    \item Monitorear incremento de carbono orgánico del suelo
    \item Documentar mejoras en propiedades físicas
\end{itemize}

\section{Perfil del Puesto}

\subsection{Requisitos Académicos y Experiencia}

\begin{table}[H]
\centering
\caption{Requisitos del Ingeniero Agrónomo Junior}
\label{tab:requisitos}
\begin{tabular}{p{4cm}p{9cm}}
\toprule
\rowcolor{sadergreen!20}
\textbf{Requisito} & \textbf{Detalle} \\
\midrule
Formación académica & Ingeniero Agrónomo con especialidad en suelos o riego. Especialización en edafología tropical o fertirrigación deseable \\
Experiencia mínima & 3 años en caracterización de suelos, 2 años en diseño de sistemas de riego \\
Conocimientos indispensables & Física y química de suelos, sistemas de riego, nutrición vegetal, interpretación de análisis, hidrología, conservación \\
Certificaciones deseables & Operador de equipos de riego tecnificado, análisis de suelos certificado \\
Idiomas & Inglés técnico para literatura especializada \\
Habilidades técnicas & Manejo de GPS/SIG, interpretación cartográfica, cálculos hidráulicos, análisis estadístico \\
Disponibilidad & 70\% trabajo de campo, disponibilidad para viajes regionales \\
\bottomrule
\end{tabular}
\end{table}

\subsection{Competencias Técnicas Específicas}

\textbf{Competencias obligatorias:}
\begin{itemize}
    \item Caracterización fisicoquímica de suelos tropicales
    \item Diseño hidráulico de sistemas de riego
    \item Interpretación de análisis de suelos y foliares
    \item Cálculo de necesidades hídricas de cultivos
    \item Manejo de equipos de medición especializados
    \item Elaboración de recomendaciones de fertilización
\end{itemize}

\textbf{Competencias deseables:}
\begin{itemize}
    \item Experiencia en suelos calcáreos kársticos
    \item Automatización de sistemas de riego
    \item Teledetección aplicada a suelos
    \item Modelación de flujo de agua en suelos
    \item Microbiología de suelos
    \item Sistemas de información geográfica (SIG)
\end{itemize}

\section{Indicadores de Desempeño}

\begin{table}[H]
\centering
\caption{Métricas del Ingeniero Agrónomo Junior}
\label{tab:kpis}
\begin{tabular}{p{6cm}p{3cm}p{3cm}}
\toprule
\rowcolor{sadergreen!20}
\textbf{Indicador} & \textbf{Meta} & \textbf{Frecuencia} \\
\midrule
UPP con caracterización edáfica & 120 & Quinquenal \\
Sistemas de riego instalados & 24/año & Anual \\
Análisis de suelos procesados & 240/año & Anual \\
Mapas de fertilidad elaborados & 24/año & Anual \\
Eficiencia de riego lograda & $\geq$85\% & Semestral \\
Incremento materia orgánica & +25\% & Trianual \\
Ahorro de agua documentado & 30\% & Anual \\
Productores capacitados & 120 & Anual \\
\bottomrule
\end{tabular}
\end{table}

\section{Metodología de Trabajo}

\subsection{Protocolo de Caracterización Edáfica}

\textbf{Fase 1: Reconocimiento y Muestreo}
\begin{itemize}
    \item Levantamiento topográfico y delimitación de unidades
    \item Apertura de perfiles modales representativos
    \item Muestreo sistemático por horizontes genéticos
    \item Georeferenciación de puntos de muestreo
    \item Documentación fotográfica especializada
\end{itemize}

\textbf{Fase 2: Análisis de Laboratorio}
\begin{itemize}
    \item Procesamiento en laboratorio certificado SEMARNAT
    \item Análisis físicos: textura, estructura, porosidad
    \item Análisis químicos: pH, MO, CIC, nutrientes disponibles
    \item Análisis biológicos: actividad enzimática, respiración
    \item Control de calidad y repetibilidad
\end{itemize}

\textbf{Fase 3: Interpretación y Recomendaciones}
\begin{itemize}
    \item Clasificación taxonómica por Soil Survey Staff
    \item Evaluación de aptitud para especies forrajeras
    \item Elaboración de mapas temáticos especializados
    \item Generación de recomendaciones específicas
    \item Validación con productores en campo
\end{itemize}

\subsection{Cronograma Anual de Actividades}

\textbf{Enero-Marzo (Caracterización):}
\begin{itemize}
    \item Muestreo intensivo de suelos (época seca)
    \item Procesamiento de análisis de laboratorio
    \item Elaboración de mapas de fertilidad
    \item Diseño de sistemas de riego nuevos
\end{itemize}

\textbf{Abril-Junio (Implementación):}
\begin{itemize}
    \item Instalación de infraestructura de riego
    \item Aplicación de enmiendas y fertilizantes
    \item Calibración de equipos de riego
    \item Capacitación en manejo de sistemas
\end{itemize}

\textbf{Julio-Septiembre (Monitoreo):}
\begin{itemize}
    \item Evaluación de eficiencia de riego
    \item Monitoreo de respuesta nutricional
    \item Ajustes en programación de riego
    \item Medición de infiltración y escurrimiento
\end{itemize}

\textbf{Octubre-Diciembre (Evaluación):}
\begin{itemize}
    \item Evaluación anual de mejoras edáficas
    \item Análisis foliares de especies establecidas
    \item Planificación siguiente ciclo
    \item Sistematización de resultados
\end{itemize}

\section{Coordinación Técnica}

El Ingeniero Agrónomo Junior coordinará con:

\begin{itemize}
    \item \textbf{Ingeniero Agrónomo Senior:} Metodologías de transferencia tecnológica
    \item \textbf{Zootecnista SSPi Senior:} Requerimientos específicos de especies
    \item \textbf{Especialista SIG/Carbono:} Mapeo y monitoreo satelital
    \item \textbf{INIFAP:} Validación de protocolos edafológicos
    \item \textbf{CICY:} Investigación en suelos tropicales
    \item \textbf{CONAGUA:} Permisos de aprovechamiento hídrico
\end{itemize}

\section{Equipamiento Especializado}

\subsection{Instrumentos de Campo}

\begin{itemize}
    \item \textbf{Caracterización:} Penetrómetro, densímetro, pH-metro portátil
    \item \textbf{Muestreo:} Barrena edafológica, cilindros volumétricos
    \item \textbf{Riego:} Pluviómetros, tensiómetros, medidores de caudal
    \item \textbf{Topografía:} GPS diferencial, nivel topográfico, estación total
    \item \textbf{Transporte:} Vehículo 4×4 equipado para trabajo de campo
\end{itemize}

\section{Condiciones Laborales}

\subsection{Condiciones Específicas}

\begin{itemize}
    \item \textbf{Base:} Laboratorio de suelos Mérida + trabajo de campo regional
    \item \textbf{Horario:} Lunes a sábado, adaptado a condiciones climáticas
    \item \textbf{Viajes:} 70\% tiempo en campo (rotación 120 UPP)
    \item \textbf{Salario anual:} \$298,740 MXN + prestaciones (Nivel O21)
    \item \textbf{Capacitación:} 40 horas anuales en edafología tropical
    \item \textbf{Seguridad:} Protocolos para trabajo en cenotes y cuevas
\end{itemize}

\section{Impacto Esperado}

\subsection{Contribución al Macroproyecto}

El Ingeniero Agrónomo Junior garantiza la base edáfica sólida para la implementación exitosa de SSPi, optimizando el uso del agua en un ambiente kárstico desafiante y maximizando la eficiencia de nutrientes para lograr las metas de captura de carbono (765,000 ton CO$_2$eq) y productividad forrajera (+280\%).

Su especialización determina la sostenibilidad a largo plazo de las 6,000 hectáreas convertidas y la viabilidad técnica del incremento de capacidad de carga proyectado.

\section{Conclusiones Técnicas}

La especialización en suelos tropicales y riego constituye el fundamento técnico para la viabilidad biofísica del macroproyecto. El manejo eficiente del recurso hídrico y la optimización de la fertilidad natural determinan el éxito de la intensificación sostenible proyectada.

\vspace{2cm}

\noindent\textbf{Ingeniero Agrónomo Senior - Transferencia Tecnológica}

\vspace{1cm}

\noindent\textbf{Jefe de Programa de Producción Pecuaria Sustentable}

\vspace{2cm}

\noindent\textit{Mérida, Yucatán, a 28 de noviembre de 2025}

\end{document}
