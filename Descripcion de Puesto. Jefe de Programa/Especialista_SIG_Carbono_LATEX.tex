\documentclass[12pt,letterpaper]{article}
\usepackage[utf8]{inputenc}
\usepackage[spanish]{babel}
\usepackage{geometry}
\usepackage{graphicx}
\usepackage{fancyhdr}
\usepackage{setspace}
\usepackage{lastpage}
\usepackage{parskip}
\usepackage{booktabs}
\usepackage{array}
\usepackage{multirow}
\usepackage{longtable}
\usepackage{float}
\usepackage{xcolor}
\usepackage{colortbl}
\usepackage{amsmath}
\usepackage{ragged2e}

% Colores SADER
\definecolor{saderblue}{RGB}{0,51,102}
\definecolor{sadergreen}{RGB}{34,139,34}
\definecolor{sadergray}{RGB}{128,128,128}
\definecolor{sadergold}{RGB}{255,215,0}

% Márgenes exactos SADER
\geometry{top=2.5cm,bottom=2.5cm,left=3cm,right=3cm,headheight=20pt}

% Encabezado y pie de página
\pagestyle{fancy}
\fancyhf{}
\rfoot{\thepage}
\renewcommand{\headrulewidth}{0pt}
\fancyhead[L]{\includegraphics[width=2.8cm]{logo_sader.png}}

\begin{document}

% ========================================
% PORTADA OFICIAL
% ========================================
\begin{titlepage}
    \centering
    \vspace*{0.3cm}
    \includegraphics[width=0.25\textwidth]{logo_sader.png}\\[0.8cm]
    
    \vspace{0.4cm}
    {\normalsize\bfseries Perfil Profesional y Descripción Técnica:\par}
    \vspace{0.6cm}
    
    {\LARGE\bfseries ESPECIALISTA SIG/CARBONO\par}
    {\LARGE\bfseries MONITOREO SATELITAL Y CAPTURA\par}
    \vspace{0.5cm}
    {\Large Experto en Teledetección y Mercados de Carbono:\par}
    \vspace{0.3cm}
    {\normalsize • Sistemas de Información Geográfica (SIG)\par}
    {\normalsize • Teledetección Satelital Multi-temporal\par}
    {\normalsize • Metodologías IPCC para Captura de Carbono\par}
    {\normalsize • Mercados Internacionales de Carbono\par}
    {\normalsize • Monitoreo, Reporte y Verificación (MRV)\par}
    \vspace{0.5cm}
    {\Large Macroproyecto Renacimiento Ganadero Maya\par}
    {\Large Yucatán 2026-2030\par}
    
    \vfill
    
    {\normalsize Mérida, Yucatán, 28 de noviembre de 2025\par}
    \vspace{0.2cm}
    {\normalsize SADER REPRESENTACIÓN ESTATAL YUCATÁN\par}
    {\normalsize Subdelegación Agropecuaria\par}
    \vspace{0.2cm}
    {\normalsize Código: ESP-SIG-CB-007\par}
    {\normalsize Secretaría de Agricultura y Desarrollo Rural (SADER)\par}
\end{titlepage}

% ========================================
% ÍNDICE AUTOMÁTICO
% ========================================
\clearpage
\thispagestyle{empty}
\vspace*{3cm}
{\large\bfseries Contenido}\\[2cm]

\tableofcontents

\clearpage
\setcounter{page}{3}

% ========================================
% CONTENIDO
% ========================================

\section{Caracterización Institucional del Puesto}

\justifying

\subsection{Identificación Administrativa}

\begin{table}[H]
\centering
\caption{Datos Generales del Especialista SIG/Carbono}
\label{tab:datos_generales}
\begin{tabular}{p{5cm}p{8cm}}
\toprule
\rowcolor{sadergreen!20}
\textbf{Campo} & \textbf{Especificación Técnica} \\
\midrule
Denominación oficial & Especialista SIG/Carbono - Monitoreo y Verificación \\
Código de identificación & ESP-SIG-CB-007 \\
Dependencia jerárquica & Jefe de Programa de Producción Pecuaria Sustentable \\
Línea de reporte directo & Coordinación técnica directa con Jefe de Programa \\
Ámbito territorial & Estado de Yucatán (6,000 ha SSPi, 765,000 ton CO$_2$eq) \\
Categoría de plaza & Especialista técnico nivel TC-14 tabulador SADER \\
Salario anual & \$480,000 MXN (financiado vía FOFAY) \\
Especialización primaria & Teledetección, SIG y metodologías IPCC de carbono \\
\bottomrule
\end{tabular}
\end{table}

\subsection{Contexto Programático Específico}

Este puesto altamente especializado coordina el monitoreo satelital integral del Macroproyecto y lidera la cuantificación científica de servicios ecosistémicos, especialmente la captura de 765,000 toneladas CO$_2$eq en sistemas silvopastoriles intensivos. La posición desarrolla e implementa protocolos de Monitoreo, Reporte y Verificación (MRV) conforme a estándares internacionales IPCC, coordina la comercialización en mercados voluntarios de carbono y garantiza la transparencia científica de los beneficios ambientales del programa de \$814.9 millones MXN mediante tecnologías geoespaciales avanzadas.

\section{Responsabilidades Principales}

\subsection{Sistemas de Información Geográfica}

\textbf{1. Plataforma SIG Integral del Macroproyecto:}
\begin{itemize}
    \item Desarrollar geodatabase centralizada de 120 UPP participantes
    \item Implementar sistema de monitoreo en tiempo real vía web
    \item Georefenciar todas las intervenciones por componente
    \item Integrar datos de sensores remotos multiespectrales
    \item Coordinar actualización cartográfica continua
    \item Generar dashboards ejecutivos para toma de decisiones
    \item Mantener interoperabilidad con sistemas SADER/SIAP
\end{itemize}

\textbf{2. Cartografía Especializada:}
\begin{itemize}
    \item Elaborar mapas base de cobertura y uso del suelo
    \item Generar cartografía de cambios de cobertura multi-temporal
    \item Producir mapas de biomasa aérea y subterránea
    \item Desarrollar mapas de servicios ecosistémicos cuantificados
    \item Crear productos cartográficos para mercados de carbono
    \item Elaborar atlas técnico del macroproyecto
\end{itemize}

\subsection{Teledetección Satelital}

\textbf{3. Monitoreo Multi-temporal por Sensores Remotos:}
\begin{itemize}
    \item Procesar imágenes Landsat 8/9 y Sentinel-2 mensuales
    \item Implementar índices de vegetación (NDVI, EVI, SAVI)
    \item Calcular biomasa mediante ecuaciones alométricas satelitales
    \item Detectar cambios de cobertura mediante análisis multi-temporal
    \item Monitorear establishment de \textit{Leucaena leucocephala} por pixel
    \item Evaluar degradación/recuperación de pastizales
    \item Validar con mediciones terrestres (ground truth)
\end{itemize}

\textbf{4. Procesamiento de Imágenes Especializadas:}
\begin{itemize}
    \item Corrección radiométrica y atmosférica de imágenes
    \item Clasificación supervisada de coberturas vegetales
    \item Análisis de series temporales de índices espectrales
    \item Detección de anomalías y cambios abruptos
    \item Fusión de datos multisensor (óptico + radar)
    \item Generación de productos derivados especializados
\end{itemize}

\subsection{Captura y Almacenamiento de Carbono}

\textbf{5. Metodologías IPCC para Cuantificación:}
\begin{itemize}
    \item Implementar metodología IPCC 2019 para sistemas agroforestales
    \item Calcular factores de emisión específicos por región
    \item Cuantificar carbono en biomasa aérea, subterránea y suelo
    \item Establecer línea base y escenario de referencia
    \item Monitorear incremento anual de reservas de carbono
    \item Documentar adicionalidad y permanencia del proyecto
    \item Coordinar verificación por tercera parte independiente
\end{itemize}

\textbf{6. Mercados Internacionales de Carbono:}
\begin{itemize}
    \item Desarrollar Documento de Diseño de Proyecto (PDD)
    \item Coordinar con estándares VCS, Gold Standard, Plan Vivo
    \item Gestionar proceso de validación y registro internacional
    \item Implementar sistema MRV (Monitoreo, Reporte, Verificación)
    \item Coordinar auditorías independientes anuales
    \item Facilitar comercialización de créditos de carbono
\end{itemize}

\section{Perfil del Puesto}

\subsection{Requisitos Académicos y Experiencia}

\begin{table}[H]
\centering
\caption{Requisitos del Especialista SIG/Carbono}
\label{tab:requisitos}
\begin{tabular}{p{4cm}p{9cm}}
\toprule
\rowcolor{sadergreen!20}
\textbf{Requisito} & \textbf{Detalle} \\
\midrule
Formación académica & Ingeniero Geógrafo, Biólogo o Forestal con especialización en SIG. Maestría en teledetección, cambio climático o servicios ecosistémicos \\
Experiencia mínima & 5 años en teledetección, 3 años en proyectos de carbono o REDD+ \\
Conocimientos indispensables & SIG avanzado, procesamiento de imágenes satelitales, metodologías IPCC, mercados de carbono, estadística espacial \\
Certificaciones obligatorias & Certificación en software SIG profesional (ArcGIS/QGIS), metodologías IPCC \\
Idiomas & Inglés avanzado (literatura científica y mercados internacionales) \\
Habilidades técnicas & Programación en Python/R, bases de datos espaciales, modelación espacial, análisis multivariado \\
Disponibilidad & 60\% oficina técnica, 40\% campo para validación \\
\bottomrule
\end{tabular}
\end{table}

\subsection{Competencias Técnicas Específicas}

\textbf{Competencias obligatorias:}
\begin{itemize}
    \item Manejo avanzado de ArcGIS Pro/QGIS y ENVI/ERDAS
    \item Procesamiento de imágenes Landsat, Sentinel, MODIS
    \item Metodologías IPCC 2019 para AFOLU (Agriculture, Forestry and Other Land Use)
    \item Estándares internacionales de carbono (VCS, Gold Standard)
    \item Programación en Python para procesamiento automatizado
    \item Análisis estadístico espacial y series temporales
\end{itemize}

\textbf{Competencias deseables:}
\begin{itemize}
    \item Experiencia en proyectos REDD+ o Mecanismo de Desarrollo Limpio
    \item Manejo de Google Earth Engine y plataformas cloud
    \item Conocimientos en LiDAR y fotogrametría con drones
    \item Modelación de carbono con herramientas especializadas
    \item Experiencia en mercados voluntarios de carbono
    \item Certificación en auditoría de proyectos ambientales
\end{itemize}

\section{Indicadores de Desempeño}

\begin{table}[H]
\centering
\caption{Métricas del Especialista SIG/Carbono}
\label{tab:kpis}
\begin{tabular}{p{6cm}p{3cm}p{3cm}}
\toprule
\rowcolor{sadergreen!20}
\textbf{Indicador} & \textbf{Meta} & \textbf{Frecuencia} \\
\midrule
Hectáreas monitoreadas vía SIG & 6,000 ha & Permanente \\
Toneladas CO$_2$eq cuantificadas & 153,000/año & Anual \\
Mapas actualizados & 12/año & Mensual \\
Imágenes procesadas & 24/año & Mensual \\
Precisión clasificación cobertura & $\geq$90\% & Semestral \\
Reportes MRV elaborados & 4/año & Trimestral \\
Créditos carbono verificados & 765,000 tot & Quinquenal \\
Dashboard actualizado & 365/año & Diario \\
\bottomrule
\end{tabular}
\end{table}

\section{Metodología de Trabajo}

\subsection{Protocolo de Monitoreo Satelital}

\textbf{Fase 1: Adquisición y Preprocesamiento}
\begin{itemize}
    \item Descarga automatizada de imágenes Landsat/Sentinel
    \item Corrección atmosférica mediante algoritmos Sen2Cor/LaSRC
    \item Construcción de composites libres de nubes mensuales
    \item Calibración radiométrica y georreferenciación precisa
    \item Control de calidad y validación de productos
\end{itemize}

\textbf{Fase 2: Procesamiento y Análisis}
\begin{itemize}
    \item Cálculo de índices de vegetación especializados
    \item Clasificación de coberturas mediante machine learning
    \item Análisis de cambios multi-temporales (LandTrendr, BFAST)
    \item Estimación de biomasa mediante ecuaciones alométricas
    \item Cuantificación de carbono por reservorio (aéreo/subterráneo/suelo)
\end{itemize}

\textbf{Fase 3: Validación y Verificación}
\begin{itemize}
    \item Validación con datos de campo (ground truth)
    \item Análisis de incertidumbre y propagación de errores
    \item Verificación cruzada con inventarios forestales
    \item Auditoría por terceras partes independientes
    \item Documentación conforme estándares internacionales
\end{itemize}

\subsection{Cronograma Anual de Actividades}

\textbf{Enero-Marzo (Línea Base):}
\begin{itemize}
    \item Actualización línea base de carbono
    \item Procesamiento imágenes época seca
    \item Elaboración reporte anual MRV
    \item Planificación auditoría externa
\end{itemize}

\textbf{Abril-Junio (Establishment):}
\begin{itemize}
    \item Monitoreo establishment SSPi intensivo
    \item Validación de campo systematic  a
    \item Actualización cartografía de avances
    \item Calibración modelos biomasa
\end{itemize}

\textbf{Julio-Septiembre (Desarrollo):}
\begin{itemize}
    \item Monitoreo crecimiento vegetal época lluvias
    \item Procesamiento series temporales completas
    \item Análisis de cambios interanuales
    \item Detección de anomalías climáticas
\end{itemize}

\textbf{Octubre-Diciembre (Cuantificación):}
\begin{itemize}
    \item Cuantificación anual captura de carbono
    \item Elaboración productos para mercados
    \item Preparación documentos verificación
    \item Sistematización metodológica anual
\end{itemize}

\section{Coordinación Técnica}

El Especialista SIG/Carbono coordinará con:

\begin{itemize}
    \item \textbf{Zootecnista SSPi Senior:} Datos de establecimiento y crecimiento
    \item \textbf{Ingeniero Agrónomo Junior:} Información edafológica especializada
    \item \textbf{CICY:} Investigación en ecuaciones alométricas regionales
    \item \textbf{CONABIO:} Intercambio de información geoespacial
    \item \textbf{CONAFOR:} Metodologías nacionales de carbono forestal
    \item \textbf{Verificadores Internacionales:} Auditorías de tercera parte
\end{itemize}

\section{Infraestructura Tecnológica}

\subsection{Equipamiento Especializado}

\begin{itemize}
    \item \textbf{Estación SIG:} Workstation con 64GB RAM, GPU especializada
    \item \textbf{Software:} Licencias ArcGIS Pro, ENVI, ERDAS IMAGINE
    \item \textbf{Almacenamiento:} Servidor con 20TB para imágenes satelitales
    \item \textbf{Campo:} GPS sub-métrico, espectroradiómetro portátil
    \item \textbf{Conectividad:} Internet de alta velocidad para descarga satelital
\end{itemize}

\section{Condiciones Laborales}

\subsection{Condiciones Específicas}

\begin{itemize}
    \item \textbf{Base:} Laboratorio SIG especializado en Mérida
    \item \textbf{Horario:} Lunes a viernes, horarios flexibles para procesamiento
    \item \textbf{Viajes:} 40\% campo para validación y ground truth
    \item \textbf{Salario anual:} \$480,000 MXN + prestaciones especializadas
    \item \textbf{Capacitación:} 80 horas anuales en tecnologías geoespaciales
    \item \textbf{Certificaciones:} Renovación certificaciones internacionales
\end{itemize}

\section{Productos Esperados}

\subsection{Entregables Especializados}

\begin{itemize}
    \item \textbf{Atlas Técnico:} Cartografía completa del macroproyecto
    \item \textbf{Reportes MRV:} Documentación trimestral conforme IPCC
    \item \textbf{Dashboard Ejecutivo:} Plataforma web tiempo real
    \item \textbf{Base de Datos Geoespacial:} Geodatabase integral del proyecto
    \item \textbf{Créditos de Carbono:} 765,000 tCO$_2$eq verificados
    \item \textbf{Publicaciones Científicas:} Papers en revistas indexadas
\end{itemize}

\section{Conclusiones Técnicas}

El Especialista SIG/Carbono garantiza la credibilidad científica y transparencia del macroproyecto mediante tecnologías geoespaciales de vanguardia. Su expertise determina la elegibilidad para mercados internacionales de carbono y la generación de ingresos adicionales estimados en \$7.65 millones USD por comercialización de créditos ambientales.

La plataforma de monitoreo satelital constituye infraestructura estratégica para el escalamiento del programa a nivel nacional e internacional.

\vspace{2cm}

\noindent\textbf{Jefe de Programa de Producción Pecuaria Sustentable}

\vspace{1cm}

\noindent\textbf{Subdelegado Agropecuario - SADER Yucatán}

\vspace{2cm}

\noindent\textit{Mérida, Yucatán, a 28 de noviembre de 2025}

\end{document}