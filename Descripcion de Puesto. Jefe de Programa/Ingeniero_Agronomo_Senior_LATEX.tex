\documentclass[12pt,letterpaper]{article}
\usepackage[utf8]{inputenc}
\usepackage[spanish]{babel}
\usepackage{geometry}
\usepackage{graphicx}
\usepackage{fancyhdr}
\usepackage{setspace}
\usepackage{lastpage}
\usepackage{parskip}
\usepackage{booktabs}
\usepackage{array}
\usepackage{multirow}
\usepackage{longtable}
\usepackage{float}
\usepackage{xcolor}
\usepackage{colortbl}
\usepackage{amsmath}
\usepackage{ragged2e}

% Colores SADER
\definecolor{saderblue}{RGB}{0,51,102}
\definecolor{sadergreen}{RGB}{34,139,34}
\definecolor{sadergray}{RGB}{128,128,128}
\definecolor{sadergold}{RGB}{255,215,0}

% Márgenes exactos SADER
\geometry{top=2.5cm,bottom=2.5cm,left=3cm,right=3cm,headheight=20pt}

% Encabezado y pie de página
\pagestyle{fancy}
\fancyhf{}
\rfoot{\thepage}
\renewcommand{\headrulewidth}{0pt}
\fancyhead[L]{\includegraphics[width=2.8cm]{logo_sader.png}}

\begin{document}

% ========================================
% PORTADA OFICIAL
% ========================================
\begin{titlepage}
    \centering
    \vspace*{0.3cm}
    \includegraphics[width=0.25\textwidth]{logo_sader.png}\\[0.8cm]
    
    \vspace{0.4cm}
    {\normalsize\bfseries Perfil Profesional y Descripción Técnica:\par}
    \vspace{0.6cm}
    
    {\LARGE\bfseries INGENIERO AGRÓNOMO SENIOR\par}
    {\LARGE\bfseries TRANSFERENCIA TECNOLÓGICA Y ECAs\par}
    \vspace{0.5cm}
    {\Large Líder en Metodologías Participativas y Extensión:\par}
    \vspace{0.3cm}
    {\normalsize • Escuelas de Campo para Agricultores (ECAs)\par}
    {\normalsize • Transferencia Tecnológica Horizontal\par}
    {\normalsize • Metodologías Participativas de Aprendizaje\par}
    {\normalsize • Sistemas de Conocimiento Local\par}
    {\normalsize • Desarrollo de Capacidades Productivas\par}
    \vspace{0.5cm}
    {\Large Macroproyecto Renacimiento Ganadero Maya\par}
    {\Large Yucatán 2026-2030\par}
    
    \vfill
    
    {\normalsize Mérida, Yucatán, 28 de noviembre de 2025\par}
    \vspace{0.2cm}
    {\normalsize SADER REPRESENTACIÓN ESTATAL YUCATÁN\par}
    {\normalsize Subdelegación Agropecuaria\par}
    \vspace{0.2cm}
    {\normalsize Código: ING-AGR-SR-005\par}
    {\normalsize Secretaría de Agricultura y Desarrollo Rural (SADER)\par}
\end{titlepage}

% ========================================
% ÍNDICE AUTOMÁTICO
% ========================================
\clearpage
\thispagestyle{empty}
\vspace*{3cm}
{\large\bfseries Contenido}\\[2cm]

\tableofcontents

\clearpage
\setcounter{page}{3}

% ========================================
% CONTENIDO
% ========================================

\section{Caracterización Institucional del Puesto}

\justifying

\subsection{Identificación Administrativa}

\begin{table}[H]
\centering
\caption{Datos Generales del Ingeniero Agrónomo Senior}
\label{tab:datos_generales}
\begin{tabular}{p{5cm}p{8cm}}
\toprule
\rowcolor{sadergreen!20}
\textbf{Campo} & \textbf{Especificación Técnica} \\
\midrule
Denominación oficial & Ingeniero Agrónomo Senior - Transferencia Tecnológica \\
Código de identificación & ING-AGR-SR-005 \\
Dependencia jerárquica & Jefe de Programa de Producción Pecuaria Sustentable \\
Línea de reporte directo & Coordinación técnica directa con Jefe de Programa \\
Ámbito territorial & Estado de Yucatán (5 ECAs, 125 productores, 120 UPP) \\
Categoría de plaza & Técnico especializado nivel TC-12 tabulador SADER \\
Salario anual & \$360,000 MXN (financiado vía FOFAY) \\
Especialización primaria & Metodologías participativas y extensión agrícola \\
\bottomrule
\end{tabular}
\end{table}

\subsection{Contexto Programático Específico}

Este puesto lidera estratégicamente la transferencia tecnológica horizontal del Macroproyecto mediante metodologías participativas de aprendizaje que garantizan la adopción exitosa de tecnologías SSPi por parte de 125 productores en 120 UPP. La posición coordina el sistema de 5 Escuelas de Campo para Agricultores (ECAs) y facilita procesos de construcción social del conocimiento que integran saberes tradicionales mayas con innovaciones tecnológicas científicamente validadas, asegurando la sostenibilidad social del programa de \$814.9 millones MXN.

\section{Responsabilidades Principales}

\subsection{Escuelas de Campo para Agricultores (ECAs)}

\textbf{1. Coordinación del Sistema de ECAs:}
\begin{itemize}
    \item Establecer y operar 5 ECAs especializadas en SSPi (25 productores/ECA)
    \item Desarrollar metodología participativa adaptada al contexto maya
    \item Facilitar 125 sesiones anuales (25 por ECA) de aprendizaje experiencial
    \item Coordinar intercambios horizontales entre ECAs regionales
    \item Establecer UPP demostrativas como centros de aprendizaje
    \item Sistematizar lecciones aprendidas y buenas prácticas
\end{itemize}

\textbf{2. Metodología Participativa Especializada:}
\begin{itemize}
    \item Implementar ciclo completo de aprendizaje experiencial
    \item Facilitar análisis agroecosistémico participativo (AAEP)
    \item Coordinar experimentación campesina validada científicamente
    \item Desarrollar materiales didácticos en español y maya
    \item Integrar conocimientos tradicionales con innovación tecnológica
    \item Evaluar impacto de capacitación en adopción de SSPi
\end{itemize}

\subsection{Transferencia Tecnológica Horizontal}

\textbf{3. Redes de Aprendizaje Campesino:}
\begin{itemize}
    \item Establecer red de productores líderes (20 innovadores tempranos)
    \item Coordinar giras tecnológicas y días de campo especializados
    \item Facilitar intercambios de experiencias inter-regionales
    \item Desarrollar sistema de campesino a campesino (CAC)
    \item Promover formación de promotores rurales especializados
    \item Documentar y sistematizar innovaciones campesinas
\end{itemize}

\textbf{4. Adopción Tecnológica Diferenciada:}
\begin{itemize}
    \item Adaptar paquetes tecnológicos por tipología de productor
    \item Diseñar rutas de adopción gradual y progresiva
    \item Coordinar asistencia técnica especializada continua
    \item Implementar seguimiento de adopción por componente tecnológico
    \item Evaluar barreras técnicas, económicas y culturales
    \item Desarrollar estrategias de superación de obstáculos
\end{itemize}

\subsection{Desarrollo de Capacidades}

\textbf{5. Fortalecimiento Organizacional:}
\begin{itemize}
    \item Fortalecer capacidades de organizaciones ganaderas (UGROY/UGRY)
    \item Facilitar procesos de planeación participativa comunitaria
    \item Desarrollar liderazgo técnico en comunidades rurales
    \item Promover participación de mujeres y jóvenes ($\geq$35\%)
    \item Coordinar con autoridades tradicionales mayas
    \item Establecer comités técnicos locales especializados
\end{itemize}

\section{Perfil del Puesto}

\subsection{Requisitos Académicos y Experiencia}

\begin{table}[H]
\centering
\caption{Requisitos del Ingeniero Agrónomo Senior}
\label{tab:requisitos}
\begin{tabular}{p{4cm}p{9cm}}
\toprule
\rowcolor{sadergreen!20}
\textbf{Requisito} & \textbf{Detalle} \\
\midrule
Formación académica & Ingeniero Agrónomo con especialidad en extensión rural o desarrollo rural. Maestría en desarrollo rural o metodologías participativas deseable \\
Experiencia mínima & 7 años en extensión agrícola, 5 años facilitando ECAs o metodologías participativas \\
Conocimientos indispensables & ECAs, metodologías participativas, extensión rural, facilitación grupal, sistematización de experiencias, desarrollo organizacional \\
Certificaciones obligatorias & Facilitador certificado en ECAs por FAO u organismo internacional equivalente \\
Idiomas & Maya yucateco básico conversacional, inglés técnico para metodologías \\
Habilidades técnicas & Facilitación de procesos grupales, sistematización, diagnósticos participativos, comunicación intercultural \\
Disponibilidad & 70\% trabajo de campo comunitario, movilidad regional completa \\
\bottomrule
\end{tabular}
\end{table}

\subsection{Competencias Técnicas Específicas}

\textbf{Competencias obligatorias:}
\begin{itemize}
    \item Metodología de Escuelas de Campo (ECAs) certificada
    \item Facilitación de procesos participativos
    \item Sistematización de experiencias rurales
    \item Comunicación intercultural (español-maya)
    \item Diagnóstico rural participativo (DRP)
    \item Metodologías de campesino a campesino
\end{itemize}

\textbf{Competencias deseables:}
\begin{itemize}
    \item Antropología rural aplicada
    \item Sistemas de conocimiento tradicional
    \item Planeación participativa comunitaria
    \item Género y desarrollo rural
    \item Comunicación para el desarrollo
    \item Evaluación de impacto social
\end{itemize}

\section{Indicadores de Desempeño}

\begin{table}[H]
\centering
\caption{Métricas del Ingeniero Agrónomo Senior}
\label{tab:kpis}
\begin{tabular}{p{6cm}p{3cm}p{3cm}}
\toprule
\rowcolor{sadergreen!20}
\textbf{Indicador} & \textbf{Meta} & \textbf{Frecuencia} \\
\midrule
ECAs operando efectivamente & 5 & Permanente \\
Productores capacitados activos & 125 & Anual \\
Sesiones ECA impartidas & 125/año & Anual \\
Tasa adopción SSPi por ECA & $\geq$80\% & Anual \\
Intercambios horizontales & 12/año & Anual \\
Promotores rurales formados & 20 & Bianual \\
Mujeres/jóvenes participantes & $\geq$35\% & Permanente \\
Experiencias sistematizadas & 5/año & Anual \\
\bottomrule
\end{tabular}
\end{table}

\section{Metodología de Trabajo}

\subsection{Ciclo de las Escuelas de Campo}

\textbf{Fase 1: Establecimiento (Meses 1-3):}
\begin{itemize}
    \item Diagnóstico participativo comunitario
    \item Selección de productores y UPP demostrativas
    \item Establecimiento de parcelas de aprendizaje
    \item Conformación de grupos de trabajo
    \item Elaboración participativa de agenda de aprendizaje
\end{itemize}

\textbf{Fase 2: Desarrollo (Meses 4-15):}
\begin{itemize}
    \item Sesiones semanales de campo (calendario agrícola)
    \item Experimentación campesina supervisada
    \item Análisis agroecosistémico participativo continuo
    \item Intercambios con otras ECAs
    \item Sistematización de observaciones y resultados
\end{itemize}

\textbf{Fase 3: Consolidación (Meses 16-24):}
\begin{itemize}
    \item Evaluación participativa de resultados
    \item Formación de promotores especializados
    \item Diseño de réplica en nuevas comunidades
    \item Sistematización de lecciones aprendidas
    \item Graduación y seguimiento post-ECA
\end{itemize}

\subsection{Cronograma Anual de Actividades}

\textbf{Enero-Marzo (Planificación):}
\begin{itemize}
    \item Diagnósticos comunitarios actualizados
    \item Programación anual participativa de ECAs
    \item Selección de nuevos participantes
    \item Capacitación de promotores rurales
\end{itemize}

\textbf{Abril-Junio (Establecimiento):}
\begin{itemize}
    \item Inicio de nuevos ciclos de ECAs
    \item Establecimiento de parcelas demostrativas
    \item Sesiones intensivas de campo
    \item Intercambios inter-ECAs programados
\end{itemize}

\textbf{Julio-Septiembre (Desarrollo):}
\begin{itemize}
    \item Seguimiento de experimentación campesina
    \item Evaluación participativa de resultados
    \item Ajustes metodológicos por ECA
    \item Documentación de innovaciones
\end{itemize}

\textbf{Octubre-Diciembre (Sistematización):}
\begin{itemize}
    \item Evaluación final de ciclos ECAs
    \item Sistematización de experiencias exitosas
    \item Planificación siguientes ciclos
    \item Graduación de promotores certificados
\end{itemize}

\section{Coordinación Técnica}

El Ingeniero Agrónomo Senior coordinará con:

\begin{itemize}
    \item \textbf{Zootecnista SSPi Senior:} Contenidos técnicos especializados
    \item \textbf{Ingeniero Agrónomo Junior:} Aspectos de suelos y nutrición
    \item \textbf{Organizaciones Ganaderas:} UGROY, UGRY, cooperativas locales
    \item \textbf{Autoridades Tradicionales:} Comisarios ejidales, líderes mayas
    \item \textbf{INIFAP:} Validación científica de innovaciones campesinas
    \item \textbf{Universidad Autónoma de Yucatán:} Metodologías de investigación participativa
\end{itemize}

\section{Materiales y Recursos}

\subsection{Infraestructura de Apoyo}

\begin{itemize}
    \item \textbf{Centros de Aprendizaje:} 5 UPP demostrativas equipadas
    \item \textbf{Material Didáctico:} Rotafolios, videos, infografías en maya
    \item \textbf{Parcelas Experimentales:} 2.5 ha por ECA con diferentes tratamientos
    \item \textbf{Equipamiento de Campo:} Balanzas, medidores, herramientas de evaluación
    \item \textbf{Transporte Especializado:} Vehículos para movilización grupal
\end{itemize}

\section{Condiciones Laborales}

\subsection{Condiciones Específicas}

\begin{itemize}
    \item \textbf{Base:} Rotación entre 5 regiones ECAs
    \item \textbf{Horario:} Adaptado a calendario agrícola y disponibilidad campesina
    \item \textbf{Viajes:} 70\% tiempo en comunidades rurales
    \item \textbf{Salario anual:} \$360,000 MXN + prestaciones
    \item \textbf{Capacitación:} 60 horas anuales en metodologías participativas
    \item \textbf{Comunicación:} Competencia básica en maya yucateco
\end{itemize}

\section{Sistema de Seguimiento}

\subsection{Monitoreo Participativo}

\begin{itemize}
    \item \textbf{Herramientas Campesinas:} Cuadernos de campo, calendarios estacionales
    \item \textbf{Indicadores Locales:} Definidos participativamente por comunidad
    \item \textbf{Evaluación Continua:} Reflexión grupal en cada sesión ECA
    \item \textbf{Sistematización:} Documentación audiovisual de procesos
    \item \textbf{Impacto Social:} Medición de cambios organizacionales
\end{itemize}

\section{Conclusiones Técnicas}

El Ingeniero Agrónomo Senior garantiza la sostenibilidad social y cultural del Macroproyecto mediante procesos participativos que respetan y potencian el conocimiento tradicional maya. Su expertise en metodologías de aprendizaje horizontal determina la tasa de adopción tecnológica y la apropiación comunitaria de innovaciones SSPi.

La coordinación efectiva de ECAs constituye el mecanismo clave para la escalabilidad del programa y su replicación en otras regiones ganaderas de México.

\vspace{2cm}

\noindent\textbf{Jefe de Programa de Producción Pecuaria Sustentable}

\vspace{1cm}

\noindent\textbf{Subdelegado Agropecuario - SADER Yucatán}

\vspace{2cm}

\noindent\textit{Mérida, Yucatán, a 28 de noviembre de 2025}

\end{document}