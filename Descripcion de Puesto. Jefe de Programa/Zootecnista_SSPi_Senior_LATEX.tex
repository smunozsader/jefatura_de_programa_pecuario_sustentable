\documentclass[12pt,letterpaper]{article}
\usepackage[utf8]{inputenc}
\usepackage[spanish]{babel}
\usepackage{geometry}
\usepackage{graphicx}
\usepackage{fancyhdr}
\usepackage{setspace}
\usepackage{lastpage}
\usepackage{parskip}
\usepackage{booktabs}
\usepackage{array}
\usepackage{multirow}
\usepackage{longtable}
\usepackage{float}
\usepackage{xcolor}
\usepackage{colortbl}
\usepackage{amsmath}
\usepackage{ragged2e}

% Colores SADER
\definecolor{saderblue}{RGB}{0,51,102}
\definecolor{sadergreen}{RGB}{34,139,34}
\definecolor{sadergray}{RGB}{128,128,128}
\definecolor{sadergold}{RGB}{255,215,0}

% Márgenes exactos SADER
\geometry{top=2.5cm,bottom=2.5cm,left=3cm,right=3cm,headheight=20pt}

% Encabezado y pie de página
\pagestyle{fancy}
\fancyhf{}
\rfoot{\thepage}
\renewcommand{\headrulewidth}{0pt}
\fancyhead[L]{\includegraphics[width=2.8cm]{logo_sader.png}}

\begin{document}

% ========================================
% PORTADA OFICIAL
% ========================================
\begin{titlepage}
    \centering
    \vspace*{0.3cm}
    \includegraphics[width=0.25\textwidth]{logo_sader.png}\\[0.8cm]
    
    \vspace{0.4cm}
    {\normalsize\bfseries Perfil Profesional y Descripción Técnica:\par}
    \vspace{0.6cm}
    
    {\LARGE\bfseries ZOOTECNISTA SENIOR\par}
    {\LARGE\bfseries SISTEMAS SILVOPASTORILES INTENSIVOS\par}
    \vspace{0.5cm}
    {\Large Líder Técnico en SSPi y Pastoreo Racional:\par}
    \vspace{0.3cm}
    {\normalsize • Leucaena leucocephala (40,000-53,000 plantas/ha)\par}
    {\normalsize • Pastoreo Racional Adaptativo\par}
    {\normalsize • Escuelas de Campo Silvopastoriles\par}
    {\normalsize • Captura de Carbono y Servicios Ecosistémicos\par}
    {\normalsize • Transferencia Tecnológica Horizontal\par}
    \vspace{0.5cm}
    {\Large Macroproyecto Renacimiento Ganadero Maya\par}
    {\Large Yucatán 2026-2030\par}
    
    \vfill
    
    {\normalsize Mérida, Yucatán, 28 de noviembre de 2025\par}
    \vspace{0.2cm}
    {\normalsize SADER REPRESENTACIÓN ESTATAL YUCATÁN\par}
    {\normalsize Subdelegación Agropecuaria\par}
    \vspace{0.2cm}
    {\normalsize Código: ZOO-SSPi-SR-003\par}
    {\normalsize Secretaría de Agricultura y Desarrollo Rural (SADER)\par}
\end{titlepage}

% ========================================
% ÍNDICE AUTOMÁTICO
% ========================================
\clearpage
\thispagestyle{empty}
\vspace*{3cm}
{\large\bfseries Contenido}\\[2cm]

\tableofcontents

\clearpage
\setcounter{page}{3}

% ========================================
% CONTENIDO
% ========================================

\section{Caracterización Institucional del Puesto}

\justifying

\subsection{Identificación Administrativa}

\begin{table}[H]
\centering
\caption{Datos Generales del Zootecnista SSPi Senior}
\label{tab:datos_generales}
\begin{tabular}{p{5cm}p{8cm}}
\toprule
\rowcolor{sadergreen!20}
\textbf{Campo} & \textbf{Especificación Técnica} \\
\midrule
Denominación oficial & Zootecnista Senior - Sistemas Silvopastoriles Intensivos \\
Código de identificación & ZOO-SSPi-SR-003 \\
Dependencia jerárquica & Jefe de Programa de Producción Pecuaria Sustentable \\
Línea de reporte directo & Coordinación técnica directa con Jefe de Programa \\
Ámbito territorial & Estado de Yucatán (6,000 hectáreas SSPi, 120 UPP) \\
Categoría de plaza & Técnico especializado nivel TC-12 tabulador SADER \\
Salario anual & \$360,000 MXN (financiado vía FOFAY) \\
Especialización primaria & Sistemas silvopastoriles intensivos con \textit{Leucaena leucocephala} \\
\bottomrule
\end{tabular}
\end{table}

\subsection{Contexto Programático Específico}

El puesto constituye el núcleo técnico del Componente 1 del Macroproyecto "Renacimiento Ganadero Maya", responsable de la reconversión de 6,000 hectáreas hacia sistemas silvopastoriles intensivos mediante una inversión de \$171.0 millones MXN. La posición lidera técnicamente la implementación del paquete tecnológico validado por la Fundación Produce Michoacán (\$18,500/ha) y coordina el establecimiento de densidades científicamente validadas de \textit{Leucaena leucocephala} (40,000-53,000 plantas/ha) para maximizar la captura de carbono (765,000 ton CO$_2$eq) y incrementar la productividad ganadera (+280\%).

\section{Responsabilidades Principales}

\subsection{Diseño e Implementación de SSPi}

\textbf{1. Coordinación Técnica del Paquete SSPi:}
\begin{itemize}
    \item Supervisar implementación del paquete tecnológico \$18,500/ha en 120 UPP
    \item Diseñar layout específicos de pastoreo racional por predio
    \item Coordinar establecimiento de \textit{Leucaena leucocephala} (40,000-53,000 plantas/ha)
    \item Supervisar instalación de infraestructura (cercos eléctricos, bebederos)
    \item Validar densidades de siembra y espaciamientos óptimos por zona
    \item Monitorear establishment y supervivencia de leguminosas arbóreas
\end{itemize}

\textbf{2. Escuelas de Campo Silvopastoriles (ECAs):}
\begin{itemize}
    \item Coordinar 5 ECAs con 125 productores (25 por escuela)
    \item Desarrollar metodología participativa para transferencia SSPi
    \item Facilitar 25 sesiones/año por ECA (125 sesiones totales)
    \item Establecer UPP demostrativas como centros de aprendizaje
    \item Coordinar intercambios técnicos entre regiones
    \item Evaluar impacto de capacitación en adopción tecnológica
\end{itemize}

\subsection{Pastoreo Racional y Manejo Animal}

\textbf{3. Sistemas de Pastoreo Adaptativo:}
\begin{itemize}
    \item Diseñar rotaciones de pastoreo por capacidad de carga
    \item Determinar períodos de ocupación y descanso por potrero
    \item Supervisar carga animal óptima (2.5-3.5 UA/ha objetivo)
    \item Monitorear condición corporal y desempeño reproductivo
    \item Evaluar impacto del pastoreo en regeneración forrajera
    \item Ajustar sistemas según variabilidad climática estacional
\end{itemize}

\textbf{4. Monitoreo de Productividad:}
\begin{itemize}
    \item Medir disponibilidad y calidad de forraje mensualmente
    \item Registrar ganancia de peso y producción láctea por sistema
    \item Documentar incrementos de capacidad de carga anual
    \item Evaluar eficiencia reproductiva en sistemas SSPi
    \item Comparar indicadores vs sistemas tradicionales (baseline)
\end{itemize}

\subsection{Captura de Carbono y Servicios Ecosistémicos}

\textbf{5. Monitoreo Ambiental:}
\begin{itemize}
    \item Coordinar medición de captura de carbono (metodología IPCC)
    \item Supervisar establecimiento de parcelas permanentes de monitoreo
    \item Documentar incremento de biomasa aérea y subterránea
    \item Evaluar servicios ecosistémicos (biodiversidad, infiltración)
    \item Coordinar con especialista SIG para mapeo de coberturas
    \item Generar reportes para mercados internacionales de carbono
\end{itemize}

\section{Perfil del Puesto}

\subsection{Requisitos Académicos y Experiencia}

\begin{table}[H]
\centering
\caption{Requisitos del Zootecnista SSPi Senior}
\label{tab:requisitos}
\begin{tabular}{p{4cm}p{9cm}}
\toprule
\rowcolor{sadergreen!20}
\textbf{Requisito} & \textbf{Detalle} \\
\midrule
Formación académica & Zootecnista titulado con especialidad en producción animal o sistemas silvopastoriles. Maestría en agroecología o ganadería tropical deseable \\
Experiencia mínima & 7 años en ganadería tropical, 5 años en sistemas silvopastoriles o agroforestería pecuaria \\
Conocimientos indispensables & Leucaena leucocephala, pastoreo racional, leguminosas arbóreas, capacidad de carga, nutrición animal tropical, agroforestería pecuaria \\
Certificaciones deseables & Facilitador certificado en metodologías participativas (ECAs), captura de carbono \\
Idiomas & Inglés técnico básico (literatura científica) \\
Habilidades técnicas & Diseño de sistemas productivos, evaluación forrajera, facilitación grupal, SIG básico \\
Disponibilidad & 80\% trabajo de campo, movilidad estatal completa \\
\bottomrule
\end{tabular}
\end{table}

\subsection{Competencias Técnicas Específicas}

\textbf{Competencias obligatorias:}
\begin{itemize}
    \item Manejo de \textit{Leucaena leucocephala} var. Cunningham
    \item Diseño de sistemas de pastoreo racional
    \item Evaluación de forrajes tropicales
    \item Cálculo de capacidad de carga animal
    \item Facilitación de procesos participativos
    \item Evaluación de servicios ecosistémicos
\end{itemize}

\textbf{Competencias deseables:}
\begin{itemize}
    \item Experiencia con Fundación Produce Michoacán
    \item Metodologías de captura de carbono
    \item Manejo de leguminosas arbóreas nativas
    \item Sistemas de información geográfica
    \item Nutrición animal en sistemas agroforestales
\end{itemize}

\section{Indicadores de Desempeño}

\begin{table}[H]
\centering
\caption{Métricas del Zootecnista SSPi Senior}
\label{tab:kpis}
\begin{tabular}{p{6cm}p{3cm}p{3cm}}
\toprule
\rowcolor{sadergreen!20}
\textbf{Indicador} & \textbf{Meta} & \textbf{Frecuencia} \\
\midrule
Hectáreas SSPi establecidas & 1,200/año & Anual \\
UPP transformadas a SSPi & 24/año & Anual \\
Supervivencia Leucaena & $\geq$85\% & Semestral \\
Incremento capacidad de carga & +150\% & Anual \\
Sesiones ECA impartidas & 125/año & Anual \\
Productores capacitados & 125 & Anual \\
UPP demostrativas operando & 5 & Permanente \\
Captura CO$_2$ documentada & 127.5 ton/ha & Quinquenal \\
\bottomrule
\end{tabular}
\end{table}

\section{Metodología de Trabajo}

\subsection{Cronograma Anual de Actividades}

\textbf{Enero-Marzo (Temporada Seca):}
\begin{itemize}
    \item Preparación de terrenos para establecimiento SSPi
    \item Capacitación en ECAs (metodología y planificación)
    \item Evaluación de sistemas establecidos año anterior
    \item Planificación participativa con productores
\end{itemize}

\textbf{Abril-Junio (Pre-lluvias):}
\begin{itemize}
    \item Siembra de pasturas y establecimiento de Leucaena
    \item Instalación de infraestructura de pastoreo
    \item Seguimiento intensivo de establishment
    \item Capacitación en manejo de praderas jóvenes
\end{itemize}

\textbf{Julio-Septiembre (Lluvias):}
\begin{itemize}
    \item Monitoreo de crecimiento y desarrollo
    \item Ajustes en densidades y replantaciones
    \item Implementación de pastoreo inicial
    \item Evaluación de servicios ecosistémicos
\end{itemize}

\textbf{Octubre-Diciembre (Post-lluvias):}
\begin{itemize}
    \item Evaluación de productividad anual
    \item Planificación siguiente ciclo
    \item Intercambios técnicos entre productores
    \item Documentación de lecciones aprendidas
\end{itemize}

\section{Coordinación Técnica}

El Zootecnista SSPi Senior coordinará con:

\begin{itemize}
    \item \textbf{Especialista SIG/Carbono:} Mapeo y monitoreo satelital
    \item \textbf{Ingenieros Agrónomos:} Especies forrajeras y suelos
    \item \textbf{Zootecnista SSPi Junior:} Biofábricas y microorganismos
    \item \textbf{INIFAP:} Validación científica y protocolos
    \item \textbf{CICY:} Investigación en leguminosas nativas
    \item \textbf{Fundación Produce Michoacán:} Transferencia metodológica
\end{itemize}

\section{Condiciones Laborales}

\subsection{Condiciones Específicas}

\begin{itemize}
    \item \textbf{Base:} Mérida con oficina regional en Tizimín
    \item \textbf{Viajes:} 80\% tiempo en campo (rotación 120 UPP)
    \item \textbf{Vehículo:} 4×4 especializado para trabajo rural
    \item \textbf{Salario anual:} \$360,000 MXN + prestaciones
    \item \textbf{Horario:} Lunes a sábado, adaptado a ciclos productivos
    \item \textbf{Capacitación:} 40 horas anuales especializadas
\end{itemize}

\section{Conclusiones Técnicas}

El Zootecnista SSPi Senior constituye el núcleo técnico para la transformación ganadera hacia sistemas sostenibles y climáticamente inteligentes. Su expertise determina directamente el éxito del componente de mayor inversión (\$171.0 MDP) y mayor impacto ambiental del macroproyecto.

La coordinación de 6,000 hectáreas de reconversión y 120 UPP requiere competencias técnicas altamente especializadas en agroforestería pecuaria, combinadas con habilidades de facilitación y transferencia tecnológica horizontal.

\vspace{2cm}

\noindent\textbf{Jefe de Programa de Producción Pecuaria Sustentable}

\vspace{1cm}

\noindent\textbf{Subdelegado Agropecuario - SADER Yucatán}

\vspace{2cm}

\noindent\textit{Mérida, Yucatán, a 28 de noviembre de 2025}

\end{document}