\documentclass[12pt,letterpaper]{article}
\usepackage[utf8]{inputenc}
\usepackage[spanish]{babel}
\usepackage{geometry}
\usepackage{graphicx}
\usepackage{fancyhdr}
\usepackage{setspace}
\usepackage{lastpage}
\usepackage{parskip}
\usepackage{booktabs}
\usepackage{array}
\usepackage{multirow}
\usepackage{longtable}
\usepackage{float}
\usepackage{xcolor}
\usepackage{colortbl}
\usepackage{amsmath}
\usepackage{ragged2e}

% Colores SADER
\definecolor{saderblue}{RGB}{0,51,102}
\definecolor{sadergreen}{RGB}{34,139,34}
\definecolor{sadergray}{RGB}{128,128,128}
\definecolor{sadergold}{RGB}{255,215,0}

% Márgenes exactos SADER
\geometry{top=2.5cm,bottom=2.5cm,left=3cm,right=3cm,headheight=20pt}

% Encabezado y pie de página
\pagestyle{fancy}
\fancyhf{}
\rfoot{\thepage}
\renewcommand{\headrulewidth}{0pt}
\fancyhead[L]{\includegraphics[width=2.8cm]{logo_sader.png}}

\begin{document}

% ========================================
% PORTADA OFICIAL
% ========================================
\begin{titlepage}
    \centering
    \vspace*{0.3cm}
    \includegraphics[width=0.25\textwidth]{logo_sader.png}\\[0.8cm]
    
    \vspace{0.4cm}
    {\normalsize\bfseries Perfil Profesional y Descripción Técnica:\par}
    \vspace{0.6cm}
    
    {\LARGE\bfseries ZOOTECNISTA JUNIOR\par}
    {\LARGE\bfseries SISTEMAS SILVOPASTORILES INTENSIVOS\par}
    \vspace{0.5cm}
    {\Large Especialista en Biofábricas y Especies Forrajeras:\par}
    \vspace{0.3cm}
    {\normalsize • Biofábricas de Microorganismos Benéficos\par}
    {\normalsize • Leguminosas Arbóreas Nativas de Yucatán\par}
    {\normalsize • Gramíneas Tropicales Mejoradas\par}
    {\normalsize • Bioinoculantes y Bioestimulantes\par}
    {\normalsize • Propagación Vegetativa y Semillas\par}
    \vspace{0.5cm}
    {\Large Macroproyecto Renacimiento Ganadero Maya\par}
    {\Large Yucatán 2026-2030\par}
    
    \vfill
    
    {\normalsize Mérida, Yucatán, 28 de noviembre de 2025\par}
    \vspace{0.2cm}
    {\normalsize SADER REPRESENTACIÓN ESTATAL YUCATÁN\par}
    {\normalsize Subdelegación Agropecuaria\par}
    \vspace{0.2cm}
    {\normalsize Código: ZOO-SSPi-JR-004\par}
    {\normalsize Secretaría de Agricultura y Desarrollo Rural (SADER)\par}
\end{titlepage}

% ========================================
% ÍNDICE AUTOMÁTICO
% ========================================
\clearpage
\thispagestyle{empty}
\vspace*{3cm}
{\large\bfseries Contenido}\\[2cm]

\tableofcontents

\clearpage
\setcounter{page}{3}

% ========================================
% CONTENIDO
% ========================================

\section{Caracterización Institucional del Puesto}

\justifying

\subsection{Identificación Administrativa}

\begin{table}[H]
\centering
\caption{Datos Generales del Zootecnista SSPi Junior}
\label{tab:datos_generales}
\begin{tabular}{p{5cm}p{8cm}}
\toprule
\rowcolor{sadergreen!20}
\textbf{Campo} & \textbf{Especificación Técnica} \\
\midrule
Denominación oficial & Zootecnista Junior - Especialista en Biofábricas SSPi \\
Código de identificación & ZOO-SSPi-JR-004 \\
Dependencia jerárquica & Zootecnista Senior SSPi \\
Línea de reporte directo & Coordinación técnica con Zootecnista SSPi Senior \\
Ámbito territorial & Estado de Yucatán (120 UPP, 5 biofábricas regionales) \\
Categoría de plaza & Técnico especializado nivel TC-10 tabulador SADER \\
Salario anual & \$360,000 MXN (financiado vía FOFAY) \\
Especialización primaria & Biofábricas de microorganismos y propagación forrajera \\
\bottomrule
\end{tabular}
\end{table}

\subsection{Contexto Programático Específico}

Este puesto especializado complementa técnicamente la implementación del Componente 1 (SSPi) mediante el desarrollo y operación de biofábricas regionales para la producción masiva de microorganismos benéficos, bioinoculantes y material vegetativo de especies forrajeras. La posición garantiza la disponibilidad y calidad de insumos biológicos críticos para el éxito de las 6,000 hectáreas de reconversión silvopastoril, coordinando la red de 5 biofábricas con capacidad de atender 120 UPP y generar 2,000 litros de bioinoculantes mensuales por instalación.

\section{Responsabilidades Principales}

\subsection{Biofábricas de Microorganismos Benéficos}

\textbf{1. Desarrollo y Operación de Biofábricas:}
\begin{itemize}
    \item Establecer 5 biofábricas regionales (capacidad 2,000 L/mes c/u)
    \item Coordinar producción de microorganismos nativos del suelo
    \item Supervisar multiplicación de \textit{Rhizobium} específico para Leucaena
    \item Producir hongos micorrízicos arbusculares (HMA)
    \item Generar bioestimulantes a base de microalgas
    \item Mantener cepas puras en condiciones de laboratorio
    \item Implementar controles de calidad microbiológicos
\end{itemize}

\textbf{2. Bioinoculantes y Bioestimulantes:}
\begin{itemize}
    \item Formular bioinoculantes específicos por especie forrajera
    \item Desarrollar bioestimulantes para establishment vegetal
    \item Producir activadores de compostaje y descomposición
    \item Generar probióticos para nutrición animal
    \item Estandarizar protocolos de aplicación por cultivo
    \item Documentar eficacia comparativa vs químicos
\end{itemize}

\subsection{Propagación de Especies Forrajeras}

\textbf{3. Material Vegetativo de Leucaena:}
\begin{itemize}
    \item Establecer jardines clonales de \textit{Leucaena leucocephala}
    \item Producir 500,000 plántulas anuales (var. Cunningham)
    \item Coordinar propagación vegetativa por estacas
    \item Supervisar viveros regionales de especies arbóreas
    \item Implementar protocolos de aclimatación y rustificación
    \item Garantizar calidad genética del material
\end{itemize}

\textbf{4. Gramíneas Tropicales Especializadas:}
\begin{itemize}
    \item Multiplicar semillas de \textit{Megathyrsus maximus} cv. Tanzania
    \item Producir material de \textit{Cynodon nlemfuensis} (Estrella africana)
    \item Propagar \textit{Brachiaria brizantha} cv. Marandú
    \item Establecer parcelas semilleras certificadas
    \item Coordinar cosecha y procesamiento de semillas
    \item Mantener pureza genética y viabilidad
\end{itemize}

\subsection{Leguminosas Arbóreas Nativas}

\textbf{5. Especies Nativas de Yucatán:}
\begin{itemize}
    \item Investigar y propagar \textit{Lysiloma latisiliquum} (Tzalam)
    \item Desarrollar protocolos para \textit{Piscidia piscipula} (Jabín)
    \item Multiplicar \textit{Gliricidia sepium} (Cocoíte) mejorado
    \item Evaluar \textit{Diphysa carthaginensis} (Ts'uts'uy) forrajera
    \item Documentar potencial nutritivo por especie
    \item Establecer bancos de germoplasma regional
\end{itemize}

\section{Perfil del Puesto}

\subsection{Requisitos Académicos y Experiencia}

\begin{table}[H]
\centering
\caption{Requisitos del Zootecnista SSPi Junior}
\label{tab:requisitos}
\begin{tabular}{p{4cm}p{9cm}}
\toprule
\rowcolor{sadergreen!20}
\textbf{Requisito} & \textbf{Detalle} \\
\midrule
Formación académica & Zootecnista con especialidad en nutrición animal o producción forrajera. Especialización en microbiología aplicada o biotecnología deseable \\
Experiencia mínima & 3 años en producción forrajera, 2 años en microbiología aplicada o biofábricas \\
Conocimientos indispensables & Microbiología del suelo, propagación vegetativa, semillas forrajeras, bioinoculantes, control de calidad microbiológico \\
Certificaciones deseables & Manejo de laboratorio microbiológico, producción de biofertilizantes, viveros forestales \\
Idiomas & Inglés técnico básico (protocolos microbiológicos) \\
Habilidades técnicas & Técnicas de laboratorio, propagación de plantas, manejo de viveros, microscopia \\
Disponibilidad & 60\% laboratorio/biofábrica, 40\% campo \\
\bottomrule
\end{tabular}
\end{table}

\subsection{Competencias Técnicas Específicas}

\textbf{Competencias obligatorias:}
\begin{itemize}
    \item Microbiología aplicada a la agricultura
    \item Técnicas de propagación vegetativa
    \item Producción y multiplicación de semillas
    \item Control de calidad microbiológico
    \item Manejo de cepas y cultivos puros
    \item Formulación de bioinoculantes
\end{itemize}

\textbf{Competencias deseables:}
\begin{itemize}
    \item Biotecnología vegetal
    \item Hongos micorrízicos arbusculares
    \item Especies forestales tropicales
    \item Técnicas de conservación de germoplasma
    \item Biología molecular básica
\end{itemize}

\section{Indicadores de Desempeño}

\begin{table}[H]
\centering
\caption{Métricas del Zootecnista SSPi Junior}
\label{tab:kpis}
\begin{tabular}{p{6cm}p{3cm}p{3cm}}
\toprule
\rowcolor{sadergreen!20}
\textbf{Indicador} & \textbf{Meta} & \textbf{Frecuencia} \\
\midrule
Biofábricas operando & 5 & Permanente \\
Plántulas Leucaena producidas & 500,000/año & Anual \\
Litros bioinoculantes/mes & 10,000 L & Mensual \\
Kg semilla forrajera/año & 2,500 kg & Anual \\
UPP atendidas con bioinsumos & 120 & Anual \\
Cepas activas mantenidas & $\geq$15 & Permanente \\
Eficacia bioinoculantes & $\geq$90\% & Semestral \\
Especies nativas propagadas & 4 especies & Anual \\
\bottomrule
\end{tabular}
\end{table}

\section{Metodología de Trabajo}

\subsection{Protocolo de Biofábricas}

\textbf{Fase 1: Aislamiento y Caracterización}
\begin{itemize}
    \item Colecta de muestras de suelo de sistemas nativos exitosos
    \item Aislamiento de microorganismos benéficos
    \item Caracterización morfológica y funcional
    \item Pruebas de compatibilidad y antagonismo
    \item Selección de cepas eficientes por especie
\end{itemize}

\textbf{Fase 2: Multiplicación Masiva}
\begin{itemize}
    \item Establecimiento de cultivos madre
    \item Escalamiento en medios líquidos y sólidos
    \item Control de pureza y viabilidad
    \item Formulación en sustratos comerciales
    \item Empaque y etiquetado especializado
\end{itemize}

\subsection{Cronograma Anual de Producción}

\textbf{Enero-Marzo:}
\begin{itemize}
    \item Preparación de medios de cultivo
    \item Reactivación de cepas conservadas
    \item Producción intensiva para temporada de siembra
    \item Multiplicación de material vegetativo
\end{itemize}

\textbf{Abril-Junio:}
\begin{itemize}
    \item Distribución masiva de bioinsumos
    \item Asesoría técnica en aplicación
    \item Seguimiento de establishment
    \item Evaluación de eficacia en campo
\end{itemize}

\textbf{Julio-Septiembre:}
\begin{itemize}
    \item Colecta de nuevas cepas nativas
    \item Mantenimiento de jardines clonales
    \item Investigación en especies nativas
    \item Desarrollo de nuevos productos
\end{itemize}

\textbf{Octubre-Diciembre:}
\begin{itemize}
    \item Cosecha de semillas forrajeras
    \item Procesamiento y almacenamiento
    \item Evaluación anual de resultados
    \item Planificación siguiente ciclo
\end{itemize}

\section{Coordinación Técnica}

El Zootecnista SSPi Junior coordinará con:

\begin{itemize}
    \item \textbf{Zootecnista SSPi Senior:} Necesidades de campo y especies prioritarias
    \item \textbf{Ingenieros Agrónomos:} Requerimientos específicos por suelo
    \item \textbf{CICY:} Investigación en microorganismos nativos
    \item \textbf{INIFAP:} Validación de protocolos microbiológicos
    \item \textbf{Tecnológico de Mérida:} Desarrollo biotecnológico
    \item \textbf{Productores:} Capacitación en uso de bioinsumos
\end{itemize}

\section{Infraestructura Requerida}

\subsection{Equipamiento de Biofábricas}

\begin{itemize}
    \item \textbf{Laboratorio microbiológico:} Autoclave, campana de flujo laminar, incubadoras, microscopia
    \item \textbf{Área de producción:} Fermentadores, agitadores, sistemas de aireación
    \item \textbf{Viveros especializados:} Invernaderos, sistemas de riego, cámaras de germinación
    \item \textbf{Almacenamiento:} Cámaras frías, deshumidificadores, sistemas de conservación
    \item \textbf{Transporte:} Vehículos refrigerados, contenedores especializados
\end{itemize}

\section{Condiciones Laborales}

\subsection{Condiciones Específicas}

\begin{itemize}
    \item \textbf{Base:} Laboratorio central Mérida + 4 biofábricas regionales
    \item \textbf{Horario:} Lunes a sábado, turnos rotativos para mantenimiento
    \item \textbf{Salario anual:} \$360,000 MXN + prestaciones especializadas
    \item \textbf{Capacitación:} 50 horas anuales en biotecnología aplicada
    \item \textbf{Seguridad:} Protocolos de bioseguridad nivel 2
\end{itemize}

\section{Conclusiones Técnicas}

El Zootecnista SSPi Junior garantiza la sostenibilidad técnica y económica del componente SSPi mediante la autosuficiencia en insumos biológicos especializados. Su expertise determina la calidad del establishment forrajero y la eficiencia de los procesos de inoculación que optimizan la captura de carbono y productividad del sistema.

La red de biofábricas constituye infraestructura estratégica para la independencia tecnológica del programa y la reducción de costos de insumos externos.

\vspace{2cm}

\noindent\textbf{Zootecnista Senior - Sistemas Silvopastoriles Intensivos}

\vspace{1cm}

\noindent\textbf{Jefe de Programa de Producción Pecuaria Sustentable}

\vspace{2cm}

\noindent\textit{Mérida, Yucatán, a 28 de noviembre de 2025}

\end{document}