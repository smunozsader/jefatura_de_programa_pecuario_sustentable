\documentclass[12pt,letterpaper]{article}
\usepackage[utf8]{inputenc}
\usepackage[spanish]{babel}
\usepackage{geometry}
\usepackage{graphicx}
\usepackage{fancyhdr}
\usepackage{setspace}
\usepackage{lastpage}
\usepackage{parskip}
\usepackage{xcolor}
\usepackage{url}
\usepackage{csquotes}

% Configuración de página
\geometry{top=2.5cm,bottom=2.5cm,left=3cm,right=3cm,headheight=20pt}

% Colores para resaltar
\definecolor{titlecolor}{RGB}{0,51,102}
\definecolor{subtitlecolor}{RGB}{128,128,128}
\definecolor{quotecolor}{RGB}{34,139,34}

% Encabezado y pie de página
\pagestyle{fancy}
\fancyhf{}
\rfoot{\thepage}
\renewcommand{\headrulewidth}{0pt}
\fancyhead[L]{\small\textit{Editorial - Diario de Yucatán}}

\begin{document}

% ========================================
% PORTADA DEL ARTÍCULO
% ========================================
\thispagestyle{empty}
\begin{center}
    \vspace*{2cm}
    
    {\Huge\bfseries\color{titlecolor} Renacimiento Ganadero, Mito}
    
    \vspace{1cm}
    
    {\large\color{subtitlecolor} Editorial - Diario de Yucatán}
    
    \vspace{0.5cm}
    
    {\large\bfseries Dr. Juan Ku Vera}
    
    \vspace{0.3cm}
    
    {\normalsize Profesor-Investigador}\\
    {\normalsize Laboratorio de Cambio Climático y Ganadería}\\
    {\normalsize Facultad de Medicina Veterinaria y Zootecnia - UADY}
    
    \vspace{0.5cm}
    
    {\normalsize 21 de noviembre de 2025, 2:00 am}
    
    \vspace{1cm}
    
    \rule{\textwidth}{0.4pt}
    
\end{center}

\clearpage
\setcounter{page}{1}

% ========================================
% CONTENIDO DEL ARTÍCULO
% ========================================

\section*{El Mito de la Excelencia Genética}

La producción bovina yucateca se ha desarrollado durante décadas con base en el mito de la excelencia genética del hato ganadero. Desde hace más de cincuenta años, decenas de vacas y toros son exhibidos anualmente en la Feria de Xmatkuil como muestra de la aclamada alta calidad genética de la ganadería, mientras que la producción de carne por hectárea no se ha incrementado.

En realidad, lo que se expone en ese popular jolgorio son vacas y toros con una apariencia portentosa, esto es, la expresión del fenotipo de una raza en particular o de un cruzamiento, pero no el progreso genético del hato bovino en pastoreo en cuanto a alguna característica productiva de interés comercial.

En el contexto nacional, Yucatán no destaca como un productor importante de carne y mucho menos de leche bovina. ¿Por qué una de las primeras acciones en el marco del renacimiento ganadero fue la importación de los Estados Unidos de 100,000 dosis de semen de toros Brahman para inseminar a las vacas de Yucatán? No parece tener sentido ¿no es así?

\section*{La Falta de Claridad Estratégica}

Resulta entonces aparente que después de un año en funciones, la administración agropecuaria estatal no tiene claro cómo convertir sus rimbombantes declaraciones, en prácticas viables en ranchos y establos con el fin de redireccionar la actividad ganadera para alcanzar mejores índices productivos.

La globalización de la economía influye para que en algunos supermercados de Mérida, se expenda carne bovina importada (cortes: rib-eye, T-bone, Tomahawk) de los Estados Unidos y Canadá, así como de algunos estados de la república; mientras que los resultados de las investigaciones de los genetistas cuantitativos permanecen en cajones empolvándose sin haber sido puestos en práctica en las unidades de producción.

Es en este sentido que la actual administración agropecuaria insiste de manera reiterada en promover un mejoramiento genético de tipo mendeliano (adquisición de toros de buen porte con un subsidio estatal), repoblación del hato bovino: \$9,000 por cada vaquilla adquirida y \$5,000 si la novillona permanece cinco años en el rancho (o sea: dinero a fondo perdido), reforzando así el mito del renacimiento ganadero progresista.

\section*{La Repoblación: Estrategia Sin Fundamento Zootécnico}

La repoblación del hato bovino como estrategia de desarrollo ganadero es un concepto sin bases en la zootecnia contemporánea. Por ejemplo, es conocido que en los Estados Unidos, en los últimos sesenta años el inventario bovino se redujo gradualmente, mientras que la eficiencia de la producción de carne y leche se incrementó en ese periodo con la implementación de tecnologías innovadoras.

En la pasada administración estatal se inauguró (noviembre de 2023) el Centro Integral de Mejoramiento Genético en Tizimín con una inversión de 44 millones de pesos, un elefante blanco más del cual no se tienen noticias; mientras que aún no se conocen el porcentaje de heredabilidad o el efecto de la heterosis de caracteres importantes (tasa de crecimiento, fertilidad, calidad de la carne) de las razas bovinas predominantes en Yucatán.

\section*{Tecnologías Modernas: El Camino No Tomado}

Las tecnologías modernas como la de GrowSafe para identificar genotipos sobresalientes en cuanto a variables tales como el consumo de alimento y la eficiencia de conversión alimenticia; o la de GreenFeed para seleccionar bovinos con bajas emisiones de gas metano, son desconocidas en Yucatán, metodologías que son útiles para seleccionar razas o cruzamientos superiores para dichas características de interés económico y ambiental respectivamente.

\section*{El Fracaso Anunciado}

El renacimiento ganadero, no existe, es un mito difundido para perpetuar el status quo, esto es, para mantener a cientos de pequeños ganaderos distraídos con exiguos apoyos (paneles solares, fertilizantes). El renacimiento ganadero en su modalidad actual fracasará como ocurrió en el sexenio pasado con el programa de \enquote{Crédito Ganadero a la Palabra}, porque están basados en objetivos políticos (clientelismo), más que en los principios de la zootecnia.

El renacimiento ganadero no se materializará si no se apoya con determinación la investigación en zootecnia por parte de la administración estatal, misma que pregona la supremacía de la genética bovina yucateca.

\section*{El Futuro de la Genética Bovina}

Yucatán aún está lejos del empleo de técnicas avanzadas en genética molecular tales como la secuenciación de nueva generación para identificar polimorfismos y características deseables en el ganado, así como la edición genética. En este sentido, el mejoramiento basado en diversas ómicas (genómica, transcriptómica y proteómica) y en la epigenética podría dar lugar a programas de cría que prioricen una baja huella ambiental de la ganadería.

La única forma de demostrar de manera concluyente el progreso genético en el ganado es logrando una alta retención de la energía consumida del pasto como proteína y grasa en la canal bovina al sacrificio. No existe evidencia de que esta variable se haya incrementado en el ganado bovino mantenido en pastoreo en los últimos cincuenta años en Yucatán.

\section*{La Fábula de la Vaca Presumida}

El renacimiento ganadero resulta ser entonces un concepto ficticio, reminiscente de la fábula de la vaca presumida que señala a una vaca vanidosa a la cual no le preocupa que su ubre produzca abundante leche, sino que únicamente desea verse atractiva (el fenotipo), mientras ésta alardea de ser muy productiva (el genotipo).

La ganadería bovina yucateca exalta su alta calidad genética, pero aún no se demuestra un cambio positivo en el mérito genético del hato bovino en pastoreo para algún rasgo indicativo de una rentabilidad económica superior como podrían ser la tasa de crecimiento o la calidad de la carne (terneza, marmoleo).

\section*{Conclusión}

Se puede decir entonces que, a más de un año del arranque de la actual administración agropecuaria, no es posible anticipar cuándo se iniciará el verdadero renacimiento ganadero de Yucatán.

\vspace{1cm}

\begin{flushright}
\textit{Mérida, Yucatán}\\
\textbf{Dr. Juan Ku Vera}\\
kvera@correo.uady.mx\\
Profesor-Investigador\\
Laboratorio de Cambio Climático y Ganadería\\
Facultad de Medicina Veterinaria y Zootecnia - UADY
\end{flushright}

\vspace{0.5cm}

\hrule

\vspace{0.3cm}

\begin{center}
\small\textbf{Fuente:} \url{https://www.yucatan.com.mx/editorial/2025/11/21/juan-ku-vera-renacimiento-ganadero-mito.html}
\end{center}

\end{document}