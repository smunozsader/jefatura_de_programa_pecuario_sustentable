\documentclass[11pt,a4paper]{article}
\usepackage[utf8]{inputenc}
\usepackage[spanish]{babel}
\usepackage{booktabs}
\usepackage{array}
\usepackage{caption}
\usepackage{geometry}
\geometry{margin=2.5cm}
\usepackage{siunitx}
\sisetup{output-decimal-marker={,}}

\title{Densidad de siembra de \textit{Leucaena leucocephala} en sistemas silvopastoriles intensivos (SSPi)\\
promovidos por Fundación Produce Michoacán A.C.\\
(Región Tierra Caliente: Apatzingán y Tepalcatepec, 2010-2020)}
\author{Equipo multidisciplinario en desarrollo agropecuario sustentable}
\date{Noviembre 2025}

\begin{document}

\maketitle

\section*{Resumen técnico}
La Fundación Produce Michoacán A.C. impulsó entre 2010 y 2020 la adopción de sistemas silvopastoriles intensivos (SSPi) con \textit{Leucaena leucocephala} en la región Tierra Caliente (Apatzingán y Tepalcatepec). Estos sistemas se caracterizan por densidades superiores a \textbf{10\,000 plantas/ha}, llegando hasta \textbf{80\,000 plantas/ha}, con el objetivo de aumentar la productividad ganadera, fijar nitrógeno (hasta \SI{550}{\kilo\gram\per\hectare\per\year}), secuestrar carbono y cumplir con los principios de la Ley de Desarrollo Rural Sustentable y los requisitos de trazabilidad (SINIIGA/SINIDA y T-MEC).

\section*{Densidades de siembra verificadas (2010-2020)}

\begin{table}[htbp]
\centering
\caption{Densidades de \textit{Leucaena leucocephala} utilizadas en SSPi de Fundación Produce Michoacán}
\begin{tabular}{@{}l c c c >{\raggedright\arraybackslash}p{6cm}@{}}
\toprule
\textbf{Densidad (plantas/ha)} & 
\textbf{Distancia entre surcos (m)} & 
\textbf{Espaciamiento en surco (m)} & 
\textbf{Semillas/m lineal} & 
\textbf{Contexto y referencia principal} \\
\midrule
34\,500 & 1,6 & 0,20--0,30 & 15--20 & Pastoreo rotacional en seca con \textit{Panicum maximum}; Tepalcatepec (Bacab y Solorio, 2011) \\
53\,000 & 1,2--1,5 & 0,20 & 18--22 & Máxima fijación de N (\SIrange{52}{320}{\kilo\gram\per\hectare}); Tepalcatepec (Fundación, 2012) \\
40\,000--60\,000 & 2,4 & 0,25--0,30 & 15--18 & Carga animal 4--5 UA/ha; ganancia de peso 200 kg destete/becerro; Apatzingán-Tepalcatepec (Flores y Solorio, 2012) \\
80\,000 & 1,6 & 0,15--0,20 & 20--25 & Máxima captura de C (\SI{128}{\tonne\per\hectare\per\year}); poda intensiva (Anguiano et al., 2013) \\
\bottomrule
\end{tabular}
\end{table}

\section*{Recomendaciones técnicas complementarias}
\begin{itemize}
    \item \textbf{Semilla}: \SIrange{8}{16}{\kilo\gram\per\hectare} (18\,000 semillas/kg, germinación 80--90\,\%).
    \item \textbf{Inoculación}: \textit{Rhizobium} específico + micorrizas.
    \item \textbf{Fertilización inicial}: \SI{700}{\kilo\gram\per\hectare} de tierra de diatomeas (control de psílido).
    \item \textbf{Poda}: Cada 6--8 semanas a \SIrange{20}{40}{\centi\metre} de altura.
    \item \textbf{Variedad recomendada}: Cunningham (alta tolerancia a sequía y psílido).
    \item \textbf{Costo de establecimiento}: \textbf{MXN 25\,000--30\,000/ha} (2010--2020).
\end{itemize}

\section*{Beneficios cuantificados}
\begin{itemize}
    \item Oferta forraje en época seca: \SIrange{2470}{2693}{\kilo\gram\ MS\per\hectare\per\pastoreo} (vs. \SI{948}{\kilo\gram} en sistemas tradicionales).
    \item Producción de leche: \SI{7}{\litre\per\vaca\per\day} $\to$ \textbf{10\,500 L/ha/año}.
    \item Carga animal: hasta \SI{5}{UA/ha}.
    \item Reducción de emisiones de GEI: hasta 50\,\% vs. monocultivo de pasto.
\end{itemize}

\section*{Alineación normativa}
Los SSPi facilitan la trazabilidad individual de bovinos bajo \textbf{SINIIGA/SINIDA} y fortalecen los compromisos México-Estados Unidos en el marco del \textbf{T-MEC} (Plan estratégico tuberculosis bovina).

\end{document}