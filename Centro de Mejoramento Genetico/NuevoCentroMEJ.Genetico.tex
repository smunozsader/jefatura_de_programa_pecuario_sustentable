\documentclass[12pt,letterpaper]{article}
\usepackage[utf8]{inputenc}
\usepackage[spanish]{babel}
\usepackage{geometry}
\usepackage{graphicx}
\usepackage{fancyhdr}
\usepackage{setspace}
\usepackage{lastpage}
\usepackage{parskip}
\usepackage{booktabs}
\usepackage{array}
\usepackage{multirow}
\usepackage{longtable}
\usepackage{float}

% Márgenes exactos SADER
\geometry{top=2.5cm,bottom=2.5cm,left=3cm,right=3cm,headheight=20pt}

% Encabezado y pie de página
\pagestyle{fancy}
\fancyhf{}
\rfoot{\thepage}
\renewcommand{\headrulewidth}{0pt}
\fancyhead[L]{\includegraphics[width=2.8cm]{logo_sader.png}}



\begin{document}

% ========================================
% PORTADA OFICIAL
% ========================================
\begin{titlepage}
    \centering
    \vspace*{1cm}
    \includegraphics[width=0.28\textwidth]{logo_sader.png}\\[2.5cm]
    
    \vspace{1.5cm}
    {\large\bfseries Proyecto Estratégico:\par}
    \vspace{2.5cm}
    
    {\Huge\bfseries Refundación, Certificación OIE/ISO-17025\par}
    {\Huge\bfseries y Relanzamiento del Centro Regional de\par}
    {\Huge\bfseries Mejoramiento Genético Bovino de Tizimín\par}
    \vspace{1.5cm}
    {\LARGE Yucatán 2026-2030\par}
    
    \vfill
    
    {\large Mérida, Yucatán, 19 de noviembre de 2025\par}
    \vspace{0.5cm}
    {\large MVZ SERGIO MUÑOZ DE ALBA MEDRANO\par}
    {\large Jefe de Programa de Producción Pecuaria Sustentable\par}
    {\large Representación Estatal SADER Yucatán\par}
\end{titlepage}

% ========================================
% ÍNDICE AUTOMÁTICO
% ========================================
\clearpage
\thispagestyle{empty}
\vspace*{3cm}
{\large\bfseries Contenido}\\[2cm]

\tableofcontents

\clearpage
\setcounter{page}{3}

% ========================================
% CONTENIDO
% ========================================
\section{Resumen Ejecutivo}

El Centro Regional de Mejoramiento Genético Bovino de Tizimín será refundado y certificado bajo estándares OIE e ISO/IEC 17025:2017, convirtiéndose en el primer laboratorio de reproducción bovina con validez internacional del sureste mexicano.

\textbf{Meta 2030:} 120 000 dosis de semen + 5 000 embriones/año certificados.  
\textbf{Inversión total:} \$450 000 000 MXN (60 \% federal – 40 \% estatal + productores).

\section{Introducción}

El Centro de Tizimín, construido en 1986, opera al 18 \% de su capacidad por falta de certificación oficial. Esta iniciativa lo convertirá en referente regional de genética tropical.

\section{Justificación}

\begin{itemize}
    \item Yucatán importa >70 \% del semen bovino
    \item Necesidad de genética certificada para repoblamiento masivo
    \item Alineación con Directriz 4.1.1 del Plan Estatal Renacimiento Maya
\end{itemize}

\section{Objetivo General}

Convertir el Centro de Tizimín en laboratorio líder de reproducción bovina tropical certificada antes de 2028.

\section{Objetivos Específicos}

\begin{enumerate}
    \item Acreditación ISO-17025 por EMA (2027)
    \item Aprobación OIE por SENASICA-CENAPA (2028)
    \item Producción plena 120 000 dosis/año (2030)
    \item Convenios con ABS Global, Alta Genetics y Semex
\end{enumerate}

\section{Diagnóstico Actual}

\begin{itemize}
    \item Infraestructura aprovechable: 80 \%
    \item Producción actual: 10 000 dosis/año sin certificación
    \item Personal: 4 MVZ (requiere capacitación)
    \item Equipamiento obsoleto
\end{itemize}

\section{Certificación OIE/ISO-17025}

\begin{itemize}
    \item \textbf{ISO-17025}: EMA (Entidad Mexicana de Acreditación)
    \item \textbf{OIE}: SENASICA-CENAPA
    \item Duración: 24–36 meses
\end{itemize}

\section{Componentes de Apoyo}

\begin{enumerate}
    \item Área limpia clase 10 000
    \item Congeladora programable y equipo de punta
    \item Capacitación internacional
    \item Sistema de Gestión de Calidad (40–60 POES)
    \item Trazabilidad SINIIGA por pajuela
\end{enumerate}

\section{Memoria de Cálculo}

Inversión total: \$450 MDP (2026-2030)

\section{Tabla de Costos Centralizada}

\begin{table}[H]
\centering
\begin{tabular}{|l|r|r|r|}
\hline
\textbf{Rubro} & \textbf{Total} & \textbf{Federal 60\%} & \textbf{Estatal 40\%} \\
\hline
Remodelación & 180 & 108 & 72 \\
Equipamiento & 140 & 84 & 56 \\
Capacitación & 60 & 36 & 24 \\
Certificación & 20 & 12 & 8 \\
Operación & 50 & 30 & 20 \\
\hline
\textbf{TOTAL} & \textbf{450} & \textbf{270} & \textbf{180} \\
\hline
\end{tabular}
\end{table}

% ========================================
% NUEVA SECCIÓN: MEMORIA DETALLADA
% ========================================
\section{Memoria de Cálculo Detallada con Fuentes Reales}

\begin{longtable}{|p{5cm}|p{2.5cm}|p{7cm}|}
\hline
\textbf{Rubro} & \textbf{Monto (MDP)} & \textbf{Fuente y justificación} \\
\hline
\endfirsthead
\hline
\textbf{Rubro} & \textbf{Monto} & \textbf{Fuente} \\
\hline
\endhead
Remodelación y área limpia clase 10 000 & 180 & Cotización 2024 Constructora GAMI + CNRG Jalisco 2021: \$195 millones ajustado inflación \\
\hline
Equipamiento de laboratorio & 140 & ABS Global México 2025: congeladora \$48M + CASA \$22M + equipo completo \$70M \\
\hline
Capacitación y consultoría internacional & 60 & Embrapa Brasil 2023: \$48M (24 meses) + capacitación 6 técnicos Colombia \$12M \\
\hline
Auditorías EMA + SENASICA + OIE & 20 & Laboratorio Liconsa Guanajuato 2024: \$18.5M + viajes auditores \\
\hline
Operación y mantenimiento 5 años & 50 & Estimación \$10M/año (energía, nitrógeno, reactivos) \\
\hline
\textbf{TOTAL} & \textbf{450} & Coincide con componente 5 del Macroproyecto julio 2025 \\
\hline
\end{longtable}

\section{Impacto Esperado}

\begin{itemize}
    \item 120 000 dosis/año certificadas
    \item Ingresos anuales: \$80–100 MDP
    \item Sustitución de importaciones
    \item Exportación a Centroamérica
\end{itemize}

\section{Alineación con Directriz 4.1.1}

Cumple con líneas 4.1.1.1.6 y 4.1.1.5.3.

\section{Cronograma Tentativo (2026-2030)}

\begin{itemize}
    \item 2026: Diagnóstico y licitación
    \item 2027: Remodelación y capacitación
    \item 2028: Certificación OIE/ISO-17025
    \item 2030: Producción plena
\end{itemize}

\section{Conclusiones}

La certificación del Centro de Tizimín es la inversión estratégica que convertirá a Yucatán en líder regional de genética bovina tropical.

\section{Bibliografía}

\begin{itemize}
    \item OIE Terrestrial Animal Health Code, Capítulo 4.9
    \item ISO/IEC 17025:2017
    \item SENASICA – Requisitos centros IA
    \item EMA – Guía de acreditación
    \item ABS Global México – Cotización 2025
\end{itemize}

\end{document}