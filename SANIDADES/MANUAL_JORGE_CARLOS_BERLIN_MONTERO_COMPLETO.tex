\documentclass[11pt,a4paper]{article}
\usepackage[utf8]{inputenc}
\usepackage[spanish]{babel}
\usepackage[letterpaper,top=2.5cm,bottom=3cm,left=2.5cm,right=2.5cm,headheight=40pt,headsep=0.8cm,footskip=1.2cm]{geometry}
\usepackage{graphicx}
\usepackage{fancyhdr}
\usepackage{xcolor}
\usepackage{titlesec}
\usepackage{enumitem}
\usepackage{array}
\usepackage{longtable}
\usepackage{booktabs}
\usepackage{multicol}
\usepackage{microtype}
\usepackage{hyperref}
\usepackage{lastpage}
\usepackage{tikz}
\usepackage{tcolorbox}
\usepackage{setspace}
\usepackage{caption}

% ==========================================
% COLORES OFICIALES GOB.MX
% ==========================================
\definecolor{gobmx-burgundy}{RGB}{139,21,56}
\definecolor{gobmx-gray}{RGB}{84,84,84}
\definecolor{gobmx-light}{RGB}{221,201,163}
\definecolor{gobmx-gold}{RGB}{188,149,92}
\definecolor{gobmx-dark}{RGB}{19,50,46}

% ==========================================
% CONFIGURACIÓN DE HYPERREF
% ==========================================
\hypersetup{
    pdftitle={Manual SADER - MVZ Jorge Carlos Berlín Montero},
    pdfauthor={SADER Yucatan},
    pdfsubject={Manual de Usuario Centro de Consulta de Acuerdos Zoosanitarios},
    pdfkeywords={SADER, Presidente, CESO, Acuerdos, Sanitarios},
    colorlinks=true,
    linkcolor=gobmx-burgundy,
    citecolor=gobmx-burgundy,
    filecolor=gobmx-burgundy,
    urlcolor=gobmx-burgundy
}

% ==========================================
% CONFIGURACIÓN DE HEADERS Y FOOTERS
% ==========================================
\pagestyle{fancy}
\fancyhf{}

% Header izquierdo con título del manual
\fancyhead[L]{
    \footnotesize\color{gobmx-burgundy}
    \textbf{Manual SADER}\\
    \textcolor{gobmx-gray}{MVZ Jorge Carlos Berlín Montero - Presidente CESO}
}

% Footer izquierdo - información oficial (texto más pequeño)
\fancyfoot[L]{
    \scriptsize\color{gobmx-gray}
    \textbf{Centro de Consulta de Acuerdos Zoosanitarios}\\
    Secretaría de Agricultura y Desarrollo Rural
}

% Footer centro - numeración de páginas (texto más pequeño)
\fancyfoot[C]{
    \scriptsize\color{gobmx-gray}
    Página \thepage\ de \pageref{LastPage}
}

% Footer derecho - version stamp debajo de numeración (texto más pequeño)
\fancyfoot[R]{
    \scriptsize\color{gobmx-gray}
    Estado de Yucatán\\
    Versión 3.0.0 • \today
}

% Header derecho con logo SADER
\fancyhead[R]{
    \includegraphics[width=3.5cm]{images/sader-logo.png}
}

% ==========================================
% CONFIGURACIÓN DE TÍTULOS
% ==========================================
\titleformat{\section}
{\Large\bfseries\sffamily\color{gobmx-burgundy}}
{\thesection}{1em}{}

\titleformat{\subsection}
{\large\bfseries\sffamily\color{gobmx-gray}}
{\thesubsection}{1em}{}

\titleformat{\subsubsection}
{\normalsize\bfseries\sffamily\color{gobmx-gray}}
{\thesubsubsection}{1em}{}

% ==========================================
% ESPACIADO Y FORMATO
% ==========================================
\setstretch{1.15}
\setlength{\parskip}{6pt}
\setlength{\parindent}{0pt}

% Configuración para listas
\setlist[itemize]{leftmargin=*,topsep=3pt,itemsep=2pt}
\setlist[enumerate]{leftmargin=*,topsep=3pt,itemsep=2pt}

% ==========================================
% INICIO DEL DOCUMENTO
% ==========================================
\begin{document}

% ==========================================
% PÁGINA DE TÍTULO
% ==========================================
\begin{titlepage}
    \centering
    \vspace{2cm}
    
    % Logo SADER grande
    \includegraphics[width=8cm]{images/sader-logo.png}\\
    \vspace{1.5cm}
    
    % Título principal
    {\Huge\bfseries\color{gobmx-burgundy}MANUAL OPERATIVO}\\
    \vspace{0.5cm}
    {\LARGE\color{gobmx-gray}REPRESENTANTE ESTATAL SADER}\\
    \vspace{0.3cm}
    {\Large Centro de Consulta de Acuerdos Zoosanitarios}\\
    \vspace{0.2cm}
    {\large\textcolor{gobmx-gray}{Versión 3.0.0}}\\
    
    \vspace{1.5cm}
    
    % Caja con información del destinatario
    \begin{tcolorbox}[
        colback=gobmx-light,
        colframe=gobmx-burgundy,
        boxrule=2pt,
        arc=8pt,
        width=0.8\textwidth
    ]
        \large\centering
        \textbf{DESTINATARIO}\\
        \vspace{10pt}
        \textbf{\textcolor{gobmx-burgundy}{MVZ Jorge Carlos Berlín Montero}}\\
        \textcolor{gobmx-gray}{Presidente del CESO}\\
        \textcolor{gobmx-gray}{Representante Estatal - Oficina de Representación SADER Yucatán}\\
        \vspace{8pt}
        \textbf{Nivel de Acceso:} \textcolor{gobmx-burgundy}{Administrador Ejecutivo}
    \end{tcolorbox}
    
    \vfill
    
    % Información del pie
    \textcolor{gobmx-gray}{\textbf{Preparado por:} MVZ Sergio Muñoz de Alba Medrano}\\
    \textcolor{gobmx-gray}{\textbf{Fecha:} 22 de noviembre de 2025}\\
    \textcolor{gobmx-gray}{\textbf{Organismo:} Secretaría de Agricultura y Desarrollo Rural (SADER) - Estado de Yucatán}\\
    
\end{titlepage}

% ==========================================
% TABLA DE CONTENIDOS
% ==========================================
\tableofcontents
\newpage

% ==========================================
% CONTENIDO PRINCIPAL
% ==========================================

\section{MANUAL PRESIDENCIAL - PRESIDENTE DEL CESO}

\textbf{DESTINATARIO}: MVZ Jorge Carlos Berlín Montero \\
\textbf{CARGO}: Representante Estatal de la Oficina de Representación SADER Yucatán \\
\textbf{FUNCIÓN EN EL CESO}: Presidente del Consejo Estatal de Seguimiento Operativo (CESO) del SINIIGA-SINIDA \\
\textbf{NIVEL DE ACCESO}: Federal \\
\textbf{FECHA}: 22 de noviembre de 2025

\textbf{ROL EN EL SISTEMA}: Administrador Ejecutivo con acceso completo a CESO y APHIS-USDA \\
Usuario: representacion.yuc@agricultura.gob.mx \\
Contraseña: \verb|SaderYuc#2025| \\
Nivel: Federal

\textbf{ATRIBUCIONES OFICIALES COMO PRESIDENTE DEL CESO}:
Conforme a los Lineamientos del CESO, el Presidente tiene las siguientes atribuciones:
\begin{itemize}
    \item Expedir convocatorias y órdenes del día para las sesiones del CESO
    \item Presidir y dirigir las sesiones del CESO
    \item Vigilar el debido cumplimiento de los acuerdos tomados por el CESO
    \item Someter a consideración del CDN del SINIIGA-SINIDA los informes que correspondan
    \item Contar con derecho a voz, voto y voto de calidad en las sesiones
    \item Firmar las actas de las reuniones
    \item Cancelar sesiones ordinarias cuando no existan asuntos a tratar
    \item Coordinar las actividades del CESO con las demás instancias del SINIIGA-SINIDA
    \item Representar al CESO ante otras instancias gubernamentales y organismos
\end{itemize}

\textbf{CAPACIDADES DISPONIBLES}: Consulta de todos los acuerdos CESO y APHIS-USDA, gestión de acuerdos asignados directamente, supervisión delegada a través de su equipo, reportes ejecutivos al CDN

\section{ACCESO AL SISTEMA}

\subsection{URL OFICIAL}
\begin{tcolorbox}[colback=gobmx-light, colframe=gobmx-burgundy, boxrule=1pt]
\texttt{https://ceso-aphis-yuc.web.app}
\end{tcolorbox}

\subsection{SUS CREDENCIALES PERSONALES}
\begin{tcolorbox}[colback=gobmx-light, colframe=gobmx-gray, boxrule=1pt]
\textbf{Usuario:} representacion.yuc@agricultura.gob.mx \\
\textbf{Contraseña:} \verb|SaderYuc#2025| \\
\textbf{Nivel:} Federal
\end{tcolorbox}

\textbf{IMPORTANTE}: Estas credenciales son confidenciales y de uso personal exclusivo.

\section{CONCEPTO DEL SISTEMA - CENTRO DE CONSULTA DE ACUERDOS SANITARIOS}

\subsection{¿Qué es el Sistema para el Presidente del CESO?}
El Centro de Consulta de Acuerdos Sanitarios es una plataforma web gubernamental oficial que sirve como \textbf{sistema de monitoreo de acuerdos} y \textbf{repositorio de documentación} para facilitar la consulta, seguimiento y gestión de acuerdos sanitarios pecuarios:
\begin{itemize}
    \item \textbf{Dashboard} = Tablero presidencial con métricas de acuerdos actualizadas
    \item \textbf{Módulos} = Áreas especializadas CESO y APHIS-USDA bajo su presidencia
    \item \textbf{Gestión} = Seguimiento de cumplimiento de acuerdos del CESO del SINIIGA-SINIDA
\end{itemize}

\subsection{Diferencia Entre Autoridad Presidencial vs. Capacidades Técnicas del Sistema}

\begin{tcolorbox}[colback=blue!5, colframe=gobmx-burgundy, boxrule=2pt, title={\textbf{IMPORTANTE: Entender las Limitaciones Técnicas}}]
\textbf{SU AUTORIDAD PRESIDENCIAL} (Según Lineamientos del CESO):
\begin{itemize}
    \item Presidir y dirigir sesiones del CESO
    \item Vigilar cumplimiento de TODOS los acuerdos tomados por el CESO
    \item Convocar y coordinar actividades con organismos participantes
    \item Representar al CESO ante instancias gubernamentales
\end{itemize}

\textbf{LO QUE EL SISTEMA PERMITE HACER DIRECTAMENTE}:
\begin{itemize}
    \item Consultar y supervisar TODOS los acuerdos de ambas salas
    \item Gestionar evidencias y cambiar estados ÚNICAMENTE en acuerdos donde aparezca como "Responsable del Seguimiento"
    \item Descargar evidencias y documentos de cualquier acuerdo
    \item Generar reportes ejecutivos completos
\end{itemize}

\textbf{MODELO DE TRABAJO SUGERIDO}: Use el sistema para supervisión general y coordine directamente con Ing. Abigail Estrada, MVZ Luis Flores, y los responsables técnicos específicos de cada acuerdo para la gestión operativa de evidencias.
\end{tcolorbox}

\subsection{Dos Grupos de Trabajo Bajo Su Presidencia del CESO}
Como \textbf{Presidente del CESO}, el sistema le proporciona seguimiento presidencial completo de acuerdos \textbf{EN AMBOS} grupos de trabajo:

\textbf{CONSEJO ESTATAL DE SEGUIMIENTO OPERATIVO (CESO) DEL SINIIGA-SINIDA}
\begin{itemize}
    \item \textbf{Función Oficial}: Instancia de coordinación y vigilancia de la operación del SINIIGA-SINIDA en la entidad de Yucatán
    \item \textbf{Tareas Presidenciales del CESO (según Lineamientos Oficiales)}:
    \begin{itemize}
        \item Aprobar calendario de sesiones ordinarias (mínimo trimestrales)
        \item Vigilar, coordinar, supervisar y evaluar actividades de VAS, VAL y CSI
        \item Informar al CDN del SINIIGA-SINIDA sobre operación estatal
        \item Analizar información para cumplimiento de NOM-001-SAG/GAN-2015
        \item Autorizar solicitudes de dispositivos de identificación oficial
        \item Validar informes de Ventanillas Autorizadas (VAS)
    \end{itemize}
    \item \textbf{Organizaciones del CESO}: SENASICA, SEDER, Organismo Auxiliar (CEFPPY), CON
\end{itemize}

\textbf{GRUPO DE TRABAJO APHIS-USDA/SENASICA}
\begin{itemize}
    \item \textbf{Función Oficial}: Grupo de trabajo bilateral México-Estados Unidos para el seguimiento a las recomendaciones críticas emitidas por APHIS-USDA en materia de tuberculosis bovina
    \item \textbf{Funciones del Grupo de Trabajo}:
    \begin{itemize}
        \item Implementación de programas de control de tuberculosis bovina
        \item Seguimiento a recomendaciones específicas de APHIS-USDA
        \item Atención a observaciones del reporte SENASICA
        \item Desarrollo de protocolos de certificación para exportación
        \item Coordinación de acciones correctivas y preventivas
        \item Facilitación del comercio de ganado bovino México-EE.UU.
    \end{itemize}
    \item \textbf{Coordinación}: SENASICA, SADER, APHIS-USDA, organismos certificadores
\end{itemize}

\section{EL SISTEMA - SU CENTRO DE GESTIÓN PRESIDENCIAL DEL CESO}

\subsection{Dashboard - Su Tablero Ejecutivo de Acuerdos}
Al acceder al sistema verá \textbf{tarjetas de resumen} que muestran el estado de los acuerdos:

\textbf{ESTADOS DE ACUERDOS}: Seguimiento de cumplimiento por categoría
\begin{itemize}
    \item \textbf{Vencidos}: Acuerdos con fecha de cumplimiento superada sin completar
    \item \textbf{Pendientes}: Acuerdos emitidos, en espera de implementación
    \item \textbf{En Progreso}: Acuerdos con implementación iniciada, en desarrollo
    \item \textbf{Completados}: Acuerdos totalmente implementados con evidencias
    \item \textbf{Total General}: Conteo completo de todos los acuerdos monitoreados
\end{itemize}

\textbf{SUBTEXTO PRESIDENCIAL}: El sistema también muestra cuántos acuerdos están asignados directamente a la Oficina de Representación Estatal SADER como responsable.

\textbf{IMPORTANTE}: Cada tarjeta es CLICKEABLE - al hacer clic se abre la lista de acuerdos de ese estado del CESO.

\section{CÓMO USAR EL SISTEMA - GUÍA EJECUTIVA PASO A PASO}

\subsection{PASO 1: Acceso al Sistema de Consulta de Acuerdos}
\begin{enumerate}
    \item Abra su navegador web (Chrome, Firefox, Safari, Edge)
    \item Vaya a: \texttt{https://ceso-aphis-yuc.web.app}
    \item En la página de inicio, verá \textbf{DOS MÓDULOS}:
    \begin{itemize}
        \item \textbf{Módulo CESO} - Para consulta y seguimiento de acuerdos del CESO del SINIIGA-SINIDA
        \item \textbf{Módulo APHIS-USDA} - Para seguimiento a recomendaciones críticas de tuberculosis bovina
    \end{itemize}
\end{enumerate}

\subsection{PASO 2: Seleccionar Área de Coordinación Presidencial del CESO}
Como \textbf{Presidente del CESO}, usted tiene acceso presidencial completo para consulta y supervisión en \textbf{AMBAS SALAS}, con gestión directa limitada a acuerdos específicamente asignados a la Representación Estatal SADER:

\textbf{Opción A: Entrar a SALA CESO (Presidencia del CESO para Protocolos del Sistema Nacional de Identificación Individual de Ganado (SINIIGA) y Sistema Nacional de Identificación Animal para Bovinos y Colmenas (SINIDA))}
\begin{itemize}
    \item Haga clic en "Ingresar a CESO"
    \item Ingrese credenciales: \texttt{representacion.yuc@agricultura.gob.mx} / \verb|SaderYuc#2025|
\end{itemize}

\textbf{Opción B: Entrar a SALA APHIS-USDA (Coordinación Presidencial del CESO con Campaña TB)}
\begin{itemize}
    \item Haga clic en "Ingresar a APHIS-USDA"
    \item Use las mismas credenciales federales operativas
\end{itemize}

\section{SUPERVISIÓN PRESIDENCIAL Y DELEGACIÓN DE EVIDENCIAS}

\subsection{LIMITACIONES IMPORTANTES DEL SISTEMA}
\begin{tcolorbox}[colback=yellow!10, colframe=orange, boxrule=2pt]
\textbf{RESTRICCIÓN TÉCNICA FUNDAMENTAL}: El sistema permite la gestión directa de evidencias y cambio de estados \textbf{ÚNICAMENTE} para acuerdos donde usted aparece específicamente asignado en el campo "Responsable del Seguimiento". Para otros acuerdos, su rol es de \textbf{supervisión y coordinación} a través de los responsables directos.
\end{tcolorbox}

\subsection{¿Qué son las Evidencias para el Presidente del CESO?}
Como Presidente del CESO, usted \textbf{supervisa y coordina} la gestión de evidencias que comprueban el cumplimiento de acuerdos:

\begin{itemize}
    \item \textbf{PDFs}: Reportes institucionales, oficios, documentos de seguimiento
    \item \textbf{Fotos}: Evidencia visual de implementación de programas
    \item \textbf{Documentos}: Word, Excel con datos de gestión o informes institucionales
\end{itemize}

\subsection{Cómo Ejercer la Supervisión Presidencial de Evidencias}
\begin{enumerate}
    \item \textbf{Revisar el tablero presidencial}: Use el Panel de Acuerdos o haga clic en las Flash Cards para supervisión general
    \item \textbf{Para acuerdos directamente asignados a usted}: Puede gestionar evidencias y cambiar estados directamente
    \item \textbf{Para otros acuerdos bajo su presidencia}: Coordine con los responsables directos (MVZ Francis Genovez, MVZ María del Refugio Medina, etc.) quienes tienen acceso técnico para gestionar evidencias
    \item \textbf{Supervisión ejecutiva}: Monitoree el progreso general y ejerza liderazgo a través de comunicación directa con los responsables asignados
\end{enumerate}

\section{ACCESO MÓVIL PRESIDENCIAL}

\subsection{Desde su Teléfono Celular o Tableta}
\begin{itemize}
    \item \textbf{URL}: La misma - \texttt{https://ceso-aphis-yuc.web.app}
    \item \textbf{Credenciales}: Las mismas que en computadora
    \item \textbf{Experiencia}: El HALL se adapta perfectamente a supervisión presidencial móvil
\end{itemize}

\subsection{Casos de Uso Móvil Presidenciales del CESO}
\begin{itemize}
    \item \textbf{En reuniones de alto nivel}: Supervisar el estado general del CESO y delegar seguimiento
    \item \textbf{En reuniones interinstitucionales}: Verificar métricas presidenciales y estatus de acuerdos
    \item \textbf{En eventos oficiales}: Revisar dashboards ejecutivos y coordinar con su equipo
    \item \textbf{Toma de decisiones presidenciales}: Instruir acciones inmediatas a su equipo SADER
\end{itemize}

\section{ROL PRESIDENCIAL vs. GESTIÓN OPERATIVA EN EL SISTEMA}

\subsection{Su Función Como Presidente del CESO}
Como \textbf{Presidente del CESO}, usted ejerce liderazgo y supervisión general sobre todos los acuerdos, pero el sistema está diseñado con \textbf{responsabilidades distribuidas} donde cada acuerdo tiene un responsable técnico específico asignado.

\begin{tcolorbox}[colback=gobmx-light, colframe=gobmx-burgundy, boxrule=2pt, title={\textbf{MODELO DE SUPERVISIÓN PRESIDENCIAL}}]
\textbf{FUNCIÓN PRESIDENCIAL}: Supervisión, coordinación y liderazgo estratégico de todos los acuerdos del CESO \\
\textbf{FUNCIÓN TÉCNICA EN EL SISTEMA}: Gestión operativa limitada a acuerdos específicamente asignados a la Representación Estatal SADER \\
\textbf{MODELO DE TRABAJO}: Supervisión presidencial + Coordinación con responsables técnicos específicos
\end{tcolorbox}

\subsection{Responsables Técnicos Bajo Su Supervisión Presidencial}
El sistema asigna la gestión operativa de acuerdos a responsables técnicos específicos, todos bajo su coordinación presidencial:

\textbf{MVZ Francis A. Genovéz Chanona} - Gerente Regional SINIIGA
\begin{itemize}
    \item \textbf{Acuerdos Típicamente Asignados}: Protocolos SINIIGA-SINIDA, identificación de ganado
    \item \textbf{Su Coordinación Presidencial}: Revisar progreso, dirigir prioridades estratégicas
\end{itemize}

\textbf{MVZ María del Refugio Medina Juárez} - SENASICA
\begin{itemize}
    \item \textbf{Acuerdos Típicamente Asignados}: Convocatorias, calendarios de sesiones
    \item \textbf{Su Coordinación Presidencial}: Aprobar convocatorias, definir agendas presidenciales
\end{itemize}

\section{COORDINACIÓN CON OTRAS ORGANIZACIONES}

\subsection{Equipo SADER con Acceso al Sistema}
\textbf{MVZ Jorge Carlos Berlín Montero} - Presidente del CESO

\begin{itemize}
    \item \textbf{Nivel de Acceso}: Federal - Supervisión presidencial completa
    \item \textbf{Visibilidad}: Consulta todos los acuerdos, gestiona directamente solo aquellos asignados a Representación Estatal SADER
    \item \textbf{Funciones Presidenciales}: Dirección estratégica, supervisión general, coordinación con responsables técnicos, toma de decisiones ejecutivas, representación institucional
\end{itemize}

\textbf{Ing. Abigail Estrada Estrada} - Subdelegado Agropecuario SADER

\begin{itemize}
    \item \textbf{Relación con el Presidente}: Suplente y brazo operativo directo
    \item \textbf{Instrucciones que recibe}: Gestión operativa de evidencias, seguimiento detallado de acuerdos, preparación de reportes ejecutivos
    \item \textbf{Funciones Delegadas}: Ejecutar directrices presidenciales, coordinar actividades del CESO, gestionar información y preparar resúmenes ejecutivos para el Presidente
\end{itemize}

\textbf{MVZ Luis Martín Flores Martínez} - Jefe de Programa de Fomento Agropecuario SADER

\begin{itemize}
    \item \textbf{Relación con el Presidente}: Apoyo logístico especializado
    \item \textbf{Instrucciones que recibe}: Implementar programas específicos, recopilar evidencias de campo, ejecutar operativos
    \item \textbf{Funciones Delegadas}: Gestión operativa de programas, recolección de evidencias, implementación de políticas presidenciales del CESO
\end{itemize}

\subsection{Organizaciones con Coordinación Federal Operativa}

\textbf{SENASICA - Servicio Nacional de Sanidad, Inocuidad y Calidad Agroalimentaria}
\begin{itemize}
    \item \textbf{MVZ José Joaquín Peral Rodríguez} - Representante Estatal SENASICA
    \item \textbf{MVZ Víctor Manuel Calderón Jiménez} - Jefe de Campañas SENASICA
\end{itemize}

\textbf{SEDER - Secretaría de Desarrollo Rural del Estado de Yucatán}
\begin{itemize}
    \item \textbf{Prof. Edgardo Gilberto Medina Estrada} - Secretario SEDER
    \item \textbf{Ing. Juan Carlos Rodríguez Andrade} - Director de Sanidad SEDER
    \item \textbf{MVZ Moisés Abraham Martín Sima} - Jefe de Campañas SEDER
\end{itemize}

\textbf{CEFPPY - Comité Estatal para el Fomento y Protección Pecuaria del Estado de Yucatán, AC}
\begin{itemize}
    \item \textbf{Dr. Roger Armando Díaz Mendoza} - Presidente Comité CEFPPY
    \item \textbf{MVZ Alfredo Colín Álvarez} - Gerente Comité CEFPPY
    \item \textbf{MVZ Jorge Carlos Concha Cutz} - Jefe de Campañas CEFPPY
    \item \textbf{MVZ Oscar Correa} - Jefe de Campañas CEFPPY
\end{itemize}

\textbf{SINIIGA - Sistema Nacional de Identificación Individual de Ganado}
\begin{itemize}
    \item \textbf{MVZ Francis A. Genovéz Chanona} - Gerente Regional SINIIGA
\end{itemize}

\textbf{UGRY - Unión Ganadera Regional de Yucatán}
\begin{itemize}
    \item \textbf{MVZ Angel Miqueas Castillo Caamal} - Jefe Ventanilla UGRY
\end{itemize}

\section{REPORTES EJECUTIVOS OPERATIVOS Y SEGUIMIENTO FEDERAL}

\subsection{Información Disponible de Desempeño del CESO}
\begin{enumerate}
    \item Acuerdos Federales Operativos Completados por Organización
    \item Tiempo Promedio de Respuesta en Supervisión Federal Operativa
    \item Tipos de Evidencias Federales Operativas Más Utilizadas
    \item Cumplimiento vs. Fechas Límite Federales Operativas por Organización
    \item Comparación de Desempeño Entre Organizaciones Bajo Supervisión Federal Operativa
    \item Métricas de Eficiencia Federal Operativa del Equipo SADER
    \item Indicadores de Desempeño Federal del Estado de Yucatán
\end{enumerate}

\section{CAPACIDADES PRESIDENCIALES Y DELEGACIÓN}

\subsection{FUNCIONALIDADES DISPONIBLES COMO PRESIDENTE DEL CESO}
\begin{itemize}
    \item \textbf{Consultar todos los acuerdos} de ambos grupos de trabajo con filtros avanzados
    \item \textbf{Revisar estados de cumplimiento} y fechas de vencimiento de acuerdos
    \item \textbf{Acceder a evidencias} de cumplimiento cargadas por responsables
    \item \textbf{Actualizar estados de acuerdos} asignados a la Representación SADER
    \item \textbf{Cargar evidencias} de cumplimiento para acuerdos bajo su responsabilidad
    \item \textbf{Consultar repositorio} de documentos oficiales de ambas organizaciones
    \item \textbf{Generar reportes} de seguimiento y cumplimiento
    \item \textbf{Coordinar con equipo SADER} el seguimiento de acuerdos institucionales
\end{itemize}

\section{HERRAMIENTAS DE GESTIÓN DISPONIBLES}

\subsection{Funcionalidades del Sistema}
\textbf{Capacidades disponibles}: \textbf{Consulta y seguimiento de acuerdos} con gestión de evidencias

\textbf{INFORMACIÓN PRESIDENCIAL - PROGRAMA CESO}:
\begin{itemize}
    \item \textbf{Dashboard Presidencial de Identificación}: Métricas ejecutivas de todos los acuerdos del programa
    \item \textbf{Lista de Supervisión Presidencial}: Asuntos que requieren su atención o delegación específica
    \item \textbf{Proceso de Aprobación}: Herramienta para autorizar recomendaciones de su equipo técnico
    \item \textbf{Historial Presidencial}: Registro de decisiones y delegaciones realizadas por su gestión
    \item \textbf{Métricas de Gestión}: Análisis de desempeño de su equipo y organizaciones bajo supervisión del CESO
\end{itemize}

\textbf{INFORMACIÓN PRESIDENCIAL - PROGRAMA APHIS-USDA}:
\begin{itemize}
    \item \textbf{Panel Presidencial TB}: Estatus ejecutivo del programa de control de tuberculosis
    \item \textbf{Documentación Estratégica}: Resúmenes ejecutivos preparados por su equipo sobre cumplimiento
    \item \textbf{Estado de Certificación}: Seguimiento presidencial del progreso hacia certificación internacional
    \item \textbf{Reportes de Gestión}: Documentación de actividades supervisadas por su administración
    \item \textbf{Coordinación Internacional}: Herramientas para dirigir relaciones con APHIS-USDA desde la presidencia del CESO
\end{itemize}

\section{SEGURIDAD PRESIDENCIAL}

\subsection{Protección de Información Presidencial}
\begin{itemize}
    \item \textbf{Autenticación presidencial segura} con credenciales de máxima autoridad
    \item \textbf{Cifrado de información estratégica} y documentación presidencial
    \item \textbf{Backup automático} de decisiones y directrices presidenciales
    \item \textbf{Logs de acceso presidencial} para auditorías de gestión institucional
    \item \textbf{Control de permisos jerárquico} con autoridad presidencial máxima
\end{itemize}

\subsection{Buenas Prácticas de Seguridad Presidencial}
\begin{itemize}
    \item Mantenga la confidencialidad absoluta de sus credenciales presidenciales
    \item Cierre sesión al concluir la supervisión de información estratégica
    \item Use dispositivos seguros para acceder a información presidencial del CESO
    \item Instruya a su equipo sobre protocolos de seguridad y reporte incidentes inmediatamente
\end{itemize}

\section{POSIBLES USOS DEL SISTEMA - CASOS PRÁCTICOS EJECUTIVOS OPERATIVOS}

\subsection{Rutina Presidencial de Supervisión Diaria}
\textbf{Flujo de Supervisión Presidencial Recomendado (10-15 minutos)}:

\begin{enumerate}
    \item \textbf{Revisar Dashboard Presidencial}: ¿Hay situaciones críticas que requieren atención presidencial?
    \item \textbf{Verificar alertas ejecutivas}: ¿Qué asuntos necesitan decisión o delegación presidencial?
    \item \textbf{Instruir al equipo}: Delegar revisiones específicas a Ing. Abigail Estrada o MVZ Luis Flores
    \item \textbf{Supervisar progreso}: ¿El equipo está cumpliendo las instrucciones presidenciales?
    \item \textbf{Revisar recomendaciones}: Evaluar propuestas del equipo y autorizar acciones
    \item \textbf{Dirigir estrategia}: Establecer prioridades y directrices para el CESO
\end{enumerate}

\subsection{Tipos de Información Presidencial que Su Equipo Le Presenta}
\begin{itemize}
    \item \textbf{Resúmenes Ejecutivos}: Informes preparados por su equipo sobre estado de acuerdos críticos
    \item \textbf{Reportes de Gestión}: Documentación del desempeño de las organizaciones bajo supervisión del CESO
    \item \textbf{Comunicaciones Institucionales}: Correspondencia oficial que requiere su conocimiento o autorización presidencial
    \item \textbf{Análisis de Cumplimiento}: Evaluaciones del equipo sobre cumplimiento de programas institucionales
    \item \textbf{Reportes de Certificación}: Documentación estratégica para procesos de certificación internacional
    \item \textbf{Correspondencia Internacional}: Documentos de coordinación con APHIS-USDA que requieren atención presidencial
    \item \textbf{Evaluaciones de Desempeño}: Análisis del rendimiento del equipo SADER y organizaciones del CESO
\end{itemize}

\section{SOPORTE TÉCNICO Y CAPACITACIÓN EJECUTIVA OPERATIVA}

\subsection{Contacto Inmediato}
\begin{tcolorbox}[colback=gobmx-light, colframe=gobmx-burgundy, boxrule=1pt]
\textbf{Responsable Técnico}: Sergio Muñoz de Alba Medrano \\
\textbf{Email}: smunoz.sader@gmail.com \\
\textbf{Teléfono}: +52 999 200 5550 \\
\textbf{Tiempo de Respuesta}: Inmediato para nivel ejecutivo operativo garantizado
\end{tcolorbox}

\subsection{Tipos de Soporte Disponible para Nivel Ejecutivo Operativo}
\begin{itemize}
    \item \textbf{Soporte técnico inmediato} para problemas de acceso o funcionamiento ejecutivo operativo
    \item \textbf{Capacitación ejecutiva operativa adicional} si requiere refuerzo en alguna función de supervisión
    \item \textbf{Consultoría operativa} para optimizar su flujo de supervisión federal operativa
    \item \textbf{Reportes especiales ejecutivos operativos} con métricas particulares federales operativas
    \item \textbf{Configuración personalizada} de alertas y notificaciones ejecutivas operativas
\end{itemize}

\section{BENEFICIOS DEL SISTEMA PARA LA REPRESENTACIÓN SADER}

\subsection{Organización y Seguimiento}
\begin{itemize}
    \item \textbf{Visibilidad completa} de todos los acuerdos y su estado de cumplimiento
    \item \textbf{Control de fechas límite} para priorizar seguimiento de compromisos
    \item \textbf{Progreso consolidado} para evaluación de desempeño institucional
    \item \textbf{Historial completo} de acuerdos cumplidos y evidencias documentadas
\end{itemize}

\subsection{Eficiencia Operativa}
\begin{itemize}
    \item \textbf{Consulta rápida} de evidencias y documentos desde cualquier dispositivo
    \item \textbf{Acceso móvil} para revisar acuerdos y estados en tiempo real
    \item \textbf{Enfoque claro} en tareas prioritarias y vencimientos próximos
    \item \textbf{Seguimiento sistemático} basado en estados y fechas de cumplimiento
    \item \textbf{Documentación centralizada} para demostrar cumplimiento institucional
\end{itemize}

\section{ALCANCE Y FUNCIONES DEL SISTEMA}

\subsection{Funcionalidades del Sistema}
El Centro de Consulta de Acuerdos Sanitarios es:
\begin{itemize}
    \item Un monitor de acuerdos y resoluciones de ambos grupos de trabajo
    \item Un repositorio de documentos oficiales organizados por categorías
    \item Una herramienta de seguimiento de cumplimiento con gestión de evidencias
    \item Un sistema de consulta que facilita la transparencia gubernamental
    \item Una plataforma de gestión documental bilingüe (español/inglés)
\end{itemize}

\subsection{Relación con Otros Sistemas}
\textbf{Importante}: El sistema monitorea acuerdos SOBRE el SINIIGA-SINIDA, no opera directamente el SINIIGA-SINIDA. Los acuerdos CESO se refieren a la coordinación, supervisión y evaluación de las actividades del SINIIGA-SINIDA según los Lineamientos oficiales del CESO.

\section{PRIMERAS REFERENCIAS PARA COMENZAR A USAR EL SISTEMA}

\subsection{Impacto de Su Liderazgo en el CESO}
\textbf{Contribución Presidencial al CESO}:

\begin{itemize}
    \item \textbf{Modernización institucional} del CESO bajo su dirección presidencial estratégica
    \item \textbf{Eficiencia organizacional} mejorada mediante delegación efectiva y supervisión presidencial
    \item \textbf{Transparencia institucional} en procesos de toma de decisiones del CESO
    \item \textbf{Coordinación interinstitucional} optimizada entre organismos federales, estatales y sector privado
    \item \textbf{Posicionamiento internacional} del Estado de Yucatán para certificación y exportación ganadera
\end{itemize}

\textbf{Desarrollo de Liderazgo Presidencial}:

\begin{itemize}
    \item \textbf{Competencias directivas} fortalecidas para la gestión presidencial moderna del CESO
    \item \textbf{Métricas de gestión presidencial} para evaluación de desempeño institucional
    \item \textbf{Experiencia ejecutiva} en dirección de sistemas gubernamentales complejos
    \item \textbf{Legado institucional documentado} en programas estratégicos nacionales e internacionales
\end{itemize}

\vspace{1cm}

\begin{center}
\textcolor{gobmx-burgundy}{\textbf{Este manual presenta las funcionalidades del Centro de Consulta de Acuerdos Sanitarios para el Presidente del CESO. El sistema está diseñado como herramienta de consulta y seguimiento de acuerdos del CESO del SINIIGA-SINIDA y del Grupo de Trabajo APHIS-USDA/SENASICA, facilitando el monitoreo de cumplimiento y la gestión documental de ambos grupos de trabajo.}}
\end{center}

\vfill

\begin{center}
\textbf{MVZ Jorge Carlos Berlín Montero}\\
\textit{Presidente del CESO}

\vspace{0.5cm}

\textcolor{gobmx-gray}{\textbf{Por}: MVZ Sergio Muñoz de Alba Medrano}\\
\textcolor{gobmx-gray}{\textit{Desarrollador del Sistema}}\\
\textcolor{gobmx-gray}{\textbf{Fecha}: 22 de noviembre de 2025}\\
\textcolor{gobmx-gray}{\textbf{Versión del Sistema}: 3.0.0}

\vspace{0.5cm}

\textit{Centro de Consulta de Acuerdos Zoosanitarios}\\
\textit{Gobierno de México - Estado de Yucatán}\\
\texttt{https://ceso-aphis-yuc.web.app}
\end{center}

\end{document}
