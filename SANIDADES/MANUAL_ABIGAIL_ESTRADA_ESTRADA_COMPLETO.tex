\documentclass[11pt,a4paper]{article}
\usepackage[utf8]{inputenc}
\usepackage[spanish]{babel}
\usepackage[letterpaper,top=2.5cm,bottom=3cm,left=2.5cm,right=2.5cm,headheight=40pt,headsep=0.8cm,footskip=1.2cm]{geometry}
\usepackage{graphicx}
\usepackage{fancyhdr}
\usepackage{xcolor}
\usepackage{titlesec}
\usepackage{enumitem}
\usepackage{array}
\usepackage{longtable}
\usepackage{booktabs}
\usepackage{multicol}
\usepackage{microtype}
\usepackage{hyperref}
\usepackage{lastpage}
\usepackage{tikz}
\usepackage{tcolorbox}
\usepackage{setspace}
\usepackage{caption}

% ==========================================
% COLORES OFICIALES GOB.MX
% ==========================================
\definecolor{gobmx-burgundy}{RGB}{139,21,56}
\definecolor{gobmx-gray}{RGB}{84,84,84}
\definecolor{gobmx-light}{RGB}{221,201,163}
\definecolor{gobmx-gold}{RGB}{188,149,92}
\definecolor{gobmx-dark}{RGB}{19,50,46}

% ==========================================
% CONFIGURACIÓN DE HYPERREF
% ==========================================
\hypersetup{
    pdftitle={Manual SADER - Abigail Estrada Estrada},
    pdfauthor={SADER Yucatan},
    pdfsubject={Manual de Usuario Centro de Consulta de Acuerdos Zoosanitarios},
    pdfkeywords={SADER, Subdelegado, Acuerdos, Sanitarios},
    colorlinks=true,
    linkcolor=gobmx-burgundy,
    citecolor=gobmx-burgundy,
    filecolor=gobmx-burgundy,
    urlcolor=gobmx-burgundy
}

% ==========================================
% CONFIGURACIÓN DE HEADERS Y FOOTERS
% ==========================================
\pagestyle{fancy}
\fancyhf{}

% Header izquierdo con título del manual
\fancyhead[L]{
    \footnotesize\color{gobmx-burgundy}
    \textbf{Manual SADER}\\
    \textcolor{gobmx-gray}{Ing. Abigail Estrada Estrada - Suplente Presidente CESO}
}

% Footer izquierdo - información oficial (texto más pequeño)
\fancyfoot[L]{
    \scriptsize\color{gobmx-gray}
    \textbf{Centro de Consulta de Acuerdos Zoosanitarios}\\
    Secretaría de Agricultura y Desarrollo Rural
}

% Footer centro - numeración de páginas (texto más pequeño)
\fancyfoot[C]{
    \scriptsize\color{gobmx-gray}
    Página \thepage\ de \pageref{LastPage}
}

% Footer derecho - version stamp debajo de numeración (texto más pequeño)
\fancyfoot[R]{
    \scriptsize\color{gobmx-gray}
    Estado de Yucatán\\
    Versión 3.0.0 • \today
}

% Header derecho con logo SADER
\fancyhead[R]{
    \includegraphics[width=3.5cm]{images/sader-logo.png}
}

% ==========================================
% CONFIGURACIÓN DE TÍTULOS
% ==========================================
\titleformat{\section}
{\Large\bfseries\sffamily\color{gobmx-burgundy}}
{\thesection}{1em}{}

\titleformat{\subsection}
{\large\bfseries\sffamily\color{gobmx-gray}}
{\thesubsection}{1em}{}

\titleformat{\subsubsection}
{\normalsize\bfseries\sffamily\color{gobmx-gray}}
{\thesubsubsection}{1em}{}

% ==========================================
% ESPACIADO Y FORMATO
% ==========================================
\setstretch{1.15}
\setlength{\parskip}{6pt}
\setlength{\parindent}{0pt}

% Configuración para listas
\setlist[itemize]{leftmargin=*,topsep=3pt,itemsep=2pt}
\setlist[enumerate]{leftmargin=*,topsep=3pt,itemsep=2pt}

% ==========================================
% INICIO DEL DOCUMENTO
% ==========================================
\begin{document}

% ==========================================
% PÁGINA DE TÍTULO
% ==========================================
\begin{titlepage}
    \centering
    \vspace{2cm}
    
    % Logo SADER grande
    \includegraphics[width=8cm]{images/sader-logo.png}\\
    \vspace{1.5cm}
    
    % Título principal
    {\Huge\bfseries\color{gobmx-burgundy}MANUAL OPERATIVO}\\
    \vspace{0.5cm}
    {\LARGE\color{gobmx-gray}SUBDELEGADO SADER}\\
    \vspace{0.3cm}
    {\Large Centro de Consulta de Acuerdos Zoosanitarios}\\
    \vspace{0.2cm}
    {\large\textcolor{gobmx-gray}{Versión 3.0.0}}\\
    
    \vspace{1.5cm}
    
    % Caja con información del destinatario
    \begin{tcolorbox}[
        colback=gobmx-light,
        colframe=gobmx-burgundy,
        boxrule=2pt,
        arc=8pt,
        width=0.8\textwidth
    ]
        \large\centering
        \textbf{DESTINATARIO}\\
        \vspace{10pt}
        \textbf{\textcolor{gobmx-burgundy}{Ing. Abigail Estrada Estrada}}\\
        \textcolor{gobmx-gray}{Suplente del Presidente del CESO}\\
        \textcolor{gobmx-gray}{Subdelegado Agropecuario - Oficina de Representación SADER Yucatán}\\
        \vspace{8pt}
        \textbf{Nivel de Acceso:} \textcolor{gobmx-burgundy}{Administrador Ejecutivo}
    \end{tcolorbox}
    
    \vfill
    
    % Información del pie
    \textcolor{gobmx-gray}{\textbf{Preparado por:} MVZ Sergio Muñoz de Alba Medrano}\\
    \textcolor{gobmx-gray}{\textbf{Fecha:} 10 de noviembre de 2025}\\
    \textcolor{gobmx-gray}{\textbf{Organismo:} Secretaría de Agricultura y Desarrollo Rural (SADER) - Estado de Yucatán}\\
    
\end{titlepage}

% ==========================================
% TABLA DE CONTENIDOS
% ==========================================
\tableofcontents
\newpage

% ==========================================
% CONTENIDO PRINCIPAL
% ==========================================

\section{MANUAL EJECUTIVO - SUPLENTE DEL PRESIDENTE DEL CESO}

\textbf{DESTINATARIO}: Ing. Abigail Estrada Estrada \\
\textbf{CARGO}: Subdelegado Agropecuario de la Oficina de Representación SADER Yucatán \\
\textbf{FUNCIÓN EN EL CESO}: Suplente del Presidente del Consejo Estatal de Seguimiento Operativo (CESO) del SINIIGA-SINIDA \\
\textbf{NIVEL DE ACCESO}: Federal \\
\textbf{FECHA}: Noviembre 11, 2025

\textbf{ROL EN EL SISTEMA}: Administrador Ejecutivo con acceso completo a CESO y APHIS-USDA \\
Usuario: abigail.estrada@yct.agricultura.gob.mx \\
Contraseña: AgroSader\$123 \\
Nivel: Federal

\textbf{ATRIBUCIONES OFICIALES COMO SUPLENTE DEL PRESIDENTE DEL CESO}:
Conforme a los Lineamientos del CESO, los suplentes tienen las siguientes atribuciones:
\begin{itemize}
    \item Ejercer TODAS las atribuciones del Presidente cuando este se encuentre ausente
    \item Expedir convocatorias y órdenes del día en ausencia del titular
    \item Presidir y dirigir sesiones del CESO cuando el titular no pueda asistir
    \item Vigilar el debido cumplimiento de los acuerdos tomados por el CESO
    \item Someter a consideración del CDN del SINIIGA-SINIDA informes cuando corresponda
    \item Contar con derecho a voz, voto y voto de calidad en ausencia del titular
    \item Firmar actas de reuniones cuando actúe como Presidente
    \item Cancelar sesiones ordinarias cuando no existan asuntos a tratar
    \item Apoyar permanentemente al Presidente titular en la coordinación del CESO
\end{itemize}

\textbf{CAPACIDADES DISPONIBLES}: Gestión completa de acuerdos CESO y APHIS-USDA, supervisión operativa federal, reportes ejecutivos al CDN

\section{ACCESO AL SISTEMA}

\subsection{URL OFICIAL}
\begin{tcolorbox}[colback=gobmx-light, colframe=gobmx-burgundy, boxrule=1pt]
\texttt{https://ceso-aphis-yuc.web.app}
\end{tcolorbox}

\subsection{SUS CREDENCIALES PERSONALES}
\begin{tcolorbox}[colback=gobmx-light, colframe=gobmx-gray, boxrule=1pt]
\textbf{Usuario:} abigail.estrada@yct.agricultura.gob.mx \\
\textbf{Contraseña:} AgroSader\$123 \\
\textbf{Nivel:} Federal
\end{tcolorbox}

\textbf{IMPORTANTE}: Estas credenciales son confidenciales y de uso personal exclusivo.

\section{CONCEPTO DEL HALL - SU CENTRO DE CONTROL EJECUTIVO DEL CESO}

\subsection{¿Qué es el HALL para el Suplente del Presidente del CESO?}
El HALL es como el \textbf{centro de control ejecutivo del CESO} desde donde usted ejerce todas las atribuciones del Presidente:
\begin{itemize}
    \item \textbf{Flash Cards} = Tablero ejecutivo con métricas del CESO actualizadas en tiempo real
    \item \textbf{Puertas} = Áreas especializadas para control ejecutivo del CESO y APHIS-USDA
    \item \textbf{Navegación ejecutiva} = Control presidencial del CESO cuando el titular se encuentra ausente
\end{itemize}

\subsection{Dos Grandes Programas Bajo la Presidencia del CESO}
Como \textbf{Suplente del Presidente del CESO}, el sistema le proporciona control ejecutivo completo de acuerdos \textbf{EN AMBOS} programas cuando actúe como Presidente:

\textbf{PROGRAMA CESO - Seguimiento de Acuerdos sobre Identificación y Movilización}
\begin{itemize}
    \item \textbf{Función SADER}: Seguimiento de acuerdos del Consejo Estatal de Seguimiento Operativo (CESO) relacionados con políticas de identificación y movilización de ganado en Yucatán
    \item \textbf{IMPORTANTE}: Este sistema NO maneja directamente el SINIIGA, SINIDA o REEMO, que son responsabilidad de CNOG-SINIIGA y otras instancias especializadas con sus propios sistemas
    \item \textbf{Tareas de Seguimiento de Acuerdos}:
    \begin{itemize}
        \item Seguimiento de acuerdos sobre protocolos de identificación y movilización
        \item Coordinación de acuerdos entre organismos estatales, federales y sector privado
        \item Seguimiento de cumplimiento de acuerdos sobre políticas de trazabilidad
        \item Monitoreo de acuerdos del CESO y su implementación por las instancias responsables
    \end{itemize}
    \item \textbf{Organizaciones Participantes}: SENASICA, SEDER, CEFPPY, Representantes SINIIGA/SINIDA, Productores (cada una con sus sistemas propios)
\end{itemize}

\textbf{PROGRAMA APHIS-USDA/SENASICA - Seguimiento de Recomendaciones Críticas}
\begin{itemize}
    \item \textbf{Función SADER}: Seguimiento de acuerdos del grupo de trabajo APHIS-USDA/SENASICA para atención de recomendaciones críticas de certificación sanitaria del Estado de Yucatán
    \item \textbf{IMPORTANTE}: Este sistema da seguimiento a ACUERDOS sobre recomendaciones, NO maneja directamente los sistemas operativos de certificación que son responsabilidad de SENASICA y APHIS-USDA
    \item \textbf{Tareas de Seguimiento de Acuerdos}:
    \begin{itemize}
        \item Seguimiento de acuerdos sobre atención a observaciones críticas APHIS-USDA
        \item Monitoreo de acuerdos para implementación de recomendaciones sobre tuberculosis bovina
        \item Seguimiento de acuerdos para cumplimiento de estándares internacionales
        \item Coordinación de acuerdos entre instancias para avance en certificación
    \end{itemize}
    \item \textbf{Organizaciones Participantes}: SENASICA, APHIS-USDA, SEDER, CEFPPY, Organismos Auxiliares (cada una operando sus sistemas especializados)
\end{itemize}

\section{EL HALL - SU CENTRO DE SUPERVISIÓN FEDERAL OPERATIVA}

\subsection{Flash Cards - Su Dashboard Ejecutivo Operativo}
Al llegar al HALL verá \textbf{5 tarjetas ejecutivas} que muestran información federal operativa a dos niveles:

\textbf{NÚMEROS PRINCIPALES (grandes)}: Estadísticas completas del programa federal operativo
\begin{itemize}
    \item \textbf{Vencidos}: Total de acuerdos federales operativos vencidos de todo el programa
    \item \textbf{Pendientes}: Total de acuerdos federales operativos pendientes del programa específico
    \item \textbf{En Progreso}: Total de acuerdos federales operativos que todo el programa está trabajando activamente
    \item \textbf{Completados}: Total de acuerdos federales operativos cumplidos con evidencias del programa
    \item \textbf{Total General}: Conteo completo de todos los acuerdos federales operativos del programa actual
\end{itemize}

\textbf{SUBTEXTO EJECUTIVO OPERATIVO}: El sistema también muestra cuántos acuerdos están asignados directamente al equipo SADER como responsable federal operativo.

\textbf{IMPORTANTE}: Cada tarjeta es CLICKEABLE - al hacer clic se abre la lista de acuerdos de ese estado federal operativo.

\section{CÓMO USAR EL SISTEMA - GUÍA EJECUTIVA OPERATIVA PASO A PASO}

\subsection{PASO 1: Acceso al Sistema Federal Operativo}
\begin{enumerate}
    \item Abra su navegador web (Chrome, Firefox, Safari, Edge)
    \item Vaya a: \texttt{https://ceso-aphis-yuc.web.app}
    \item En la página de inicio, verá \textbf{DOS OPCIONES}:
    \begin{itemize}
        \item \textbf{Sala CESO} - Para acuerdos del Sistema Nacional de Identificación Individual de Ganado (SINIIGA) y Sistema Nacional de Identificación Animal para Bovinos y Colmenas (SINIDA) (supervisión federal operativa)
        \item \textbf{Sala APHIS-USDA} - Para acuerdos relativos a las observaciones APHIS-USDA y supervisión federal operativa de recomendaciones
    \end{itemize}
\end{enumerate}

\subsection{PASO 2: Seleccionar Área de Supervisión Federal Operativa}
Como \textbf{Subdelegado SADER}, usted tiene supervisión ejecutiva operativa completa en \textbf{AMBAS SALAS}:

\textbf{Opción A: Entrar a SALA CESO (Supervisión Federal Operativa de Protocolos del Sistema Nacional de Identificación Individual de Ganado (SINIIGA) y Sistema Nacional de Identificación Animal para Bovinos y Colmenas (SINIDA))}
\begin{itemize}
    \item Haga clic en "Ingresar a CESO"
    \item Ingrese credenciales: \texttt{abigail.estrada@yct.agricultura.gob.mx} / \texttt{AgroSader\$123}
\end{itemize}

\textbf{Opción B: Entrar a SALA APHIS-USDA (Supervisión Federal Operativa de Campaña TB)}
\begin{itemize}
    \item Haga clic en "Ingresar a APHIS-USDA"
    \item Use las mismas credenciales federales operativas
\end{itemize}

\section{GESTIÓN EJECUTIVA OPERATIVA DE EVIDENCIAS}

\subsection{¿Qué son las Evidencias Federales Operativas?}
Las evidencias son \textbf{documentos que comprueban} que las organizaciones bajo supervisión federal operativa cumplieron con políticas federales:

\begin{itemize}
    \item \textbf{PDFs}: Reportes federales operativos, oficios, documentos de supervisión federal operativa
    \item \textbf{Fotos}: Evidencia visual de actividades federales operativas de campo
    \item \textbf{Documentos}: Word, Excel con datos federales operativos o informes de supervisión
\end{itemize}

\subsection{Cómo Supervisar la Subida de Evidencias Federales Operativas}
\begin{enumerate}
    \item \textbf{Encontrar el acuerdo}: Use el Panel de Acuerdos o haga clic en las Flash Cards
    \item \textbf{Abrir detalles ejecutivos operativos}: Clic en el acuerdo específico que las organizaciones completaron
    \item \textbf{Revisar evidencias}: Verificar archivos subidos de supervisión federal operativa
    \item \textbf{Aprobar/validar}: Una vez revisada la evidencia federal operativa, validar como "Completado"
\end{enumerate}

\section{ACCESO MÓVIL PARA SUPERVISIÓN FEDERAL OPERATIVA}

\subsection{Desde su Teléfono Celular o Tableta}
\begin{itemize}
    \item \textbf{URL}: La misma - \texttt{https://ceso-aphis-yuc.web.app}
    \item \textbf{Credenciales}: Las mismas que en computadora
    \item \textbf{Experiencia}: El HALL se adapta perfectamente a supervisión federal operativa móvil
\end{itemize}

\subsection{Casos de Uso Móvil Federales Operativos}
\begin{itemize}
    \item \textbf{En inspecciones federales operativas}: Supervisar evidencias de cumplimiento inmediatamente
    \item \textbf{En reuniones interinstitucionales}: Verificar acuerdos federales operativos pendientes al momento
    \item \textbf{En operativos federales}: Revisar estado de supervisión federal operativa del día
    \item \textbf{Documentación federal operativa}: Verificar evidencias de actividades de supervisión federal operativa
\end{itemize}

\section{COORDINACIÓN CON OTRAS ORGANIZACIONES}

\subsection{Equipo SADER con Acceso al Sistema}
\textbf{MVZ Jorge Carlos Berlín Montero} - Representante Estatal SADER

\begin{itemize}
    \item \textbf{Nivel de Acceso}: Administrador Ejecutivo SADER
    \item \textbf{Visibilidad}: Puede ver todos los acuerdos federales y supervisar evidencias
    \item \textbf{Funciones Ejecutivas}: Supervisión federal completa, reportes ejecutivos federales
\end{itemize}

\textbf{Ing. Abigail Estrada Estrada} - Subdelegado Agropecuario SADER

\begin{itemize}
    \item \textbf{Nivel de Acceso}: Administrador Ejecutivo - Suplente del Presidente del CESO
    \item \textbf{Visibilidad}: Supervisión ejecutiva completa de todos los acuerdos y evidencias federales
    \item \textbf{Funciones Ejecutivas}: Suplencia presidencial, coordinación ejecutiva, supervisión de desempeño federal
\end{itemize}

\textbf{MVZ Luis Martín Flores Martínez} - Jefe de Programa de Fomento Agropecuario SADER

\begin{itemize}
    \item \textbf{Nivel de Acceso}: Gestor Operativo especializado en programas federales
    \item \textbf{Visibilidad}: Puede ver acuerdos operativos federales y subir evidencias de gestión
    \item \textbf{Funciones Operativas}: Gestión de programas federales, evidencias federales de campo
\end{itemize}

\subsection{Organizaciones con Coordinación Federal Operativa}

\textbf{SENASICA - Servicio Nacional de Sanidad, Inocuidad y Calidad Agroalimentaria}
\begin{itemize}
    \item \textbf{MVZ José Joaquín Peral Rodríguez} - Representante Estatal SENASICA
    \item \textbf{MVZ Víctor Manuel Calderón Jiménez} - Jefe de Campañas SENASICA
\end{itemize}

\textbf{SEDER - Secretaría de Desarrollo Rural del Estado de Yucatán}
\begin{itemize}
    \item \textbf{Prof. Edgardo Gilberto Medina Estrada} - Secretario SEDER
    \item \textbf{Ing. Juan Carlos Rodríguez Andrade} - Director de Sanidad SEDER
    \item \textbf{MVZ Moisés Abraham Martín Sima} - Jefe de Campañas SEDER
\end{itemize}

\textbf{CEFPPY - Comité Estatal para el Fomento y Protección Pecuaria del Estado de Yucatán, AC}
\begin{itemize}
    \item \textbf{Dr. Roger Armando Díaz Mendoza} - Presidente Comité CEFPPY
    \item \textbf{MVZ Alfredo Colín Álvarez} - Gerente Comité CEFPPY
    \item \textbf{MVZ Jorge Carlos Concha Cutz} - Jefe de Campañas CEFPPY
    \item \textbf{MVZ Oscar Correa} - Jefe de Campañas CEFPPY
\end{itemize}

\textbf{SINIIGA - Sistema Nacional de Identificación Individual de Ganado}
\begin{itemize}
    \item \textbf{MVZ Francis A. Genovéz Chanona} - Gerente Regional SINIIGA
\end{itemize}

\textbf{UGRY - Unión Ganadera Regional de Yucatán}
\begin{itemize}
    \item \textbf{MVZ Angel Miqueas Castillo Caamal} - Jefe Ventanilla UGRY
\end{itemize}

\section{REPORTES EJECUTIVOS OPERATIVOS Y SEGUIMIENTO FEDERAL}

\subsection{Información Disponible de Desempeño Federal Operativo}
\begin{enumerate}
    \item Acuerdos Federales Operativos Completados por Organización
    \item Tiempo Promedio de Respuesta en Supervisión Federal Operativa
    \item Tipos de Evidencias Federales Operativas Más Utilizadas
    \item Cumplimiento vs. Fechas Límite Federales Operativas por Organización
    \item Comparación de Desempeño Entre Organizaciones Bajo Supervisión Federal Operativa
    \item Métricas de Eficiencia Federal Operativa del Equipo SADER
    \item Indicadores de Desempeño Federal del Estado de Yucatán
\end{enumerate}

\section{ALERTAS Y NOTIFICACIONES EJECUTIVAS OPERATIVAS}

\subsection{LO QUE PUEDE HACER EN EL SISTEMA (NIVEL EJECUTIVO OPERATIVO)}
\begin{itemize}
    \item \textbf{Ver todos los acuerdos federales operativos} de todas las organizaciones con fechas de vencimiento
    \item \textbf{Supervisar evidencias federales operativas} (PDF, imágenes, documentos) para validar cumplimiento de políticas federales
    \item \textbf{Aprobar acuerdos como completados} una vez que haya validado las evidencias federales operativas requeridas
    \item \textbf{Consultar el historial completo} de todos los acuerdos federales operativos ya cumplidos por todas las organizaciones
    \item \textbf{Generar reportes ejecutivos operativos} de desempeño federal por organización
    \item \textbf{Configurar alertas} de acuerdos próximos a vencer que requieren supervisión federal operativa
    \item \textbf{Gestionar usuarios del equipo SADER} y permisos del sistema de supervisión federal
    \item \textbf{Coordinar con el Representante Estatal} para políticas federales
\end{itemize}

\section{FUNCIONES AVANZADAS EJECUTIVAS OPERATIVAS}

\subsection{Herramientas Disponibles para Supervisión Federal Operativa}
\textbf{Capacidades del sistema}: \textbf{Supervisión ejecutiva operativa completa} de todos los acuerdos federales operativos

\textbf{INFORMACIÓN DISPONIBLE - PROGRAMA CESO (SEGUIMIENTO DE ACUERDOS)}:
\begin{itemize}
    \item \textbf{Panel de Seguimiento de Acuerdos CESO}: Estado de todos los acuerdos del Consejo Estatal de Seguimiento Operativo
    \item \textbf{Lista de Acuerdos Pendientes}: Acuerdos específicos que requieren seguimiento y cumplimiento
    \item \textbf{Validación de Evidencias de Cumplimiento}: Herramienta para revisar documentos que comprueban el cumplimiento de acuerdos
    \item \textbf{Historial de Acuerdos}: Registro completo de todos los acuerdos CESO por organización participante
    \item \textbf{Métricas de Cumplimiento}: Análisis de cumplimiento de acuerdos por organización y período
    \item \textbf{NOTA IMPORTANTE}: Los sistemas operativos SINIIGA, SINIDA y REEMO son manejados directamente por CNOG-SINIIGA y otras instancias especializadas
\end{itemize}

\textbf{INFORMACIÓN DISPONIBLE - PROGRAMA APHIS-USDA (SEGUIMIENTO DE RECOMENDACIONES)}:
\begin{itemize}
    \item \textbf{Panel de Seguimiento TB}: Acuerdos específicos sobre recomendaciones críticas de tuberculosis bovina
    \item \textbf{Evidencias de Cumplimiento Requeridas}: Lista de documentación que demuestra el cumplimiento de acuerdos sobre recomendaciones APHIS-USDA
    \item \textbf{Estado de Acuerdos Completados}: Seguimiento de acuerdos ya cumplidos sobre recomendaciones críticas
    \item \textbf{Reportes de Seguimiento}: Documentación del avance en el cumplimiento de recomendaciones
    \item \textbf{Coordinación para Certificación}: Seguimiento de acuerdos relacionados con el proceso de certificación (los sistemas de certificación son operados por SENASICA/APHIS-USDA)
\end{itemize}

\section{SEGURIDAD OPERATIVA FEDERAL}

\subsection{Protección de Datos Federales Operativos}
\begin{itemize}
    \item \textbf{Autenticación federal operativa segura} con sus credenciales personales ejecutivas
    \item \textbf{Cifrado de archivos} subidos como evidencia federal operativa
    \item \textbf{Backup automático} de todos los documentos federales operativos
    \item \textbf{Logs de acceso ejecutivos operativos} para auditorías federales operativas
    \item \textbf{Control de permisos} por nivel de supervisión federal operativa
\end{itemize}

\subsection{Buenas Prácticas de Seguridad Federal Operativa}
\begin{itemize}
    \item No comparta sus credenciales ejecutivas operativas con colegas federales
    \item Cierre sesión al terminar de trabajar con información federal operativa
    \item Use dispositivos seguros para revisar evidencias confidenciales federales operativas
    \item Reporte inmediatamente cualquier problema de acceso o seguridad federal operativo
\end{itemize}

\section{POSIBLES USOS DEL SISTEMA - CASOS PRÁCTICOS EJECUTIVOS OPERATIVOS}

\subsection{Rutina Federal Operativa de Trabajo Diaria}
\textbf{Flujo de Trabajo Ejecutivo Operativo Recomendado (20-25 minutos)}:

\begin{enumerate}
    \item \textbf{Revisar Flash Cards Ejecutivas Operativas}: ¿Hay acuerdos vencidos (rojos) federales operativos?
    \item \textbf{Clic en "Pendientes"}: Ver lista de supervisión federal operativa del día
    \item \textbf{Validar acuerdos}: Revisar evidencias de supervisión federal operativa y aprobar como completados
    \item \textbf{Verificar "En Progreso"}: ¿Hay acuerdos federales operativos que requieren seguimiento ejecutivo operativo?
    \item \textbf{Generar reportes}: Crear reportes ejecutivos operativos de desempeño federal si es necesario
    \item \textbf{Coordinar con equipo SADER}: ¿Necesita gestionar algún acuerdo con el equipo federal?
\end{enumerate}

\subsection{Tipos de Evidencias Típicas para Supervisión Federal Operativa SADER}
\begin{itemize}
    \item \textbf{Reportes Ejecutivos Operativos}: Documentación de actividades de supervisión federal operativa
    \item \textbf{Fotografías Federales Operativas}: Evidencia visual de implementación de políticas federales operativas
    \item \textbf{Oficios Federales Operativos}: Comunicaciones oficiales entre organismos federales operativos y estatales
    \item \textbf{Listas de Inspección Federal Operativa}: Documentos de participación y cumplimiento en programas federales operativos
    \item \textbf{Informes de Certificación Operativos}: Documentación para procesos de certificación federal operativa
    \item \textbf{Correspondencia Internacional Operativa}: Documentos de comunicación operativa con APHIS-USDA
    \item \textbf{Reportes de Desempeño del Equipo}: Documentación del desempeño del equipo SADER
\end{itemize}

\section{SOPORTE TÉCNICO Y CAPACITACIÓN EJECUTIVA OPERATIVA}

\subsection{Contacto Inmediato}
\begin{tcolorbox}[colback=gobmx-light, colframe=gobmx-burgundy, boxrule=1pt]
\textbf{Responsable Técnico}: Sergio Muñoz de Alba Medrano \\
\textbf{Email}: smunoz.sader@gmail.com \\
\textbf{Teléfono}: +52 999 200 5550 \\
\textbf{Tiempo de Respuesta}: Inmediato para nivel ejecutivo operativo garantizado
\end{tcolorbox}

\subsection{Tipos de Soporte Disponible para Nivel Ejecutivo Operativo}
\begin{itemize}
    \item \textbf{Soporte técnico inmediato} para problemas de acceso o funcionamiento ejecutivo operativo
    \item \textbf{Capacitación ejecutiva operativa adicional} si requiere refuerzo en alguna función de supervisión
    \item \textbf{Consultoría operativa} para optimizar su flujo de supervisión federal operativa
    \item \textbf{Reportes especiales ejecutivos operativos} con métricas particulares federales operativas
    \item \textbf{Configuración personalizada} de alertas y notificaciones ejecutivas operativas
\end{itemize}

\section{BENEFICIOS PARA SU SUBDELEGACIÓN FEDERAL OPERATIVA}

\subsection{Organización Ejecutiva Federal Operativa}
\begin{itemize}
    \item \textbf{Visibilidad completa} de todo lo que cada organización debe hacer para supervisión federal operativa
    \item \textbf{Control de fechas límite} federales operativas para priorizar supervisión
    \item \textbf{Progreso consolidado} para evaluación ejecutiva operativa de desempeño federal
    \item \textbf{Historial ejecutivo operativo completo} de todo lo que se ha logrado federalmente a nivel operativo
\end{itemize}

\subsection{Eficiencia Operativa Federal}
\begin{itemize}
    \item \textbf{Supervisión rápida} de evidencias federales operativas desde cualquier dispositivo
    \item \textbf{Acceso móvil ejecutivo operativo} para supervisar trabajo federal operativo al momento
    \item \textbf{Enfoque ejecutivo operativo claro} en tareas prioritarias de supervisión federal operativa
    \item \textbf{Mejora continua} basada en métricas objetivas federales operativas
    \item \textbf{Reportes ejecutivos operativos} para demostrar desempeño federal operativo del equipo SADER
\end{itemize}

\section{PRIMERAS REFERENCIAS PARA COMENZAR A USAR EL SISTEMA}

\subsection{Impacto de Su Supervisión Federal Operativa}
\textbf{Contribución Federal Operativa}:

\begin{itemize}
    \item \textbf{Modernización federal operativa} liderada desde la subdelegación de SADER
    \item \textbf{Eficiencia mejorada} en supervisión y seguimiento federal operativo
    \item \textbf{Transparencia total} en procesos de supervisión federal operativa
    \item \textbf{Cooperación optimizada} entre organismos federales operativos, estatales y sector privado
    \item \textbf{Certificación internacional} del estado para exportación de ganado bovino desde supervisión federal operativa
\end{itemize}

\textbf{Desarrollo Profesional Ejecutivo Operativo}:

\begin{itemize}
    \item \textbf{Competencias digitales ejecutivas operativas} mejoradas para la supervisión federal operativa moderna
    \item \textbf{Métricas objetivas ejecutivas operativas} de desempeño para evaluaciones federales operativas
    \item \textbf{Experiencia práctica} con sistemas gubernamentales desde subdelegación federal operativa
    \item \textbf{Contribución documentada} a programas estratégicos federales operativos nacionales e internacionales
\end{itemize}

\vspace{1cm}

\begin{center}
\textcolor{gobmx-burgundy}{\textbf{Este manual ejecutivo presenta las funcionalidades completas del Centro de Consulta de Acuerdos Zoosanitarios disponibles para el Suplente del Presidente del CESO. El sistema está diseñado para facilitar el ejercicio de todas las atribuciones del Presidente del CESO cuando este se encuentre ausente, incluyendo la supervisión integral de acuerdos zoosanitarios del SINIIGA-SINIDA y la coordinación ejecutiva con APHIS-USDA.}}
\end{center}

\vfill

\begin{center}
\textbf{Abigail Estrada Estrada}\\
\textit{Subdelegado - SADER}

\vspace{0.5cm}

\textcolor{gobmx-gray}{\textbf{Por}: MVZ Sergio Muñoz de Alba Medrano}\\
\textcolor{gobmx-gray}{\textit{Desarrollador del Sistema}}\\
\textcolor{gobmx-gray}{\textbf{Fecha}: 10 de noviembre de 2025}\\
\textcolor{gobmx-gray}{\textbf{Versión del Sistema}: 3.0.0}

\vspace{0.5cm}

\textit{Centro de Consulta de Acuerdos Zoosanitarios}\\
\textit{Gobierno de México - Estado de Yucatán}\\
\texttt{https://ceso-aphis-yuc.web.app}
\end{center}

\end{document}
