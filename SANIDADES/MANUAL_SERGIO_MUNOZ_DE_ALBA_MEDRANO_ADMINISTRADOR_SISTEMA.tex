\documentclass[11pt,letterpaper,oneside]{article}

% ==========================================
% PAQUETES ESENCIALES
% ==========================================
\usepackage[utf8]{inputenc}
\usepackage[spanish,es-tabla]{babel}
\usepackage[letterpaper,top=2.5cm,bottom=2cm,left=2.5cm,right=2.5cm]{geometry}
\usepackage{graphicx}
\usepackage{fancyhdr}
\usepackage{xcolor}
\usepackage{titlesec}
\usepackage{enumitem}
\usepackage{array}
\usepackage{longtable}
\usepackage{booktabs}
\usepackage{multicol}
\usepackage{microtype}
\usepackage{hyperref}
\usepackage{lastpage}
\usepackage{tikz}
\usepackage{tcolorbox}
\usepackage{setspace}
\usepackage{caption}

% ==========================================
% COLORES OFICIALES DEL GOBIERNO
% ==========================================
\definecolor{gobmx-burgundy}{RGB}{139,21,56}    % Color principal gobierno
\definecolor{gobmx-gold}{RGB}{188,149,92}       % Dorado oficial
\definecolor{gobmx-gray}{RGB}{84,84,84}         % Gris texto
\definecolor{gobmx-lightgray}{RGB}{248,249,250} % Gris claro fondos
\definecolor{sader-green}{RGB}{34,139,34}       % Verde SADER

% ==========================================
% CONFIGURACIÓN DE HYPERLINKS
% ==========================================
\hypersetup{
    colorlinks=true,
    linkcolor=gobmx-burgundy,
    urlcolor=gobmx-burgundy,
    citecolor=gobmx-burgundy,
    pdfauthor={Gobierno de México - SADER},
    pdftitle={Manual Administrador - MVZ Sergio Muñoz de Alba Medrano},
    pdfsubject={Manual de Administrador del Sistema - Centro de Consulta de Acuerdos Zoosanitarios},
    pdfkeywords={SADER, SENASICA, CESO, APHIS, Administrador, Sistema, Gobierno},
}

% ==========================================
% CONFIGURACIÓN DE HEADERS Y FOOTERS
% ==========================================
\pagestyle{fancy}
\fancyhf{}

% Header izquierdo con título del manual
\fancyhead[L]{
    \footnotesize\color{gobmx-burgundy}
    \textbf{Manual SADER - Sistema Administrador}\\
    \textcolor{gobmx-gray}{Sergio Muñoz de Alba Medrano - Desarrollador}
}

% Footer izquierdo - información oficial (texto más pequeño)
\fancyfoot[L]{
    \scriptsize\color{gobmx-gray}
    \textbf{Centro de Consulta de Acuerdos Zoosanitarios}\\
    Secretaría de Agricultura y Desarrollo Rural
}

% Footer centro - numeración de páginas (texto más pequeño)
\fancyfoot[C]{
    \scriptsize\color{gobmx-gray}
    Página \thepage\ de \pageref{LastPage}
}

% Footer derecho - version stamp debajo de numeración (texto más pequeño)
\fancyfoot[R]{
    \scriptsize\color{gobmx-gray}
    Estado de Yucatán\\
    Versión 3.0.0 • \today
}

% Header derecho con logo SADER
\fancyhead[R]{
    \includegraphics[width=3.5cm]{images/sader-logo.png}
}

% ==========================================
% CONFIGURACIÓN DE TÍTULOS
% ==========================================
\titleformat{\section}
{\Large\bfseries\sffamily\color{gobmx-burgundy}}
{\thesection}{1em}{}

\titleformat{\subsection}
{\large\bfseries\sffamily\color{gobmx-gray}}
{\thesubsection}{1em}{}

% ==========================================
% DOCUMENTO PRINCIPAL
% ==========================================
\begin{document}

% ==========================================
% PORTADA OFICIAL
% ==========================================
\begin{titlepage}
\centering

% Logo oficial centrado
\vspace{1cm}
\includegraphics[width=8cm]{./images/sader-logo.png}

\vspace{2cm}

% Título principal
{\Huge\bfseries\color{gobmx-burgundy}
    \textbf{Manual del Administrador}\\
    \vspace{0.5cm}
    \textbf{Sistema de Consulta}\\
    \vspace{0.5cm}
    \textbf{Acuerdos Zoosanitarios}\\
}

\vspace{1cm}

% Subtítulo descriptivo
{\LARGE\color{gobmx-gray}
    Manual Operativo Completo\\
    \vspace{0.3cm}
    {\LARGE\color{gobmx-gray}ADMINISTRADOR DEL SISTEMA}\\
}

\vspace{1.5cm}

% Información del usuario específico
{\Large\color{gobmx-burgundy}
    \textbf{MVZ SERGIO MUÑOZ DE ALBA MEDRANO}\\
}
\vspace{0.3cm}
{\large\color{gobmx-gray}
    Prestador de Servicios Independiente - Administrador del Sistema\\
}
\vspace{0.3cm}
{\large\color{gobmx-gray}
    Secretaría de Agricultura y Desarrollo Rural (SADER)\\
}

\vspace{2cm}

% Información técnica
\begin{tcolorbox}[colback=gobmx-lightgray,colframe=gobmx-burgundy,boxrule=2pt,arc=5pt]
\centering
{\large\bfseries INFORMACIÓN TÉCNICA DEL SISTEMA}\\
\vspace{0.5cm}
\begin{tabular}{ll}
\textbf{Plataforma:} & \texttt{https://ceso-aphis-yuc.web.app} \\
\textbf{Tecnología:} & Firebase • React • Responsive Design \\
\textbf{Nivel de Acceso:} & Administrador Completo \\
\textbf{Organizaciones:} & CESO • APHIS-USDA/SENASICA \\
\textbf{Permisos:} & Admin • Edit • Upload • Download • View
\end{tabular}
\end{tcolorbox}

\vfill

% Pie de página oficial
{\footnotesize
    \textcolor{gobmx-gray}{\textbf{Gobierno de México}}\\
    \textcolor{gobmx-gray}{Centro de Consulta de Acuerdos Zoosanitarios}\\
    \textcolor{gobmx-gray}{Estado de Yucatán • Noviembre 2025}
}

\end{titlepage}

% ==========================================
% TABLA DE CONTENIDOS
% ==========================================
\tableofcontents
\newpage

% ==========================================
% INFORMACIÓN DEL ADMINISTRADOR
% ==========================================
\section{INFORMACIÓN DEL ADMINISTRADOR DEL SISTEMA}

\textbf{NOMBRE}: MVZ Sergio Muñoz de Alba Medrano \\
\textbf{CARGO}: Prestador de Servicios Independiente - Administrador del Sistema \\
\textbf{DEPENDENCIA}: SADER Yucatán \\
\textbf{FECHA}: Noviembre 26, 2025

\textbf{ROL EN EL SISTEMA}: Administrador Principal con acceso completo a todas las funcionalidades \\
Usuario: smunoz.sader\@gmail.com \\
Contraseña: MunozSader\#99 \\
Nivel: Administrador

\textbf{RESPONSABILIDADES COMO ADMINISTRADOR DEL SISTEMA}:
Como desarrollador y administrador principal del sistema, tiene las siguientes responsabilidades:
\begin{itemize}
    \item Administración completa del sistema Firebase (Firestore, Storage, Authentication, Hosting)
    \item Gestión de usuarios y permisos de acceso para todos los niveles (Federal, Estatal, Comité, UGRY)
    \item Supervisión técnica de la integridad de los datos y evidencias
    \item Configuración y mantenimiento de reglas de seguridad (Firestore Rules, Storage Rules)
    \item Soporte técnico especializado a todos los usuarios del sistema
    \item Desarrollo y implementación de nuevas funcionalidades
    \item Respaldo y recuperación de información crítica del sistema
    \item Coordinación técnica con todas las organizaciones participantes
    \item Generación de reportes técnicos y estadísticas del sistema
    \item Mantenimiento de la documentación técnica y manuales de usuario
\end{itemize}

\section{ARQUITECTURA Y ACCESO TÉCNICO DEL SISTEMA}

\subsection{Información Técnica de la Plataforma}
\textbf{URL de Acceso}: \texttt{https://ceso-aphis-yuc.web.app} \\
\textbf{Tecnología}: 
\begin{itemize}
    \item \textbf{Frontend}: React 18 + Vite 4.4.5, React Router v7.8.0
    \item \textbf{Backend}: Firebase (Firestore Database, Storage, Authentication, Hosting)
    \item \textbf{Diseño}: Bootstrap 5 + Framework GOB.mx v3 oficial
    \item \textbf{Deployment}: Firebase Hosting con CDN global
\end{itemize}

\subsection{Estructura de Datos}
El sistema maneja dos colecciones principales en Firestore:
\begin{itemize}
    \item \textbf{acuerdos-ceso}: Acuerdos del Consejo Estatal de Seguimiento Operativo (SINIIGA-SINIDA)
    \item \textbf{acuerdos-aphis}: Acuerdos del Grupo de Trabajo APHIS-USDA/SENASICA
    \item \textbf{Sub-colecciones de evidencias}: \texttt{\{collection\}/\{docId\}/evidencias}
\end{itemize}

\subsection{Sistema de Autenticación}
\begin{itemize}
    \item \textbf{Firebase Authentication}: Gestión de usuarios con email/password
    \item \textbf{Roles disponibles}: Administrador, Federal, Estatal, Comite, UGRY, Siniiga
    \item \textbf{Permisos granulares}: view, download, upload, edit, admin
    \item \textbf{Acceso público}: Consulta de acuerdos sin autenticación
    \item \textbf{Acceso autenticado}: Upload de evidencias y gestión de estados
\end{itemize}

\section{FUNCIONALIDADES DE ADMINISTRACIÓN}

\subsection{Panel de Control Administrativo}
Como administrador del sistema, usted tiene acceso a funcionalidades especiales:

\begin{enumerate}
    \item \textbf{Gestión de Usuarios}
    \begin{itemize}
        \item Crear, editar y eliminar cuentas de usuario
        \item Asignar y modificar roles y permisos
        \item Restablecer contraseñas de usuario
        \item Monitorear actividad de usuarios en el sistema
    \end{itemize}
    
    \item \textbf{Gestión de Contenido}
    \begin{itemize}
        \item Crear, editar y eliminar acuerdos en ambas colecciones
        \item Aprobar o rechazar evidencias subidas por usuarios
        \item Moderar contenido y mantener calidad de datos
        \item Exportar datos completos del sistema
    \end{itemize}
    
    \item \textbf{Configuración del Sistema}
    \begin{itemize}
        \item Configurar reglas de seguridad de Firestore y Storage
        \item Ajustar permisos de acceso por organización
        \item Configurar notificaciones y alertas del sistema
        \item Gestionar respaldos automatizados
    \end{itemize}
    
    \item \textbf{Monitoreo y Estadísticas}
    \begin{itemize}
        \item Dashboard completo con métricas del sistema
        \item Estadísticas de uso por organización y usuario
        \item Reportes de actividad y rendimiento
        \item Alertas de seguridad y errores del sistema
    \end{itemize}
\end{enumerate}

\subsection{Acceso a Firebase Console}
Para administración avanzada, utilice Firebase Console:
\begin{itemize}
    \item \textbf{URL}: \texttt{https://console.firebase.google.com}
    \item \textbf{Proyecto}: \texttt{tb-yucatan}
    \item \textbf{Acceso}: Con su cuenta de Google asociada al proyecto
\end{itemize}

\section{PROCEDIMIENTOS DE ADMINISTRACIÓN}

\subsection{Gestión de Usuarios Nuevos}
\textbf{Proceso para agregar nuevos usuarios al sistema}:
\begin{enumerate}
    \item Acceder al panel de administración en la plataforma
    \item Navegar a "Gestión de Usuarios" > "Nuevo Usuario"
    \item Completar información del usuario:
    \begin{itemize}
        \item Nombre completo y cargo
        \item Organización (SADER, SEDER, SENASICA, CEFPPY, SINIIGA, UGRY, UGROY)
        \item Email institucional
        \item Nivel de acceso (Federal, Estatal, Comité, etc.)
        \item Permisos específicos según rol
    \end{itemize}
    \item Generar contraseña temporal segura
    \item Enviar credenciales al usuario de forma segura
    \item Solicitar cambio de contraseña en el primer acceso
\end{enumerate}

\subsection{Mantenimiento de Datos}
\textbf{Rutina de mantenimiento recomendada}:
\begin{enumerate}
    \item \textbf{Respaldos diarios}: Exportar datos críticos automáticamente
    \item \textbf{Limpieza semanal}: Revisar y limpiar evidencias no válidas
    \item \textbf{Auditoría mensual}: Revisar logs de acceso y actividad de usuarios
    \item \textbf{Actualización trimestral}: Verificar y actualizar reglas de seguridad
\end{enumerate}

\subsection{Gestión de Evidencias}
\textbf{Supervisión de evidencias subidas}:
\begin{enumerate}
    \item Monitorear evidencias pendientes de aprobación
    \item Verificar calidad y relevancia de los documentos
    \item Aprobar evidencias válidas o solicitar correcciones
    \item Mantener organización de archivos en Firebase Storage
    \item Generar reportes de evidencias por período y organización
\end{enumerate}

\section{CONFIGURACIONES AVANZADAS}

\subsection{Firestore Security Rules}
\textbf{Configuración actual de reglas de seguridad}:
\begin{verbatim}
// Permitir lectura pública de acuerdos
match /{collection}/{document} {
  allow read: if true;
  allow create, update: if request.auth != null;
}

// Evidencias requieren autenticación
match /{collection}/{docId}/evidencias/{evidId} {
  allow read: if true;
  allow write: if request.auth != null;
}
\end{verbatim}

\subsection{Storage Security Rules}
\textbf{Configuración para archivos de evidencias}:
\begin{verbatim}
// Archivos públicos para lectura, auth para escritura
match /acuerdos-{source}/{docId}/{fileName} {
  allow read: if true;
  allow write: if request.auth != null
    && request.resource.size < 20 * 1024 * 1024; // 20MB limit
}
\end{verbatim}

\section{SOPORTE TÉCNICO Y CONTACTOS}

\subsection{Información de Contacto del Desarrollador}
\textbf{MVZ Sergio Muñoz de Alba Medrano}
\begin{itemize}
    \item \textbf{Email}: smunoz.sader\@gmail.com
    \item \textbf{Teléfono}: +52 999 200 5550
    \item \textbf{Horario de soporte}: Lunes a Viernes, 8:00 AM - 6:00 PM
    \item \textbf{Soporte de emergencia}: Disponible 24/7 para incidencias críticas
\end{itemize}

\subsection{Escalación de Problemas Técnicos}
\textbf{Procedimiento para problemas críticos}:
\begin{enumerate}
    \item \textbf{Nivel 1}: Problemas de usuario - Soporte directo del administrador
    \item \textbf{Nivel 2}: Problemas de sistema - Revisión de Firebase Console
    \item \textbf{Nivel 3}: Problemas de infraestructura - Contacto con Google Firebase Support
    \item \textbf{Nivel 4}: Problemas críticos - Activación de plan de contingencia
\end{enumerate}

\section{ACCESO MÓVIL PARA ADMINISTRACIÓN}

\subsection{Administración desde Dispositivos Móviles}
El sistema es completamente responsivo y permite administración desde:
\begin{itemize}
    \item \textbf{Smartphones}: iPhone, Android con navegadores modernos
    \item \textbf{Tablets}: iPad, Android tablets
    \item \textbf{Laptops}: Acceso completo desde cualquier computadora portátil
    \item \textbf{URL}: La misma - \texttt{https://ceso-aphis-yuc.web.app}
    \item \textbf{Credenciales}: Las mismas que en computadora de escritorio
\end{itemize}

\subsection{Funcionalidades Móviles de Administración}
\begin{itemize}
    \item Dashboard de administración adaptativo
    \item Gestión de usuarios desde dispositivos móviles
    \item Aprobación de evidencias en tiempo real
    \item Notificaciones push para alertas críticas
    \item Acceso a Firebase Console desde navegadores móviles
\end{itemize}

\section{DESARROLLO Y MANTENIMIENTO FUTURO}

\subsection{Roadmap de Funcionalidades}
\textbf{Desarrollos planificados para próximas versiones}:
\begin{enumerate}
    \item \textbf{v3.1.0}: Sistema de notificaciones automáticas por email
    \item \textbf{v3.2.0}: Dashboard avanzado con gráficos interactivos
    \item \textbf{v3.3.0}: API REST para integración con otros sistemas
    \item \textbf{v3.4.0}: Sistema de firma digital para acuerdos
    \item \textbf{v4.0.0}: Migración a Firebase v9 modular SDK
\end{enumerate}

\subsection{Consideraciones de Escalabilidad}
\begin{itemize}
    \item \textbf{Usuarios concurrentes}: Actualmente soporta hasta 1,000 usuarios simultáneos
    \item \textbf{Almacenamiento}: Límite actual de 10GB, expandible según necesidades
    \item \textbf{Tráfico}: Optimizado para uso gubernamental con CDN global
    \item \textbf{Backup}: Respaldos automáticos diarios con retención de 30 días
\end{itemize}

\section{CONCLUSIÓN}

\textcolor{gobmx-burgundy}{\textbf{Este manual presenta las funcionalidades administrativas completas del Centro de Consulta de Acuerdos Zoosanitarios. Como administrador principal del sistema, usted tiene la responsabilidad de mantener la integridad, seguridad y disponibilidad de la plataforma para todos los usuarios de las organizaciones participantes (SADER, SEDER, SENASICA, CEFPPY, SINIIGA, UGRY, UGROY) en los programas CESO y APHIS-USDA/SENASICA. El sistema está diseñado para crecer y adaptarse a las necesidades futuras de la sanidad animal en Yucatán, manteniendo siempre los más altos estándares de seguridad y cumplimiento normativo gubernamental.}}

\vspace{1cm}

\textit{Administrador del Sistema}

\textbf{MVZ Sergio Muñoz de Alba Medrano}\\
\textit{Prestador de Servicios Independiente}\\
\textit{Secretaría de Agricultura y Desarrollo Rural (SADER)}\\
\textit{Estado de Yucatán}

\end{document}